\documentclass{article}
\usepackage[fleqn]{amsmath}
\usepackage{scrextend}
\usepackage{amsmath,amssymb}
\changefontsizes[12pt]{8pt}
\usepackage[a4paper, left=0.7in,right=0.7in,top=1in,bottom=1in]{geometry}
\pagenumbering{gobble}
\usepackage{fancyhdr}
\renewcommand{\headrulewidth}{0pt}
\pagestyle{fancy}
\lfoot{\textcopyright\, One Maths Limited}
\rfoot{}
\begin{document}
\noindent\Huge{\textbf{Edexcel A Level Maths}}\\[5pt]
\noindent\large{Core 1, Core 2, Core 3, Core 4 and two units from (Decision 1, Decision 2, Mechanics 1, Mechanics 2, Statistics 1 and Statistics 2).  Each unit is out of 100 UMS, giving a total of 600 UMS.  480 and above is grade A.  To obtain an A*, total score must be 480 or more, and total of C3 and C4 must be 180 or more.}\\[20pt]
\noindent\huge{\textbf{Unit 1 Core 1}}\\[18pt]
\noindent\huge{\textbf{Chapter 1 Sequence and Series 1}}\\[15pt]
\noindent\large{\textbf{Lesson 1 Sequence and Summation}}\\[12pt]
\noindent\textbf{Question 1}\hspace{20pt}Experience: 10\hspace{20pt}Order: \hspace{20pt}Level: \hspace{20pt}Question-ID: 26\\[2pt]
A sequence is defined by $a_n=3n^2-4$, find the value of $a_2$.\\[4pt]
\noindent\textbf{Solution 1}\\[2pt]
\\[-35pt]\begin{align*}
a_2=3(2)^2-4=8\\[2pt]
\end{align*}
Choice 1: \hspace{20pt}$a_2=-1$\hspace{20pt}false\\
Choice 2: \hspace{20pt}$a_2=23$\hspace{20pt}false\\
Choice 3: \hspace{20pt}$a_2=5$\hspace{20pt}false\\
Choice 4: \hspace{20pt}$a_2=10$\hspace{20pt}false\\
Choice 5: \hspace{20pt}$a_2=8$\hspace{20pt}true\\
\\[4pt]
\noindent\textbf{Question 2}\hspace{20pt}Experience: 25\hspace{20pt}Order: \hspace{20pt}Level: \hspace{20pt}Question-ID: 30\\[2pt]
A sequence is defined by $x_n=3n^2-5n+2$, find the value of $n$ such that $x_n=14$.\\[4pt]
\noindent\textbf{Solution 2}\\[2pt]
\\[-35pt]\begin{align*}
x_n=3n^2-5n+2&=14\\[2pt]
3n^2-5n-12&=0 \hspace{20pt} S=-5 \quad P=-36\\[2pt]
\left(n+\displaystyle\frac{4}{3}\right)(n-3)&=0 \hspace{19pt} (4,-9) \quad \left(\displaystyle\frac{4}{3},-3\right)\\[2pt]
n&=3\\
\end{align*}
Choice 1: \hspace{20pt}$n=4$\hspace{20pt}false\\
Choice 2: \hspace{20pt}$n=2$\hspace{20pt}false\\
Choice 3: \hspace{20pt}$n=5$\hspace{20pt}false\\
Choice 4: \hspace{20pt}$n=6$\hspace{20pt}false\\
Choice 5: \hspace{20pt}$n=3$\hspace{20pt}true\\
\\[4pt]
\noindent\textbf{Question 3}\hspace{20pt}Experience: 15\hspace{20pt}Order: \hspace{20pt}Level: \hspace{20pt}Question-ID: 27\\[2pt]
A sequence is defined by $x_n=6n-3$, find the value of $x_3$ and $x_5$.\\[4pt]
\noindent\textbf{Solution 3}\\[2pt]
\\[-35pt]\begin{align*}
x_3=6(3)-3=15\\[2pt]
x_5=6(5)-3=27\\[2pt]
\end{align*}
Choice 1: \hspace{20pt}$x_3=3 \,\, x_5=15$\hspace{20pt}false\\
Choice 2: \hspace{20pt}$x_3=9 \,\, x_5=21$\hspace{20pt}false\\
Choice 3: \hspace{20pt}$x_3=15 \,\, x_5=21$\hspace{20pt}false\\
Choice 4: \hspace{20pt}$x_3=3 \,\, x_5=21$\hspace{20pt}false\\
Choice 5: \hspace{20pt}$x_3=15 \,\, x_5=27$\hspace{20pt}true\\
\\[4pt]
\noindent\textbf{Question 4}\hspace{20pt}Experience: 15\hspace{20pt}Order: \hspace{20pt}Level: \hspace{20pt}Question-ID: 31\\[2pt]
A sequence is defined by $u_n=an-b$, find the sum of the first four terms in terms of $a$ and $b$.\\[4pt]
\noindent\textbf{Solution 4}\\[2pt]
\\[-35pt]\begin{align*}
u_1+u_2+u_3+u_4=(a-b)+(2a-b)+(3a-b)+(4a-b)=10a-4b
\end{align*}
Choice 1: \hspace{20pt}$10a-6b$\hspace{20pt}false\\
Choice 2: \hspace{20pt}$6a-4b$\hspace{20pt}false\\
Choice 3: \hspace{20pt}$6a-6b$\hspace{20pt}false\\
Choice 4: \hspace{20pt}$10a-8b$\hspace{20pt}false\\
Choice 5: \hspace{20pt}$10a-4b$\hspace{20pt}true\\
\\[4pt]
\noindent\textbf{Question 5}\hspace{20pt}Experience: 15\hspace{20pt}Order: \hspace{20pt}Level: \hspace{20pt}Question-ID: 32\\[2pt]
A sequence is defined by $x_n=an^2-4$, find the sum of the first three terms in terms of $a$.\\[4pt]
\noindent\textbf{Solution 5}\\[2pt]
\\[-35pt]\begin{align*}
x_1+x_2+x_3=(a-4)+(4a-4)+(9a-4)=14a-12
\end{align*}
Choice 1: \hspace{20pt}$14a-8$\hspace{20pt}false\\
Choice 2: \hspace{20pt}$5a-12$\hspace{20pt}false\\
Choice 3: \hspace{20pt}$5a-8$\hspace{20pt}false\\
Choice 4: \hspace{20pt}$5a-14$\hspace{20pt}false\\
Choice 5: \hspace{20pt}$14a-12$\hspace{20pt}true\\
\\[4pt]
\noindent\textbf{Question 6}\hspace{20pt}Experience: 25\hspace{20pt}Order: \hspace{20pt}Level: \hspace{20pt}Question-ID: 33\\[2pt]
A sequence is defined by $y_n=an^2+bn+c$, find the sum of the first three terms in terms of $a,b$ and $c$.\\[4pt]
\noindent\textbf{Solution 6}\\[2pt]
\\[-35pt]\begin{align*}
y_1+y_2+y_3=(a+b+c)+(4a+2b+c)+(9a+3b+c)=14a+6b+3c
\end{align*}
wrong choice\\[4pt]
Choice 1: \hspace{20pt}$14a-6b+3c$\hspace{20pt}false\\
Choice 2: \hspace{20pt}$6a+6b+3c$\hspace{20pt}false\\
Choice 3: \hspace{20pt}$6a+4b+2c$\hspace{20pt}false\\
Choice 4: \hspace{20pt}$6a+4b+3c$\hspace{20pt}false\\
Choice 5: \hspace{20pt}$14a+6b+3c$\hspace{20pt}true\\
\\[4pt]
\noindent\textbf{Question 7}\hspace{20pt}Experience: 10\hspace{20pt}Order: \hspace{20pt}Level: \hspace{20pt}Question-ID: 34\\[2pt]
A sequence is defined by $x_n=4n-b$, find the third term in terms of $b$ .\\[4pt]
\noindent\textbf{Solution 7}\\[2pt]
\\[-35pt]\begin{align*}
x_3&=4(3)-b\\
x_3&=12-b
\end{align*}
Choice 1: \hspace{20pt}$x_3=12-3b$\hspace{20pt}false\\
Choice 2: \hspace{20pt}$x_3=6-b$\hspace{20pt}false\\
Choice 3: \hspace{20pt}$x_3=6-3b$\hspace{20pt}false\\
Choice 4: \hspace{20pt}$x_3=8-3b$\hspace{20pt}false\\
Choice 5: \hspace{20pt}$x_3=12-b$\hspace{20pt}true\\
\\[4pt]
\noindent\textbf{Question 8}\hspace{20pt}Experience: 10\hspace{20pt}Order: \hspace{20pt}Level: \hspace{20pt}Question-ID: 35\\[2pt]
A sequence is defined by $U_n=\displaystyle\frac{a}{n}+b$, find the fourth term in terms of $a$ and $b$ .\\[4pt]
\noindent\textbf{Solution 8}\\[2pt]
\\[-35pt]\begin{align*}
U_4=\displaystyle\frac{a}{4}+b
\end{align*}
Choice 1: \hspace{20pt}$U_4=\displaystyle\frac{a+b}{4}$\hspace{20pt}false\\
Choice 2: \hspace{20pt}$U_4=\displaystyle\frac{a}{4}+4b$\hspace{20pt}false\\
Choice 3: \hspace{20pt}$U_4=\displaystyle\frac{a}{4}+2b$\hspace{20pt}false\\
Choice 4: \hspace{20pt}$U_4=\displaystyle\frac{a}{4}+b$\hspace{20pt}true\\
Choice 5: \hspace{20pt}$U_4=\displaystyle\frac{a+4b}{8}$\hspace{20pt}false\\
\\[4pt]
\noindent\textbf{Question 9}\hspace{20pt}Experience: 15\hspace{20pt}Order: \hspace{20pt}Level: \hspace{20pt}Question-ID: 36\\[2pt]
A sequence is defined by $y_n=\displaystyle\frac{a-3b}{n^2}$, find the fifth term in terms of $a$ and $b$ .\\[4pt]
\noindent\textbf{Solution 9}\\[2pt]
\\[-35pt]\begin{align*}
y_5&=\displaystyle\frac{a-3b}{(5)^2}\\[2pt]
y_5&=\displaystyle\frac{a-3b}{25}\\[2pt]
\end{align*}
Choice 1: \hspace{20pt}$y_5=\displaystyle\frac{5a-3b}{25}$\hspace{20pt}false\\
Choice 2: \hspace{20pt}$y_5=\displaystyle\frac{5a-3b}{16}$\hspace{20pt}false\\
Choice 3: \hspace{20pt}$y_5=\displaystyle\frac{a-3b}{16}$\hspace{20pt}false\\
Choice 4: \hspace{20pt}$y_5=\displaystyle\frac{5a-b}{25}$\hspace{20pt}false\\
Choice 5: \hspace{20pt}$y_5=\displaystyle\frac{a-3b}{25}$\hspace{20pt}true\\
\\[4pt]
\noindent\textbf{Question 10}\hspace{20pt}Experience: 50\hspace{20pt}Order: \hspace{20pt}Level: \hspace{20pt}Question-ID: 37\\[2pt]
A sequence is defined by $U_n=an+2b$, given the Sum of the first four terms is $26$ and the fifth term is $9$, find the values of $a$ and $b$.\\[4pt]
\noindent\textbf{Solution 10}\\[2pt]
\\[-35pt]\begin{align*}
S_4&=(a+2b)+(2a+2b)+(3a+2b)+(4a+2b)\\[2pt]
S_4&=10a+8b\hspace{20pt} S_4=26\\[2pt]
10a+8b&=26\\[2pt]
5a+4b&=13\quad (1)\\[12pt]
U_5&=5a+2b\hspace{20pt}U_5=9\\[2pt]
5a+2b&=9\quad (2)\\[12pt]
(1)-(2)\quad 5a+4b-(5a+2b)&=13-9\\[2pt]
2b&=4\\[2pt]
b&=2\\[12pt]
\text{sub into}\quad (2) \quad 5a+2(2)&=9\\[2pt]
5a&=5\\[2pt]
a&=1
\end{align*}
Choice 1: \hspace{20pt}$a=1\quad b=3$\hspace{20pt}false\\
Choice 2: \hspace{20pt}$a=2 \quad b=3$\hspace{20pt}false\\
Choice 3: \hspace{20pt}$a=2 \quad b=2$\hspace{20pt}false\\
Choice 4: \hspace{20pt}$a=3 \quad b=2$\hspace{20pt}false\\
Choice 5: \hspace{20pt}$a=1 \quad b=2$\hspace{20pt}true\\
\\[4pt]
\noindent\textbf{Question 11}\hspace{20pt}Experience: 25\hspace{20pt}Order: \hspace{20pt}Level: \hspace{20pt}Question-ID: 40\\[2pt]
A sequence is defined by $U_{n+1}=U_n-4, \quad U_1=20$, find the values of $U_2,U_3$ and $U_4$.\\[4pt]
\noindent\textbf{Solution 11}\\[2pt]
\\[-35pt]\begin{align*}
U_2&=U_1-4=20-4=16\\[2pt]
U_3&=U_2-4=16-4=12\\[2pt]
U_4&=U_3-4=12-4=8
\end{align*}
Choice 1: \hspace{20pt}$U_2=16 \,\, U_3=12 \,\, U_4=4$\hspace{20pt}false\\
Choice 2: \hspace{20pt}$U_2=12 \,\, U_3=8 \,\, U_4=4$\hspace{20pt}false\\
Choice 3: \hspace{20pt}$U_2=12 \,\, U_3=4 \,\, U_4=0$\hspace{20pt}false\\
Choice 4: \hspace{20pt}$U_2=16 \,\, U_3=4 \,\, U_4=4$\hspace{20pt}false\\
Choice 5: \hspace{20pt}$U_2=16 \,\, U_3=12 \,\, U_4=8$\hspace{20pt}true\\
\\[4pt]
\noindent\textbf{Question 12}\hspace{20pt}Experience: 25\hspace{20pt}Order: \hspace{20pt}Level: \hspace{20pt}Question-ID: 41\\[2pt]
A sequence is defined by $X_{n+1}=X_n+5, \quad X_4=17$, find the values of $X_1,X_2$ and $X_3$.\\[4pt]
\noindent\textbf{Solution 12}\\[2pt]
\\[-35pt]\begin{align*}
X_4&=X_3+5\\[2pt]
17&=X_3+5\\[2pt]
X_3&=12\\[12pt]
X_3&=X_2+5\\[2pt]
12&=X_2+5\\[2pt]
X_2&=7\\[12pt]
X_2&=X_1+5\\[2pt]
7&=X_1+5\\[2pt]
X_1&=2\\[12pt]
\end{align*}
Choice 1: \hspace{20pt}$X_1=2 \,\, X_2=6 \,\, X_3=12$\hspace{20pt}false\\
Choice 2: \hspace{20pt}$X_1=5 \,\, X_2=8 \,\, X_3=11$\hspace{20pt}false\\
Choice 3: \hspace{20pt}$X_1=5 \,\, X_2=7 \,\, X_3=9$\hspace{20pt}false\\
Choice 4: \hspace{20pt}$X_1=5 \,\, X_2=6 \,\, X_3=10$\hspace{20pt}false\\
Choice 5: \hspace{20pt}$X_1=2 \,\, X_2=7 \,\, X_3=12$\hspace{20pt}true\\
\\[4pt]
\noindent\textbf{Question 13}\hspace{20pt}Experience: 30\hspace{20pt}Order: \hspace{20pt}Level: \hspace{20pt}Question-ID: 42\\[2pt]
A sequence is defined by $a_{n+1}=(a_n)^2-4, \quad a_1=2$, find the values of $a_2,a_3$ and $a_4$.\\[4pt]
\noindent\textbf{Solution 13}\\[2pt]
\\[-35pt]\begin{align*}
a_2&=(a_1)^2-4=4-4=0\\[2pt]
a_3&=(a_2)^2-4=0-4=-4\\[2pt]
a_4&=(a_3)^2-4=(-4)^2-4=16-4=12
\end{align*}
Choice 1: \hspace{20pt}$a_2=0 \,\, a_3=4 \,\, a_4=12 $\hspace{20pt}false\\
Choice 2: \hspace{20pt}$a_2=4 \,\, a_3=-4 \,\, a_4=8 $\hspace{20pt}false\\
Choice 3: \hspace{20pt}$a_2=4 \,\, a_3=8 \,\, a_4=-8 $\hspace{20pt}false\\
Choice 4: \hspace{20pt}$a_2=0 \,\, a_3=8 \,\, a_4=12 $\hspace{20pt}false\\
Choice 5: \hspace{20pt}$a_2=0 \,\, a_3=-4 \,\, a_4=12 $\hspace{20pt}true\\
\\[4pt]
\noindent\textbf{Question 14}\hspace{20pt}Experience: 25\hspace{20pt}Order: \hspace{20pt}Level: \hspace{20pt}Question-ID: 43\\[2pt]
A sequence is defined by $y_{n+2}=3y_{n+1}-y_n, \quad y_1=3,y_2=2$, find the values of $y_3,y_4$ and $y_5$.\\[4pt]
\noindent\textbf{Solution 14}\\[2pt]
\\[-35pt]\begin{align*}
y_3&=3(y_2)-y_1=3(2)-3=3\\[2pt]
y_4&=3(y_3)-y_2=3(3)-2=7\\[2pt]
y_5&=3(y_4)-y_3=3(7)-3=18\\[-30pt]
\end{align*}
Choice 1: \hspace{20pt}$y_3=3 \,\, y_4=5 \,\, y_5=18 $\hspace{20pt}false\\
Choice 2: \hspace{20pt}$y_3=7 \,\, y_4=4 \,\, y_5=5 $\hspace{20pt}false\\
Choice 3: \hspace{20pt}$y_3=7 \,\, y_4=8 \,\, y_5=5 $\hspace{20pt}false\\
Choice 4: \hspace{20pt}$y_3=3 \,\, y_4=7 \,\, y_5=5 $\hspace{20pt}false\\
Choice 5: \hspace{20pt}$y_3=3 \,\, y_4=7 \,\, y_5=18 $\hspace{20pt}true\\
\\[4pt]
\noindent\textbf{Question 15}\hspace{20pt}Experience: 15\hspace{20pt}Order: \hspace{20pt}Level: \hspace{20pt}Question-ID: 46\\[2pt]
Calculate the following sum:
\begin{align*}
\sum_{r=2}^{5} (r-1)
\end{align*}
\noindent\textbf{Solution 15}\\[2pt]
\\[-35pt]\begin{align*}
\sum_{r=2}^{5} (r-1)&=(2-1)+(3-1)+(4-1)+(5-1)\\[2pt]
&=1+2+3+4\\[2pt]
&=10
\end{align*}
Choice 1: \hspace{20pt}9\hspace{20pt}false\\
Choice 2: \hspace{20pt}8\hspace{20pt}false\\
Choice 3: \hspace{20pt}11\hspace{20pt}false\\
Choice 4: \hspace{20pt}12\hspace{20pt}false\\
Choice 5: \hspace{20pt}10\hspace{20pt}true\\
\\[4pt]
\noindent\textbf{Question 16}\hspace{20pt}Experience: 15\hspace{20pt}Order: \hspace{20pt}Level: \hspace{20pt}Question-ID: 47\\[2pt]
Calculate the following sum:
\begin{align*}
\sum_{r=4}^{8} (r^2-2r+1)
\end{align*}
\noindent\textbf{Solution 16}\\[2pt]
\\[-35pt]\begin{align*}
&\sum_{r=4}^{8} (r^2-2r+1)\\[2pt]
=\,\,&\sum_{r=4}^{8} (r-1)^2\\[2pt]
=\,\,&(4-1)^2+(5-1)^2+(6-1)^2+(7-1)^2+(8-1)^2\\[2pt]
=\,\,&9+16+25+36+49\\[2pt]
=\,\,&135
\end{align*}
Choice 1: \hspace{20pt}137\hspace{20pt}false\\
Choice 2: \hspace{20pt}128\hspace{20pt}false\\
Choice 3: \hspace{20pt}130\hspace{20pt}false\\
Choice 4: \hspace{20pt}136\hspace{20pt}false\\
Choice 5: \hspace{20pt}135\hspace{20pt}true\\
\\[4pt]
\noindent\textbf{Question 17}\hspace{20pt}Experience: 10\hspace{20pt}Order: \hspace{20pt}Level: \hspace{20pt}Question-ID: 28\\[2pt]
A sequence is defined by $X_n=2n-1$, find the value of $n$ such that $a_n=15$.\\[4pt]
\noindent\textbf{Solution 17}\\[2pt]
\\[-35pt]\begin{align*}
15&=2n-1\\[2pt]
n&=8
\end{align*}
Choice 1: \hspace{20pt}$n=6$\hspace{20pt}false\\
Choice 2: \hspace{20pt}$n=7$\hspace{20pt}false\\
Choice 3: \hspace{20pt}$n=3$\hspace{20pt}false\\
Choice 4: \hspace{20pt}$n=9$\hspace{20pt}false\\
Choice 5: \hspace{20pt}$n=8$\hspace{20pt}true\\
\\[4pt]
\noindent\textbf{Question 18}\hspace{20pt}Experience: 30\hspace{20pt}Order: \hspace{20pt}Level: \hspace{20pt}Question-ID: 53\\[2pt]
Calculate the following sum:
\begin{align*}
\sum_{r=5}^{9} U_r \hspace{20pt}U_r=3r^2+4
\end{align*}
\noindent\textbf{Solution 18}\\[2pt]
\\[-35pt]\begin{align*}
&\sum_{r=5}^{9} U_r\\[2pt]
=\,\,&\sum_{r=5}^{9} 3r^2+4\\[2pt]
=\,\,&(3(5)^2+4)+(3(6)^2+4)+(3(7)^2+4)+(3(8)^2+4)+(3(9)^2+4) \\[2pt]
=\,\,&785
\end{align*}
Choice 1: \hspace{20pt}795\hspace{20pt}false\\
Choice 2: \hspace{20pt}790\hspace{20pt}false\\
Choice 3: \hspace{20pt}780\hspace{20pt}false\\
Choice 4: \hspace{20pt}800\hspace{20pt}false\\
Choice 5: \hspace{20pt}785\hspace{20pt}true\\
\\[4pt]
\noindent\textbf{Question 19}\hspace{20pt}Experience: 30\hspace{20pt}Order: \hspace{20pt}Level: \hspace{20pt}Question-ID: 52\\[2pt]
Calculate the following sum:
\begin{align*}
\sum_{r=1}^{3} a_r \hspace{20pt}a_r=4r-1
\end{align*}
\noindent\textbf{Solution 19}\\[2pt]
\\[-35pt]\begin{align*}
&\sum_{r=1}^{3} a_r\\[2pt]
=\,\,&\sum_{r=1}^{3} 4r-1\\[2pt]
=\,\,&(4(1)-1)+(4(2)-1)+(4(3)-1) \\[2pt]
=\,\,&21
\end{align*}
Choice 1: \hspace{20pt}22\hspace{20pt}false\\
Choice 2: \hspace{20pt}19\hspace{20pt}false\\
Choice 3: \hspace{20pt}20\hspace{20pt}false\\
Choice 4: \hspace{20pt}18\hspace{20pt}false\\
Choice 5: \hspace{20pt}21\hspace{20pt}true\\
\\[4pt]
\noindent\textbf{Question 20}\hspace{20pt}Experience: 45\hspace{20pt}Order: \hspace{20pt}Level: \hspace{20pt}Question-ID: 54\\[2pt]
A sequence is defined by $U_{n+1}=3(U_n -1), U_1=2,$ find the following sum: $\displaystyle\sum_{2}^{4} (U_r+2)^2$\\[4pt]
\noindent\textbf{Solution 20}\\[2pt]
\\[-35pt]\begin{align*}
U_2&=3(2-1)\\[2pt]
U_2&=3\\[12pt]
U_3&=3(3-1)\\[2pt]
U_3&=6\\[12pt]
U_4&=3(6-1)\\[2pt]
U_4&=15\\[12pt]
\displaystyle\sum_{2}^{4} (U_r+2)^2&=(U_2+2)^2+(U_3+2)^2+(U_4+2)^2\\[2pt]
&=(3+2)^2+(6+2)^2+(15+2)^2\\[2pt]
&=5^2+8^2+17^2\\[2pt]
&=378
\end{align*}
Choice 1: \hspace{20pt}380\hspace{20pt}false\\
Choice 2: \hspace{20pt}377\hspace{20pt}false\\
Choice 3: \hspace{20pt}379\hspace{20pt}false\\
Choice 4: \hspace{20pt}381\hspace{20pt}false\\
Choice 5: \hspace{20pt}378\hspace{20pt}true\\
\\[4pt]
\noindent\textbf{Question 21}\hspace{20pt}Experience: 15\hspace{20pt}Order: \hspace{20pt}Level: \hspace{20pt}Question-ID: 48\\[2pt]
Calculate the following sum:
\begin{align*}
\sum_{r=1}^{4} (2r+4)
\end{align*}
\noindent\textbf{Solution 21}\\[2pt]
\\[-35pt]\begin{align*}
&\sum_{r=1}^{4} (2r+4)\\[2pt]
=\,\,&\sum_{r=1}^{4} 2(r+2)\\[2pt]
=\,\,&2\sum_{r=1}^{4} (r+2)\\[2pt]
=\,\,&2[(1+2)+(2+2)+(3+2)+(4+2)]\\[2pt]
=\,\,&2(3+4+5+6)\\[2pt]
=\,\,&36
\end{align*}
Choice 1: \hspace{20pt}37\hspace{20pt}false\\
Choice 2: \hspace{20pt}35\hspace{20pt}false\\
Choice 3: \hspace{20pt}34\hspace{20pt}false\\
Choice 4: \hspace{20pt}33\hspace{20pt}false\\
Choice 5: \hspace{20pt}36\hspace{20pt}true\\
\\[4pt]
\noindent\textbf{Question 22}\hspace{20pt}Experience: 25\hspace{20pt}Order: \hspace{20pt}Level: \hspace{20pt}Question-ID: 50\\[2pt]
Calculate the following sum:
\begin{align*}
\sum_{r=3}^{6} (r^2-1)
\end{align*}
\noindent\textbf{Solution 22}\\[2pt]
\\[-35pt]\begin{align*}
&\sum_{r=3}^{6} (r^2-1)\\[2pt]
=\,\,&(3^2-1)+(4^2-1)+(5^2-1)+(6^2-1)\\[2pt]
=\,\,&8+15+24+35\\[2pt]
=\,\,&82\\[-20pt]
\end{align*}
Choice 1: \hspace{20pt}81\hspace{20pt}false\\
Choice 2: \hspace{20pt}80\hspace{20pt}false\\
Choice 3: \hspace{20pt}83\hspace{20pt}false\\
Choice 4: \hspace{20pt}84\hspace{20pt}false\\
Choice 5: \hspace{20pt}82\hspace{20pt}true\\
\\[4pt]
\noindent\textbf{Question 23}\hspace{20pt}Experience: 15\hspace{20pt}Order: \hspace{20pt}Level: \hspace{20pt}Question-ID: 51\\[2pt]
Calculate the following sum:
\begin{align*}
\sum_{r=1}^{45} 2
\end{align*}
\noindent\textbf{Solution 23}\\[2pt]
\\[-35pt]\begin{align*}
&\sum_{r=1}^{45} 2\\[2pt]
=\,\,&2+2+2+2+2+...+2\\[2pt]
=\,\,&2 \, \text{x} \, 45\\[2pt]
=\,\,&90
\end{align*}
Choice 1: \hspace{20pt}94\hspace{20pt}false\\
Choice 2: \hspace{20pt}92\hspace{20pt}false\\
Choice 3: \hspace{20pt}88\hspace{20pt}false\\
Choice 4: \hspace{20pt}86\hspace{20pt}false\\
Choice 5: \hspace{20pt}90\hspace{20pt}true\\
\\[4pt]
\noindent\textbf{Question 24}\hspace{20pt}Experience: 15\hspace{20pt}Order: \hspace{20pt}Level: \hspace{20pt}Question-ID: 49\\[2pt]
Calculate the following sum:
\begin{align*}
\sum_{r=1}^{100} 5
\end{align*}
\noindent\textbf{Solution 24}\\[2pt]
\\[-35pt]\begin{align*}
&\sum_{r=1}^{100} 5\\[2pt]
=\,\,&5+5+5+5+5+5+...+5\\[2pt]
=\,\,&5 \, \text{x} \, 100\\[2pt]
=\,\,&500
\end{align*}
Choice 1: \hspace{20pt}495\hspace{20pt}false\\
Choice 2: \hspace{20pt}490\hspace{20pt}false\\
Choice 3: \hspace{20pt}480\hspace{20pt}false\\
Choice 4: \hspace{20pt}500\hspace{20pt}true\\
Choice 5: \hspace{20pt}485\hspace{20pt}false\\
\\[4pt]
\noindent\textbf{Question 25}\hspace{20pt}Experience: 25\hspace{20pt}Order: \hspace{20pt}Level: \hspace{20pt}Question-ID: 25\\[2pt]
A sequence is defined by $U_n=2n+3$, find the value of $U_2, U_4$ and $U_5$.\\[4pt]
\noindent\textbf{Solution 25}\\[2pt]
\\[-35pt]\begin{align*}
U_2&=2(2)+3=7\\[2pt]
U_4&=2(4)+3=11\\[2pt]
U_5&=2(5)+3=13\\[2pt]
\end{align*}
Choice 1: \hspace{20pt}$U_2=5 \,\, U_4=9 \,\, U_5=11$\hspace{20pt}false\\
Choice 2: \hspace{20pt}$U_2=7 \,\, U_4=9 \,\, U_5=13$\hspace{20pt}false\\
Choice 3: \hspace{20pt}$U_2=7 \,\, U_4=10 \,\, U_5=15$\hspace{20pt}false\\
Choice 4: \hspace{20pt}$U_2=7 \,\, U_4=11 \,\, U_5=13$\hspace{20pt}true\\
Choice 5: \hspace{20pt}$U_2=5 \,\, U_4=11 \,\, U_5=13$\hspace{20pt}false\\
\\[4pt]
\noindent\textbf{Question 26}\hspace{20pt}Experience: 25\hspace{20pt}Order: \hspace{20pt}Level: \hspace{20pt}Question-ID: 29\\[2pt]
A sequence is defined by $u_n=2n^2-5n-3$, find the value of $n$ such that $u_n=9$.\\[4pt]
\noindent\textbf{Solution 26}\\[2pt]
\\[-35pt]\begin{align*}
u_n=2n^2-5n-3&=9\\[2pt]
2n^2-5n-12&=0\hspace{20pt}S=-5 \quad P=-24\\[2pt]
\left(n+\displaystyle\frac{3}{2}\right)(n-4)&=0\hspace{20pt} (3,-8)\quad\left(\displaystyle\frac{3}{2},-4\right)\\[2pt]
n&=4
\end{align*}
Choice 1: \hspace{20pt}$n=5$\hspace{20pt}false\\
Choice 2: \hspace{20pt}$n=2$\hspace{20pt}false\\
Choice 3: \hspace{20pt}$n=3$\hspace{20pt}false\\
Choice 4: \hspace{20pt}$n=6$\hspace{20pt}false\\
Choice 5: \hspace{20pt}$n=4$\hspace{20pt}true\\
\\[4pt]
\noindent\textbf{Question 27}\hspace{20pt}Experience: 50\hspace{20pt}Order: \hspace{20pt}Level: \hspace{20pt}Question-ID: 39\\[2pt]
A sequence is defined by $a_n=an^2+b$, given the Sum of the first five terms is $-5$ and the sixth term is $4$, find the values of $a$ and $b$.\\[4pt]
\noindent\textbf{Solution 27}\\[2pt]
\\[-35pt]\begin{align*}
S_5&=(a+b)+(4a+b)+(9a+b)+(16a+b)\\[2pt]
S_5&=30a+5b\hspace{20pt} S_5=-5\\[2pt]
30a+5b&=-5\\[2pt]
6a+b&=-1\quad (1)\\[12pt]
a_6&=25a+b\hspace{20pt}a_6=4\\[2pt]
36a+b&=4\quad (2)\\[12pt]
(2)-(1)\quad 36a+b-(6a+b)&=4-(-1)\\[2pt]
30a&=5\\[2pt]
a&=\displaystyle\frac{1}{6}\\[12pt]
\text{sub into}\quad (1) \quad 6\left(\displaystyle\frac{1}{6}\right)+b&=-1\\[2pt]
b&=-2\\[2pt]
\end{align*}
Choice 1: \hspace{20pt}$a=\displaystyle\frac{1}{6}\quad b=2$\hspace{20pt}false\\
Choice 2: \hspace{20pt}$a=1 \quad b=-2$\hspace{20pt}false\\
Choice 3: \hspace{20pt}$a=1 \quad b=2$\hspace{20pt}false\\
Choice 4: \hspace{20pt}$a=2 \quad b=\displaystyle\frac{1}{6}$\hspace{20pt}false\\
Choice 5: \hspace{20pt}$a=\displaystyle\frac{1}{6} \quad b=-2$\hspace{20pt}true\\
\\[4pt]
\noindent\large{\textbf{Lesson 2 Arithmetic Sequence 1}}\\[12pt]
\noindent\textbf{Question 1}\hspace{20pt}Experience: 30\hspace{20pt}Order: \hspace{20pt}Level: \hspace{20pt}Question-ID: 114\\[2pt]
Evaluate $\displaystyle\sum_{r=1}^{15} (5r+2)$\\[4pt]
\noindent\textbf{Solution 1}\\[2pt]
\\[-35pt]\begin{align*}
\displaystyle\sum_{r=1}^{15} (5r+2)&=7+12+17+22+...+77\\[2pt]
a&=7\quad l=77 \quad n=15\\[2pt]
\displaystyle\sum_{r=1}^{15} (5r+2)&=\displaystyle\frac{15}{2}(7+77)\\[2pt]
&=630
\end{align*}
Choice 1: \hspace{20pt}$\displaystyle\sum_{r=1}^{15} (5r+2)=625$\hspace{20pt}false\\
Choice 2: \hspace{20pt}$\displaystyle\sum_{r=1}^{15} (5r+2)=620$\hspace{20pt}false\\
Choice 3: \hspace{20pt}$\displaystyle\sum_{r=1}^{15} (5r+2)=615$\hspace{20pt}false\\
Choice 4: \hspace{20pt}$\displaystyle\sum_{r=1}^{15} (5r+2)=635$\hspace{20pt}false\\
Choice 5: \hspace{20pt}$\displaystyle\sum_{r=1}^{15} (5r+2)=630$\hspace{20pt}true\\
\\[4pt]
\noindent\textbf{Question 2}\hspace{20pt}Experience: 30\hspace{20pt}Order: \hspace{20pt}Level: \hspace{20pt}Question-ID: 100\\[2pt]
How many terms are there in the arithmetic sequence 19,21,23,...,87\\[4pt]
\noindent\textbf{Solution 2}\\[2pt]
\\[-35pt]\begin{align*}
a&=19 \quad d=2\\[2pt]
U_n&=a+(n-1)d\\[2pt]
87&=19+(n-1)2\\[2pt]
n-1&=34\\[2pt]
n&=35
\end{align*}
Choice 1: \hspace{20pt}$n=38$\hspace{20pt}false\\
Choice 2: \hspace{20pt}$n=37$\hspace{20pt}false\\
Choice 3: \hspace{20pt}$n=36$\hspace{20pt}false\\
Choice 4: \hspace{20pt}$n=34$\hspace{20pt}false\\
Choice 5: \hspace{20pt}$n=35$\hspace{20pt}true\\
\\[4pt]
\noindent\textbf{Question 3}\hspace{20pt}Experience: 30\hspace{20pt}Order: \hspace{20pt}Level: \hspace{20pt}Question-ID: 101\\[2pt]
How many terms are there in the arithmetic sequence 21,26,31,...,256\\[4pt]
\noindent\textbf{Solution 3}\\[2pt]
\\[-35pt]\begin{align*}
a&=21 \quad d=5\\[2pt]
U_n&=a+(n-1)d\\[2pt]
256&=21+(n-1)5\\[2pt]
n-1&=47\\[2pt]
n&=48
\end{align*}
Choice 1: \hspace{20pt}$n=51$\hspace{20pt}false\\
Choice 2: \hspace{20pt}$n=47$\hspace{20pt}false\\
Choice 3: \hspace{20pt}$n=50$\hspace{20pt}false\\
Choice 4: \hspace{20pt}$n=49$\hspace{20pt}false\\
Choice 5: \hspace{20pt}$n=48$\hspace{20pt}true\\
\\[4pt]
\noindent\textbf{Question 4}\hspace{20pt}Experience: 35\hspace{20pt}Order: \hspace{20pt}Level: \hspace{20pt}Question-ID: 102\\[2pt]
How many terms are there in the arithmetic sequence 88,86,84,...,22\\[4pt]
\noindent\textbf{Solution 4}\\[2pt]
Reverse the order of the sequence$\quad$ 22,24,26,28...88
\begin{align*}
a&=88 \quad d=2\\[2pt]
U_n&=a+(n-1)d\\[2pt]
88&=22+(n-1)2\\[2pt]
n-1&=33\\[2pt]
n&=34\\[-100pt]
\end{align*}
Choice 1: \hspace{20pt}$n=36$\hspace{20pt}false\\
Choice 2: \hspace{20pt}$n=35$\hspace{20pt}false\\
Choice 3: \hspace{20pt}$n=32$\hspace{20pt}false\\
Choice 4: \hspace{20pt}$n=33$\hspace{20pt}false\\
Choice 5: \hspace{20pt}$n=34$\hspace{20pt}true\\
\\[4pt]
\noindent\textbf{Question 5}\hspace{20pt}Experience: 30\hspace{20pt}Order: \hspace{20pt}Level: \hspace{20pt}Question-ID: 103\\[2pt]
Evaluate $S=1+2+3+4+...+50$\\[4pt]
\noindent\textbf{Solution 5}\\[2pt]
\\[-35pt]\begin{align*}
S&=1+2+3+4+...+50\\[2pt]
S&=50+49+48+47+...+1\\[2pt]
2S&=51\,\,\text{x}\,\,50\\[2pt]
S&=\displaystyle\frac{51\,\,\text{x}\,\,50}{2}\\[2pt]
S&=1275
\end{align*}
Choice 1: \hspace{20pt}$S=1270$\hspace{20pt}false\\
Choice 2: \hspace{20pt}$S=1280$\hspace{20pt}false\\
Choice 3: \hspace{20pt}$S=1285$\hspace{20pt}false\\
Choice 4: \hspace{20pt}$S=1290$\hspace{20pt}false\\
Choice 5: \hspace{20pt}$S=1275$\hspace{20pt}true\\
\\[4pt]
\noindent\textbf{Question 6}\hspace{20pt}Experience: 30\hspace{20pt}Order: \hspace{20pt}Level: \hspace{20pt}Question-ID: 104\\[2pt]
Evaluate $T=2+4+6+8+...+100$\\[4pt]
\noindent\textbf{Solution 6}\\[2pt]
\\[-35pt]\begin{align*}
T&=2+4+6+8+...+100\\[2pt]
T&=100+98+96+94+...+2\\[2pt]
2T&=102\,\,\text{x}\,\,50\\[2pt]
T&=\displaystyle\frac{102\,\,\text{x}\,\,50}{2}\\[2pt]
T&=2550
\end{align*}
Choice 1: \hspace{20pt}$T=2565$\hspace{20pt}false\\
Choice 2: \hspace{20pt}$T=2560$\hspace{20pt}false\\
Choice 3: \hspace{20pt}$T=2555$\hspace{20pt}false\\
Choice 4: \hspace{20pt}$T=2545$\hspace{20pt}false\\
Choice 5: \hspace{20pt}$T=2550$\hspace{20pt}true\\
\\[4pt]
\noindent\textbf{Question 7}\hspace{20pt}Experience: 30\hspace{20pt}Order: \hspace{20pt}Level: \hspace{20pt}Question-ID: 105\\[2pt]
Evaluate $R=1+3+5+7+...+99$\\[4pt]
\noindent\textbf{Solution 7}\\[2pt]
\\[-35pt]\begin{align*}
R&=1+3+5+7+...+99\\[2pt]
R&=99+97+95+93+...+1\\[2pt]
2R&=100\,\,\text{x}\,\,100\\[2pt]
R&=\displaystyle\frac{100\,\,\text{x}\,\,100}{2}\\[2pt]
R&=5000
\end{align*}
Choice 1: \hspace{20pt}$R=5015$\hspace{20pt}false\\
Choice 2: \hspace{20pt}$R=5010$\hspace{20pt}false\\
Choice 3: \hspace{20pt}$R=5005$\hspace{20pt}false\\
Choice 4: \hspace{20pt}$R=4995$\hspace{20pt}false\\
Choice 5: \hspace{20pt}$R=5000$\hspace{20pt}true\\
\\[4pt]
\noindent\textbf{Question 8}\hspace{20pt}Experience: 30\hspace{20pt}Order: \hspace{20pt}Level: \hspace{20pt}Question-ID: 106\\[2pt]
Evaluate $S=1+2+3+4+...+200$\\[4pt]
\noindent\textbf{Solution 8}\\[2pt]
\\[-35pt]\begin{align*}
S&=1+2+3+4+...+200\\[2pt]
S&=200+199+198+197+...+1\\[2pt]
2S&=201\,\,\text{x}\,\,200\\[2pt]
S&=\displaystyle\frac{201\,\,\text{x}\,\,200}{2}\\[2pt]
S&=20100
\end{align*}
Choice 1: \hspace{20pt}$S=20115$\hspace{20pt}false\\
Choice 2: \hspace{20pt}$S=20110$\hspace{20pt}false\\
Choice 3: \hspace{20pt}$S=20105$\hspace{20pt}false\\
Choice 4: \hspace{20pt}$S=20095$\hspace{20pt}false\\
Choice 5: \hspace{20pt}$S=20100$\hspace{20pt}true\\
\\[4pt]
\noindent\textbf{Question 9}\hspace{20pt}Experience: 30\hspace{20pt}Order: \hspace{20pt}Level: \hspace{20pt}Question-ID: 107\\[2pt]
Evaluate $T=102+104+106+108+...+200$\\[4pt]
\noindent\textbf{Solution 9}\\[2pt]
\\[-35pt]\begin{align*}
T&=102+104+106+108+...+200\\[2pt]
T&=200+198+196+194+...+102\\[2pt]
2T&=302\,\,\text{x}\,\,50\\[2pt]
T&=\displaystyle\frac{302\,\,\text{x}\,\,50}{2}\\[2pt]
T&=7550
\end{align*}
Choice 1: \hspace{20pt}$T=7565$\hspace{20pt}false\\
Choice 2: \hspace{20pt}$T=7560$\hspace{20pt}false\\
Choice 3: \hspace{20pt}$T=7555$\hspace{20pt}false\\
Choice 4: \hspace{20pt}$T=7545$\hspace{20pt}false\\
Choice 5: \hspace{20pt}$T=7550$\hspace{20pt}true\\
\\[4pt]
\noindent\textbf{Question 10}\hspace{20pt}Experience: 40\hspace{20pt}Order: \hspace{20pt}Level: \hspace{20pt}Question-ID: 109\\[2pt]
Find the sum of all numbers divisible by 5 between $1$ and $300$\\[4pt]
\noindent\textbf{Solution 10}\\[2pt]
\\[-35pt]\begin{align*}
&300 \div 5 = 60\\[2pt]
&\Rightarrow \quad \text{last term }= 300\\[2pt]
S&=5+10+15+...+300\\[12pt]
a&=5\quad d=5 \quad U_n=300\\[2pt]
U_n&=a+(n-1)d\\[2pt]
300&=5+(n-1)5\\[2pt]
n&=60\\[12pt]
S&=\displaystyle\frac{n}{2}(a+l)\\[2pt]
S&=\displaystyle\frac{60}{2}(5+300)\\[2pt]
S&=9150\\[-50pt]
\end{align*}
Choice 1: \hspace{20pt}$S=9165$\hspace{20pt}false\\
Choice 2: \hspace{20pt}$S=9160$\hspace{20pt}false\\
Choice 3: \hspace{20pt}$S=9155$\hspace{20pt}false\\
Choice 4: \hspace{20pt}$S=9145$\hspace{20pt}false\\
Choice 5: \hspace{20pt}$S=9150$\hspace{20pt}true\\
\\[4pt]
\noindent\textbf{Question 11}\hspace{20pt}Experience: 40\hspace{20pt}Order: \hspace{20pt}Level: \hspace{20pt}Question-ID: 110\\[2pt]
Find the sum of all numbers divisible by 7 between $1$ and $200$\\[4pt]
\noindent\textbf{Solution 11}\\[2pt]
\\[-35pt]\begin{align*}
&200 \div 7 = 28 \,\, \text{remainder}\,\, 4\\[2pt]
&\Rightarrow \quad \text{last term }= 7 \,\,\text{x}\,\, 28=196\\[2pt]
S&=7+14+21+...+196\\[12pt]
a&=7\quad d=7 \quad U_n=196\\[2pt]
U_n&=a+(n-1)d\\[2pt]
196&=7+(n-1)7\\[2pt]
n&=28\\[12pt]
S&=\displaystyle\frac{n}{2}(a+l)\\[2pt]
S&=\displaystyle\frac{28}{2}(7+196)\\[2pt]
S&=2842\\[-140pt]
\end{align*}
Choice 1: \hspace{20pt}$S=2863$\hspace{20pt}false\\
Choice 2: \hspace{20pt}$S=2856$\hspace{20pt}false\\
Choice 3: \hspace{20pt}$S=2849$\hspace{20pt}false\\
Choice 4: \hspace{20pt}$S=2835$\hspace{20pt}false\\
Choice 5: \hspace{20pt}$S=2842$\hspace{20pt}true\\
\\[4pt]
\noindent\textbf{Question 12}\hspace{20pt}Experience: 45\hspace{20pt}Order: \hspace{20pt}Level: \hspace{20pt}Question-ID: 111\\[2pt]
Evaluate $S=27+31+35+39+...+107$\\[4pt]
\noindent\textbf{Solution 12}\\[2pt]
\\[-35pt]\begin{align*}
S&=27+31+35+39+...+107\\[2pt]
a&=27\quad d=4 \quad U_n=107\\[2pt]
U_n&=a+(n-1)d\\[2pt]
107&=27+(n-1)4\\[2pt]
n&=21\\[12pt]
S&=\displaystyle\frac{n}{2}(a+l)\\[2pt]
S&=\displaystyle\frac{21}{2}(27+107)\\[2pt]
S&=1407\\[-140pt]
\end{align*}
Choice 1: \hspace{20pt}$S=1414$\hspace{20pt}false\\
Choice 2: \hspace{20pt}$S=1386$\hspace{20pt}false\\
Choice 3: \hspace{20pt}$S=1393$\hspace{20pt}false\\
Choice 4: \hspace{20pt}$S=1400$\hspace{20pt}false\\
Choice 5: \hspace{20pt}$S=1407$\hspace{20pt}true\\
\\[4pt]
\noindent\textbf{Question 13}\hspace{20pt}Experience: 45\hspace{20pt}Order: \hspace{20pt}Level: \hspace{20pt}Question-ID: 112\\[2pt]
Evaluate $T=31+33+35+37+...+81$\\[4pt]
\noindent\textbf{Solution 13}\\[2pt]
\\[-35pt]\begin{align*}
T&=31+33+35+37+...+81\\[2pt]
a&=31\quad d=2 \quad U_n=81\\[2pt]
U_n&=a+(n-1)d\\[2pt]
81&=31+(n-1)2\\[2pt]
n&=26\\[12pt]
S&=\displaystyle\frac{n}{2}(a+l)\\[2pt]
S&=\displaystyle\frac{26}{2}(31+81)\\[2pt]
S&=1456\\[-140pt]
\end{align*}
Choice 1: \hspace{20pt}$S=1462$\hspace{20pt}false\\
Choice 2: \hspace{20pt}$S=1460$\hspace{20pt}false\\
Choice 3: \hspace{20pt}$S=1458$\hspace{20pt}false\\
Choice 4: \hspace{20pt}$S=1454$\hspace{20pt}false\\
Choice 5: \hspace{20pt}$S=1456$\hspace{20pt}true\\
\\[4pt]
\noindent\textbf{Question 14}\hspace{20pt}Experience: 45\hspace{20pt}Order: \hspace{20pt}Level: \hspace{20pt}Question-ID: 113\\[2pt]
Evaluate $R=97+92+87+82+...+22$\\[4pt]
\noindent\textbf{Solution 14}\\[2pt]
Reverse the sequence: $R=22+27+32+27+...+97$
\begin{align*}
R&=22+27+32+27+...+97\\[2pt]
a&=22\quad d=5 \quad U_n=97\\[2pt]
U_n&=a+(n-1)d\\[2pt]
97&=22+(n-1)5\\[2pt]
n&=16\\[12pt]
S&=\displaystyle\frac{n}{2}(a+l)\\[2pt]
S&=\displaystyle\frac{16}{2}(22+97)\\[2pt]
S&=952\\
\end{align*}
Choice 1: \hspace{20pt}$S=950$\hspace{20pt}false\\
Choice 2: \hspace{20pt}$S=954$\hspace{20pt}false\\
Choice 3: \hspace{20pt}$S=948$\hspace{20pt}false\\
Choice 4: \hspace{20pt}$S=950$\hspace{20pt}false\\
Choice 5: \hspace{20pt}$S=952$\hspace{20pt}true\\
\\[4pt]
\noindent\textbf{Question 15}\hspace{20pt}Experience: 30\hspace{20pt}Order: \hspace{20pt}Level: \hspace{20pt}Question-ID: 115\\[2pt]
Evaluate $\displaystyle\sum_{r=9}^{35} (3r-1)$\\[4pt]
\noindent\textbf{Solution 15}\\[2pt]
\\[-35pt]\begin{align*}
\displaystyle\sum_{r=9}^{35} (3r-1)&=26+29+32+35+...+104\\[2pt]
a&=26\quad l=104 \quad n=27\\[2pt]
\displaystyle\sum_{r=9}^{35} (3r-1)&=\displaystyle\frac{27}{2}(26+104)\\[2pt]
&=1755
\end{align*}
Choice 1: \hspace{20pt}$\displaystyle\sum_{r=9}^{35} (3r-1)=1760$\hspace{20pt}false\\
Choice 2: \hspace{20pt}$\displaystyle\sum_{r=9}^{35} (3r-1)=1740$\hspace{20pt}false\\
Choice 3: \hspace{20pt}$\displaystyle\sum_{r=9}^{35} (3r-1)=1745$\hspace{20pt}false\\
Choice 4: \hspace{20pt}$\displaystyle\sum_{r=9}^{35} (3r-1)=1750$\hspace{20pt}false\\
Choice 5: \hspace{20pt}$\displaystyle\sum_{r=9}^{35} (3r-1)=1755$\hspace{20pt}true\\
\\[4pt]
\noindent\textbf{Question 16}\hspace{20pt}Experience: 30\hspace{20pt}Order: \hspace{20pt}Level: \hspace{20pt}Question-ID: 116\\[2pt]
Evaluate $\displaystyle\sum_{r=1}^{20} (3r-1)$\\[4pt]
\noindent\textbf{Solution 16}\\[2pt]
\\[-35pt]\begin{align*}
\displaystyle\sum_{r=1}^{20} (3r-1)&=2+5+8+11+...+59\\[2pt]
a&=2\quad l=59 \quad n=20\\[2pt]
\displaystyle\sum_{r=1}^{20} (3r-1)&=\displaystyle\frac{20}{2}(2+59)\\[2pt]
&=610
\end{align*}
Choice 1: \hspace{20pt}$\displaystyle\sum_{r=1}^{20} (3r-1)=625$\hspace{20pt}false\\
Choice 2: \hspace{20pt}$\displaystyle\sum_{r=1}^{20} (3r-1)=605$\hspace{20pt}false\\
Choice 3: \hspace{20pt}$\displaystyle\sum_{r=1}^{20} (3r-1)=615$\hspace{20pt}false\\
Choice 4: \hspace{20pt}$\displaystyle\sum_{r=1}^{20} (3r-1)=620$\hspace{20pt}false\\
Choice 5: \hspace{20pt}$\displaystyle\sum_{r=1}^{20} (3r-1)=610$\hspace{20pt}true\\
\\[4pt]
\noindent\textbf{Question 17}\hspace{20pt}Experience: 30\hspace{20pt}Order: \hspace{20pt}Level: \hspace{20pt}Question-ID: 117\\[2pt]
Evaluate $\displaystyle\sum_{r=21}^{45} (2r-25)$\\[4pt]
\noindent\textbf{Solution 17}\\[2pt]
\\[-35pt]\begin{align*}
\displaystyle\sum_{r=21}^{45} (2r-25)&=17+19+21+23+...+65\\[2pt]
a&=17\quad l=65 \quad n=25\\[2pt]
\displaystyle\sum_{r=21}^{45} (2r-25)&=\displaystyle\frac{25}{2}(17+65)\\[2pt]
&=1025
\end{align*}
Choice 1: \hspace{20pt}$\displaystyle\sum_{r=21}^{45} (2r-25)=1020$\hspace{20pt}false\\
Choice 2: \hspace{20pt}$\displaystyle\sum_{r=21}^{45} (2r-25)=1015$\hspace{20pt}false\\
Choice 3: \hspace{20pt}$\displaystyle\sum_{r=21}^{45} (2r-25)=1010$\hspace{20pt}false\\
Choice 4: \hspace{20pt}$\displaystyle\sum_{r=21}^{45} (2r-25)=1030$\hspace{20pt}false\\
Choice 5: \hspace{20pt}$\displaystyle\sum_{r=21}^{45} (2r-25)=1025$\hspace{20pt}true\\
\\[4pt]
\noindent\textbf{Question 18}\hspace{20pt}Experience: 20\hspace{20pt}Order: \hspace{20pt}Level: \hspace{20pt}Question-ID: 84\\[2pt]
The first three terms of an arithmetic sequence are 3,5,7, find $U_{10}$\\[4pt]
\noindent\textbf{Solution 18}\\[2pt]
\\[-35pt]\begin{align*}
a&=3\quad n=10 \quad d=5-3=2\\[2pt]
U_n&=a+(n-1)d\\[12pt]
U_{10}&=3+(10-1)2=21\\[2pt]
\end{align*}
Choice 1: \hspace{20pt}$U_{10}=20$\hspace{20pt}false\\
Choice 2: \hspace{20pt}$U_{10}=17$\hspace{20pt}false\\
Choice 3: \hspace{20pt}$U_{10}=18$\hspace{20pt}false\\
Choice 4: \hspace{20pt}$U_{10}=19$\hspace{20pt}false\\
Choice 5: \hspace{20pt}$U_{10}=21$\hspace{20pt}true\\
\\[4pt]
\noindent\textbf{Question 19}\hspace{20pt}Experience: 20\hspace{20pt}Order: \hspace{20pt}Level: \hspace{20pt}Question-ID: 85\\[2pt]
The first four terms of an arithmetic sequence are 5,9,13,17, find $A_7$\\[4pt]
\noindent\textbf{Solution 19}\\[2pt]
\\[-35pt]\begin{align*}
a&=5 \quad n=7 \quad d=9-5=4\\[2pt]
A_n&=a+(n-1)d\\[2pt]
A_7&=5+(7-1)4=29\\[2pt]
\end{align*}
Choice 1: \hspace{20pt}$A_7=28$\hspace{20pt}false\\
Choice 2: \hspace{20pt}$A_7=27$\hspace{20pt}false\\
Choice 3: \hspace{20pt}$A_7=30$\hspace{20pt}false\\
Choice 4: \hspace{20pt}$A_7=26$\hspace{20pt}false\\
Choice 5: \hspace{20pt}$A_7=29$\hspace{20pt}true\\
\\[4pt]
\noindent\textbf{Question 20}\hspace{20pt}Experience: 20\hspace{20pt}Order: \hspace{20pt}Level: \hspace{20pt}Question-ID: 89\\[2pt]
The first three terms of an arithmetic sequence are 22,19,16, find $X_6$\\[4pt]
\noindent\textbf{Solution 20}\\[2pt]
\\[-35pt]\begin{align*}
a&=22 \quad n=6 \quad d=22-19=3\\[2pt]
X_n&=a+(n-1)d\\[2pt]
X_7&=22+(6-1)3=37
\end{align*}
Choice 1: \hspace{20pt}$A_7=28$\hspace{20pt}false\\
Choice 2: \hspace{20pt}$A_7=27$\hspace{20pt}false\\
Choice 3: \hspace{20pt}$A_7=30$\hspace{20pt}false\\
Choice 4: \hspace{20pt}$A_7=26$\hspace{20pt}false\\
Choice 5: \hspace{20pt}$A_7=29$\hspace{20pt}true\\
\\[4pt]
\noindent\textbf{Question 21}\hspace{20pt}Experience: 40\hspace{20pt}Order: \hspace{20pt}Level: \hspace{20pt}Question-ID: 90\\[2pt]
$a_n$ is an arithmetic sequence, given that $a_3=13$ and $a_6=19$, find $a_{11}$\\[4pt]
\noindent\textbf{Solution 21}\\[2pt]
\\[-35pt]\begin{align*}
a_n&=a+(n-1)d\\[2pt]
a_3&=a+(3-1)d=a+2d=13\quad (1)\\[2pt]
a_6&=a+(6-1)d=a+5d=19\quad (2)\\[2pt]
(2)-(1)\quad a+5d-(a+2d)&=19-13\\[2pt]
3d&=6\\[2pt]
d&=2\\[12pt]
\text{Sub into} (1) \quad a+2(2)&=13\\[2pt]
a&=9\\[12pt]
a_{11}&=9+(11-1)2=29
\end{align*}
Choice 1: \hspace{20pt}$a_{11}=25$\hspace{20pt}false\\
Choice 2: \hspace{20pt}$a_{11}=26$\hspace{20pt}false\\
Choice 3: \hspace{20pt}$a_{11}=27$\hspace{20pt}false\\
Choice 4: \hspace{20pt}$a_{11}=28$\hspace{20pt}false\\
Choice 5: \hspace{20pt}$a_{11}=29$\hspace{20pt}true\\
\\[4pt]
\noindent\textbf{Question 22}\hspace{20pt}Experience: 40\hspace{20pt}Order: \hspace{20pt}Level: \hspace{20pt}Question-ID: 91\\[2pt]
$U_n$ is an arithmetic sequence, given that $U_4=25$ and $U_9=40$, find $U_{13}$\\[4pt]
\noindent\textbf{Solution 22}\\[2pt]
\\[-35pt]\begin{align*}
U_n&=a+(n-1)d\\[2pt]
U_4&=a+(4-1)d=a+3d=25\quad (1)\\[2pt]
U_9&=a+(9-1)d=a+8d=40\quad (2)\\[2pt]
(2)-(1)\quad a+8d-(a+3d)&=40-25\\[2pt]
5d&=15\\[2pt]
d&=3\\[12pt]
\text{Sub into} (1) \quad a+3(3)&=25\\[2pt]
a&=16\\[12pt]
U_{13}&=16+(13-1)3=52\\[-40pt]
\end{align*}
Choice 1: \hspace{20pt}$U_{13}=51$\hspace{20pt}false\\
Choice 2: \hspace{20pt}$U_{13}=50$\hspace{20pt}false\\
Choice 3: \hspace{20pt}$U_{13}=49$\hspace{20pt}false\\
Choice 4: \hspace{20pt}$U_{13}=53$\hspace{20pt}false\\
Choice 5: \hspace{20pt}$U_{13}=52$\hspace{20pt}true\\
\\[4pt]
\noindent\textbf{Question 23}\hspace{20pt}Experience: 45\hspace{20pt}Order: \hspace{20pt}Level: \hspace{20pt}Question-ID: 96\\[2pt]
$X_n$ is an arithmetic sequence, given that $X_{13}=51$ and $X_{19}=33$, find $X_{10}$\\[4pt]
\noindent\textbf{Solution 23}\\[2pt]
\\[-35pt]\begin{align*}
X_n&=a+(n-1)d\\[2pt]
X_{13}&=a+(13-1)d=a+12d=51\quad (1)\\[2pt]
X_{19}&=a+(19-1)d=a+18d=33\quad (2)\\[2pt]
(2)-(1)\quad a+18d-(a+12d)&=33-51\\[2pt]
6d&=-18\\[2pt]
d&=-3\\[12pt]
\text{Sub into} (1) \quad a+12(-3)&=51\\[2pt]
a&=87\\[12pt]
X_{10}&=87+(10-1)(-3)=60\\[-40pt]
\end{align*}
Choice 1: \hspace{20pt}$X_{10}=63$\hspace{20pt}false\\
Choice 2: \hspace{20pt}$X_{10}=62$\hspace{20pt}false\\
Choice 3: \hspace{20pt}$X_{10}=61$\hspace{20pt}false\\
Choice 4: \hspace{20pt}$X_{10}=59$\hspace{20pt}false\\
Choice 5: \hspace{20pt}$X_{10}=60$\hspace{20pt}true\\
\\[4pt]
\noindent\textbf{Question 24}\hspace{20pt}Experience: 45\hspace{20pt}Order: \hspace{20pt}Level: \hspace{20pt}Question-ID: 97\\[2pt]
$u_n$ is an arithmetic sequence, given that $u_{3}=5$ and $u_{7}=13$, for what value of $n$ is $a_n=71$\\[4pt]
\noindent\textbf{Solution 24}\\[2pt]
\\[-35pt]\begin{align*}
u_n&=a+(n-1)d\\[2pt]
u_3&=a+(3-1)d=a+2d=5 \hspace{15pt} (1)\\[2pt]
u_7&=a+(7-1)d=a+6d=13\quad (2) \\[2pt]
(2)-(1)\quad a+6d-(a+2d)&=13-5\\[2pt]
4d&=8\\[2pt]
d&=2\\[12pt]
\text{Sub into} (1) \quad a+2(2)&=5\\[2pt]
a&=1\\[12pt]
u_n&=1+(n-1)2=71\\[2pt]
n-1&=35\\[2pt]
n&=36\\[-60pt]
\end{align*}
Choice 1: \hspace{20pt}$n=35$\hspace{20pt}false\\
Choice 2: \hspace{20pt}$n=32$\hspace{20pt}false\\
Choice 3: \hspace{20pt}$n=33$\hspace{20pt}false\\
Choice 4: \hspace{20pt}$n=34$\hspace{20pt}false\\
Choice 5: \hspace{20pt}$n=36$\hspace{20pt}true\\
\\[4pt]
\noindent\textbf{Question 25}\hspace{20pt}Experience: 30\hspace{20pt}Order: \hspace{20pt}Level: \hspace{20pt}Question-ID: 98\\[2pt]
The first three terms of an arithmetic sequence are 11,14,17, find a $n$ for which $U_n=83$\\[4pt]
\noindent\textbf{Solution 25}\\[2pt]
\\[-35pt]\begin{align*}
u_n&=83 \quad a=11 \quad d=3\\[2pt]
u_n&=a+(n-1)d\\[2pt]
83&=11+(n-1)3\\[2pt]
n-1&=24\\[2pt]
n&=25
\end{align*}
Choice 1: \hspace{20pt}$n=24$\hspace{20pt}false\\
Choice 2: \hspace{20pt}$n=23$\hspace{20pt}false\\
Choice 3: \hspace{20pt}$n=22$\hspace{20pt}false\\
Choice 4: \hspace{20pt}$n=26$\hspace{20pt}false\\
Choice 5: \hspace{20pt}$n=25$\hspace{20pt}true\\
\\[4pt]
\noindent\textbf{Question 26}\hspace{20pt}Experience: 45\hspace{20pt}Order: \hspace{20pt}Level: \hspace{20pt}Question-ID: 99\\[2pt]
$Y_n$ is an arithmetic sequence, given that $Y_{15}=51$ and $X_{19}=71$, find $Y_{26}$\\[4pt]
\noindent\textbf{Solution 26}\\[2pt]
\\[-35pt]\begin{align*}
Y_n&=a+(n-1)d\\[2pt]
Y_{15}&=a+(15-1)d=a+14d=51 \quad (1)\\[2pt]
Y_{19}&=a+(19-1)d=a+18d=71\quad (2) \\[2pt]
(2)-(1)\quad a+18d-(a+14d)&=71-51\\[2pt]
4d&=20\\[2pt]
d&=5\\[12pt]
\text{Sub into} (1) \quad a+14(5)&=51\\[2pt]
a&=-19\\[12pt]
Y_{26}&=-19+(26-1)5=106\\[-60pt]
\end{align*}
Choice 1: \hspace{20pt}$Y_{26}=102$\hspace{20pt}false\\
Choice 2: \hspace{20pt}$Y_{26}=103$\hspace{20pt}false\\
Choice 3: \hspace{20pt}$Y_{26}=104$\hspace{20pt}false\\
Choice 4: \hspace{20pt}$Y_{26}=105$\hspace{20pt}false\\
Choice 5: \hspace{20pt}$Y_{26}=106$\hspace{20pt}true\\
\\[4pt]
\noindent\textbf{Question 27}\hspace{20pt}Experience: 50\hspace{20pt}Order: \hspace{20pt}Level: \hspace{20pt}Question-ID: 147\\[2pt]
Kendrick decides to open up a savings account. He puts in £100 for the first month, £120 for the second month and an extra £20 for subsequent months till he's putting in £300 a month. Find the total amount he's saved in 2 years.\\[4pt]
\noindent\textbf{Solution 27}\\[2pt]
Sequence goes: 100,120,140,160,180,200...300,300,300,300...$\\[2pt]$
\begin{align*}
U_n&=a+(n-1)d\\[2pt]
U_n&=300\quad a=100 \quad d=20\\[2pt]
300&=100+(n-1)20\\[2pt]
n&=11\\[12pt]
S_n&=\displaystyle\frac{n}{2}(a+l)\\[2pt]
n&=11\quad a=100 \quad l=300\\[2pt]
S_{11}&=\displaystyle\frac{11}{2}(100+300)\\[2pt]
S_{11}&=2200\\[12pt]
&\text{Every term after is 300}\\[2pt]
\sum_{r=12}^{24}300&=13 \,\, \text{x} \,\, 300\\[2pt]
&=3900\\[12pt]
\Rightarrow \quad \text{Total days}&=2200+3900=6100\\[2pt]
\end{align*}
Choice 1: \hspace{20pt}$6105$\hspace{20pt}false\\
Choice 2: \hspace{20pt}$6085$\hspace{20pt}false\\
Choice 3: \hspace{20pt}$6090$\hspace{20pt}false\\
Choice 4: \hspace{20pt}$6095$\hspace{20pt}false\\
Choice 5: \hspace{20pt}$6100$\hspace{20pt}true\\
\\[4pt]
\noindent\textbf{Question 28}\hspace{20pt}Experience: 50\hspace{20pt}Order: \hspace{20pt}Level: \hspace{20pt}Question-ID: 144\\[2pt]
Avery is playing with 340 sticks, she puts them in rows. The first row has 7 sticks, next row has 13 sticks, subsequent rows have 6 more sticks then the previous row. She has enough for $k$ rows but not enough for $k+1$ rows. Find k.\\[4pt]
\noindent\textbf{Solution 28}\\[2pt]
Sequence goes: 7,13,19,25,31,37....$\\[2pt]$
Not having enough for k+1 rows means that $S_k\leq340$
\begin{align*}
	S_n&=\displaystyle\frac{n}{2}(2a+(k-1)d)\\[2pt]
	S_k&=\displaystyle\frac{k}{2}(2(7)+(k-1)6)\\[2pt]
	S_k&=k(7+3(k-1))\\[2pt]
	S_k&=k(3k+4)\\[2pt]
	S_k&=3k^2+4k \qquad (1)\\[12pt]
	S_k&\leq 340 \\[2pt]
	(1)\qquad 3k^2+4k& \leq 340\\[2pt]
	3k^2+4k-340&\leq 0\qquad P=-1020 \quad S=4\\[2pt]
	\left(k+\displaystyle\frac{34}{3}\right)(k-10)&\leq 0 \qquad (34,-30) \qquad \left(\displaystyle\frac{34}{3},-10\right)\\[2pt]
	k&=10\\[-80pt]
\end{align*}
Choice 1: \hspace{20pt}$k=9$\hspace{20pt}false\\
Choice 2: \hspace{20pt}$k=11$\hspace{20pt}false\\
Choice 3: \hspace{20pt}$k=7$\hspace{20pt}false\\
Choice 4: \hspace{20pt}$k=8$\hspace{20pt}false\\
Choice 5: \hspace{20pt}$k=10$\hspace{20pt}true\\
\\[4pt]
\noindent\textbf{Question 29}\hspace{20pt}Experience: 50\hspace{20pt}Order: \hspace{20pt}Level: \hspace{20pt}Question-ID: 146\\[2pt]
Griffin is training daily for a cycling marathon in 100 days. He cycles 10km on the first day, 11km on the second day and 1 more km then the previous day till he's cycling 40km a day. Calculate the total number of km he's cycled as training for the marathon.\\[4pt]
\noindent\textbf{Solution 29}\\[2pt]
Sequence goes: 10,11,12,13,14,15...40,40,40,40...$\\[2pt]$
\begin{align*}
U_n&=a+(n-1)d\\[2pt]
U_n&=40\quad a=10 \quad d=1\\[2pt]
40&=10+(n-1)1\\[2pt]
n&=31\\[12pt]
S_n&=\displaystyle\frac{n}{2}(a+l)\\[2pt]
n&=31\quad a=10 \quad l=40\\[2pt]
S_{31}&=\displaystyle\frac{31}{2}(10+40)\\[2pt]
S_{31}&=775\\[12pt]
&\text{Every term after is 40}\\[2pt]
\sum_{r=32}^{100}40&=69 \,\, \text{x} \,\, 40\\[2pt]
&=2760\\[12pt]
\Rightarrow \quad \text{Total days}&=775+2760=3535\\[2pt]
\end{align*}
Choice 1: \hspace{20pt}$3540$\hspace{20pt}false\\
Choice 2: \hspace{20pt}$3520$\hspace{20pt}false\\
Choice 3: \hspace{20pt}$3525$\hspace{20pt}false\\
Choice 4: \hspace{20pt}$3530$\hspace{20pt}false\\
Choice 5: \hspace{20pt}$3535$\hspace{20pt}true\\
\\[4pt]
\noindent\textbf{Question 30}\hspace{20pt}Experience: 50\hspace{20pt}Order: \hspace{20pt}Level: \hspace{20pt}Question-ID: 145\\[2pt]
Heidi is training daily for a swimming competition in 60 days. She swims 10 laps on the first day, 12 laps on the second day and 2 more laps then the previous day till she's swimming 30 laps a day. Calculate the total number of laps she's swum as training for the competition.\\[4pt]
\noindent\textbf{Solution 30}\\[2pt]
Sequence goes: 10,12,14,16,18,20...30,30,30,30...$\\[2pt]$
\begin{align*}
U_n&=a+(n-1)d\\[2pt]
U_n&=30\quad a=10 \quad d=2\\[2pt]
30&=10+(n-1)2\\[2pt]
n&=11\\[12pt]
S_n&=\displaystyle\frac{n}{2}(a+l)\\[2pt]
n&=11\quad a=10 \quad l=30\\[2pt]
S_{11}&=\displaystyle\frac{11}{2}(10+30)\\[2pt]
S_{11}&=220\\[12pt]
&\text{Every term after is 30}\\[2pt]
\sum_{r=12}^{60}30&=49 \,\, \text{x} \,\, 30\\[2pt]
&=1470\\[12pt]
\Rightarrow \quad \text{Total days}&=220+1470=1690
\end{align*}
Choice 1: \hspace{20pt}$1685$\hspace{20pt}false\\
Choice 2: \hspace{20pt}$1705$\hspace{20pt}false\\
Choice 3: \hspace{20pt}$1700$\hspace{20pt}false\\
Choice 4: \hspace{20pt}$1695$\hspace{20pt}false\\
Choice 5: \hspace{20pt}$1690$\hspace{20pt}true\\
\\[4pt]
\noindent\textbf{Question 31}\hspace{20pt}Experience: 40\hspace{20pt}Order: \hspace{20pt}Level: \hspace{20pt}Question-ID: 108\\[2pt]
Find the sum of all numbers divisible by 3 between $2$ and $200$\\[4pt]
\noindent\textbf{Solution 31}\\[2pt]
\\[-35pt]\begin{align*}
&200 \div 3 = 66 \,\, \text{remainder}\,\, 2\\[2pt]
&\Rightarrow \quad \text{last term }= 3 \,\,\text{x}\,\, 66=198\\[2pt]
S&=3+6+9+...+198\\[12pt]
a&=3\quad d=3 \quad U_n=198\\[2pt]
U_n&=a+(n-1)d\\[2pt]
198&=3+(n-1)3\\[2pt]
n&=66\\[12pt]
S&=\displaystyle\frac{n}{2}(a+l)\\[2pt]
S&=\displaystyle\frac{66}{2}(3+198)\\[2pt]
S&=6633\\
\end{align*}
Choice 1: \hspace{20pt}$S=6642$\hspace{20pt}false\\
Choice 2: \hspace{20pt}$S=6639$\hspace{20pt}false\\
Choice 3: \hspace{20pt}$S=6636$\hspace{20pt}false\\
Choice 4: \hspace{20pt}$S=6630$\hspace{20pt}false\\
Choice 5: \hspace{20pt}$S=6633$\hspace{20pt}true\\
\\[4pt]
\noindent\textbf{Question 32}\hspace{20pt}Experience: 50\hspace{20pt}Order: \hspace{20pt}Level: \hspace{20pt}Question-ID: 143\\[2pt]
James is playing with 324 sticks, she puts them in rows. The first row has 5 sticks, next row has 9 sticks, subsequent rows have 4 more sticks then the previous row. She has enough for $k$ rows but not enough for $k+1$ rows. Find k.\\[4pt]
\noindent\textbf{Solution 32}\\[2pt]
Sequence goes: 5,9,13,17,21,25....$\\[2pt]$
Not having enough for k+1 rows means that $S_k\leq324$
\begin{align*}
	S_n&=\displaystyle\frac{n}{2}(2a+(k-1)d)\\[2pt]
	S_k&=\displaystyle\frac{k}{2}(2(5)+(k-1)4)\\[2pt]
	S_k&=k(5+2k-2)\\[2pt]
	S_k&=k(2k+3)\\[2pt]
	S_k&=2k^2+3k \qquad (1)\\[12pt]
	S_k&\leq 324 \\[2pt]
	(1)\qquad 2k^2+3k& \leq 324\\[2pt]
	2k^2+3k-324&\leq 0\qquad P=-648 \quad S=3\\[2pt]
	\left(k+\displaystyle\frac{27}{2}\right)(k-12)&\leq 0 \qquad (27,-24)\qquad \left(\displaystyle\frac{27}{2},-12\right)\\[2pt]
	k&=12\\
\end{align*}
Choice 1: \hspace{20pt}$k=11$\hspace{20pt}false\\
Choice 2: \hspace{20pt}$k=15$\hspace{20pt}false\\
Choice 3: \hspace{20pt}$k=14$\hspace{20pt}false\\
Choice 4: \hspace{20pt}$k=13$\hspace{20pt}false\\
Choice 5: \hspace{20pt}$k=12$\hspace{20pt}true\\
\\[4pt]
\noindent\large{\textbf{Lesson 3 Recurrence Relations}}\\[12pt]
\noindent\textbf{Question 1}\hspace{20pt}Experience: 50\hspace{20pt}Order: \hspace{20pt}Level: \hspace{20pt}Question-ID: 57\\[2pt]
A sequence is defined by the recurrence relation $X_{n+1}=\sqrt{k}X_n-2, X_1=2,k>0$, given that $X_3=2$ find the value of $k$.\\[4pt]
\noindent\textbf{Solution 1}\\[2pt]
\\[-35pt]\begin{align*}
X_2&=\sqrt{k}X_1-2\\[2pt]
X_2&=2\sqrt{k}-2\\[12pt]
X_3&=\sqrt{k}X_2-2\\[2pt]
X_3&=\sqrt{k}(2\sqrt{k}-2)-2\\[2pt]
X_3&=2k-2\sqrt{k}-2\quad \text{set}\quad x=\sqrt{k}\\[2pt]
X_3&=2x^2-2x-2 \quad X_3=2\\[2pt]
2&=2x^2-2x-2\\[2pt]
1&=x^2-x-1\\[2pt]
0&=x^2-x-2 \hspace{37pt} S=-1 \quad P=-2\\[2pt]
0&=(x-2)(x+1)\hspace{20pt} (-2,1)\\[2pt]
\sqrt{k}&=2\\[2pt]
k&=4
\end{align*}
Choice 1: \hspace{20pt}$k=5$\hspace{20pt}false\\
Choice 2: \hspace{20pt}$k=3$\hspace{20pt}false\\
Choice 3: \hspace{20pt}$k=6$\hspace{20pt}false\\
Choice 4: \hspace{20pt}$k=7$\hspace{20pt}false\\
Choice 5: \hspace{20pt}$k=4$\hspace{20pt}true\\
\\[4pt]
\noindent\textbf{Question 2}\hspace{20pt}Experience: 50\hspace{20pt}Order: \hspace{20pt}Level: \hspace{20pt}Question-ID: 58\\[2pt]
A sequence is defined by the recurrence relation $U_{n+1}=aU_n+\displaystyle\frac{1}{b}, U_1=3$, given that $U_2=7$ and $U_3=15$ find the value of $a$ and $b$.\\[4pt]
\noindent\textbf{Solution 2}\\[2pt]
\\[-35pt]\begin{align*}
U_2&=aU_1+\displaystyle\frac{1}{b} \quad U_2=7\\[2pt]
7&=3a+\displaystyle\frac{1}{b}\quad (1)\\[12pt]
U_3&=aU_2+\displaystyle\frac{1}{b} \quad U_2=7,U_3=15\\[2pt]
15&=7a+\displaystyle\frac{1}{b}\quad (2)\\[12pt]
(2)-(1)\quad 15-7&=7a+\displaystyle\frac{1}{b}-\left(3a+\displaystyle\frac{1}{b}\right)\\[2pt]
8&=4a\\[2pt]
a&=2\\[2pt]
\text{Sub into}\,\, (1)\quad 7&=3(2)+\displaystyle\frac{1}{b}\\[2pt]
\displaystyle\frac{1}{b}&=1\\[2pt]
b&=1\\[2pt]
\end{align*}
Choice 1: \hspace{20pt}$a=2\quad b=3$\hspace{20pt}false\\
Choice 2: \hspace{20pt}$a=3\quad b=3$\hspace{20pt}false\\
Choice 3: \hspace{20pt}$a=3\quad b=1$\hspace{20pt}false\\
Choice 4: \hspace{20pt}$a=1\quad b=1$\hspace{20pt}false\\
Choice 5: \hspace{20pt}$a=2\quad b=1$\hspace{20pt}true\\
\\[4pt]
\noindent\textbf{Question 3}\hspace{20pt}Experience: 70\hspace{20pt}Order: \hspace{20pt}Level: \hspace{20pt}Question-ID: 63\\[2pt]
A sequence is defined by the recurrence relation $a_{n+1}=ka_n-4,k>0, a_1=5$, given that $\displaystyle\sum_{r=1}^{3} a_r=19$, find the value of $k$.\\[4pt]
\noindent\textbf{Solution 3}\\[2pt]
\\[-35pt]\begin{align*}
a_2&=ka_1-4\\[2pt]
a_2&=5k-4\\[12pt]
a_3&=ka_2-4\\[2pt]
a_3&=k(5k-4)-4\\[2pt]
a_3&=5k^2-4k-4\\[12pt]
\displaystyle\sum_{r=1}^{3} a_r&=a_1+a_2+a_3\\[2pt]
\displaystyle\sum_{r=1}^{3} a_r&=(5)+(5k-4)+(5k^2-4k-4)\\[2pt]
\displaystyle\sum_{r=1}^{3} a_r&=5k^2+k-3\hspace{20pt}\displaystyle\sum_{r=1}^{3} a_r=19\\[2pt]
19&=5k^2+k-3\\[2pt]
0&=5k^2+k-22\hspace{43pt}S=1 \quad P=-110\\[2pt]
0&=\left(k+\displaystyle\frac{11}{5}\right)(k-2)\hspace{20pt}(11,-10)\quad \Rightarrow \quad \left(\displaystyle\frac{11}{5},-2\right)\\[2pt]
k&=2
\end{align*}
Choice 1: \hspace{20pt}$k=3$\hspace{20pt}false\\
Choice 2: \hspace{20pt}$k=4$\hspace{20pt}false\\
Choice 3: \hspace{20pt}$k=1$\hspace{20pt}false\\
Choice 4: \hspace{20pt}$k=5$\hspace{20pt}false\\
Choice 5: \hspace{20pt}$k=2$\hspace{20pt}true\\
\\[4pt]
\noindent\textbf{Question 4}\hspace{20pt}Experience: 60\hspace{20pt}Order: \hspace{20pt}Level: \hspace{20pt}Question-ID: 64\\[2pt]
A sequence is defined by the recurrence relation $U_{n+1}=5U_n-\displaystyle\frac{1}{k},  k>0, U_1=2$, given that $\displaystyle\sum_{r=1}^{4} U_r=293$, find the value of $k$.\\[4pt]
\noindent\textbf{Solution 4}\\[2pt]
\\[-35pt]\begin{align*}
U_2&=5U_1-\displaystyle\frac{1}{k}\\[2pt]
U_2&=5(2)-\displaystyle\frac{1}{k}\\[2pt]
U_2&=10-\displaystyle\frac{1}{k}\\[12pt]
U_3&=5U_2-\displaystyle\frac{1}{k}\\[2pt]
U_3&=5\left(10-\displaystyle\frac{1}{k}\right)-\displaystyle\frac{1}{k}\\[2pt]
U_3&=50-\displaystyle\frac{6}{k}\\[12pt]
U_4&=5U_3-\displaystyle\frac{1}{k}\\[2pt]
U_4&=5\left(50-\displaystyle\frac{6}{k}\right)-\displaystyle\frac{1}{k}\\[2pt]
U_4&=250-\displaystyle\frac{31}{k}\\[12pt]
\displaystyle\sum_{r=1}^{4} U_r&=U_1+U_2+U_3+U_4\\[2pt]
\displaystyle\sum_{r=1}^{4} U_r&=(2)+\left(10-\displaystyle\frac{1}{k}\right)+\left(50-\displaystyle\frac{6}{k}\right)+\left(250-\displaystyle\frac{31}{k}\right)\\[2pt]
\displaystyle\sum_{r=1}^{4} U_r&=312-\displaystyle\frac{38}{k}\quad\displaystyle\sum_{r=1}^{4} U_r=293\\[2pt]
312-\displaystyle\frac{38}{k}&=293\\[2pt]
19&=\displaystyle\frac{38}{k}\\[2pt]
k&=2\\[-30pt]
\end{align*}
Choice 1: \hspace{20pt}$k=5$\hspace{20pt}false\\
Choice 2: \hspace{20pt}$k=4$\hspace{20pt}false\\
Choice 3: \hspace{20pt}$k=3$\hspace{20pt}false\\
Choice 4: \hspace{20pt}$k=1$\hspace{20pt}false\\
Choice 5: \hspace{20pt}$k=2$\hspace{20pt}false\\
\\[4pt]
\noindent\textbf{Question 5}\hspace{20pt}Experience: 100\hspace{20pt}Order: \hspace{20pt}Level: \hspace{20pt}Question-ID: 65\\[2pt]
A sequence is defined by the recurrence relation $X_{n+1}=\displaystyle\frac{k}{X_n}+3, X_1=1$, given that $2\displaystyle\sum_{r=1}^{3} X_r=21$, find the value of $k$.\\[4pt]
\noindent\textbf{Solution 5}\\[2pt]
\\[-35pt]\begin{align*}
X_2&=\displaystyle\frac{k}{X_1}+3\\[2pt]
X_2&=\displaystyle\frac{k}{1}+3\\[2pt]
X_2&=k+3\\[12pt]
X_3&=\displaystyle\frac{k}{X_2}+3\\[2pt]
X_3&=\displaystyle\frac{k}{k+3}+3\\[12pt]
\displaystyle\sum_{r=1}^{3} X_r&=X_1+X_2+X_3\\[2pt]
\displaystyle\sum_{r=1}^{3} X_r&=(1)+(k+3)+\left(\displaystyle\frac{k}{k+3}+3\right)\\[2pt]
\displaystyle\sum_{r=1}^{3} X_r&=k+7+\displaystyle\frac{k}{k+3}\quad 2\displaystyle\sum_{r=1}^{3} X_r=21\\[2pt]
21&=2\left(k+7+\displaystyle\frac{k}{k+3}\right)\\[2pt]
21&=2k+14+\displaystyle\frac{2k}{k+3}\\[2pt]
7&=2k+\displaystyle\frac{2k}{k+3}\\[2pt]
7(k+3)&=2k(k+3)+2k\\[2pt]
7k+21&=2k^2+6k+2k\\[2pt]
0&=2k^2-k-21\hspace{29pt}S=-1\quad P=-42\\[2pt]
0&=\left(k+\displaystyle\frac{7}{2}\right)(k-3)\hspace{10pt}(7,-6)\quad \Rightarrow \quad \left(\displaystyle\frac{7}{2},-3\right)\\[2pt]
k&=3
\end{align*}
Choice 1: \hspace{20pt}$k=5$\hspace{20pt}false\\
Choice 2: \hspace{20pt}$k=2$\hspace{20pt}false\\
Choice 3: \hspace{20pt}$k=4$\hspace{20pt}false\\
Choice 4: \hspace{20pt}$k=1$\hspace{20pt}false\\
Choice 5: \hspace{20pt}$k=3$\hspace{20pt}true\\
\\[4pt]
\noindent\textbf{Question 6}\hspace{20pt}Experience: 35\hspace{20pt}Order: \hspace{20pt}Level: \hspace{20pt}Question-ID: 66\\[2pt]
A sequence is defined by the recurrence relation $a_{n+1}=a_n^2-a_n$, given that $a_n$ is a positive sequence and that $a_3=132$ find the value of $a_1$.\\[4pt]
\noindent\textbf{Solution 6}\\[2pt]
\\[-35pt]\begin{align*}
a_3&=a_2^2-a_2\\[2pt]
132&=a_2^2-a_2\\[2pt]
0&=a_2^2-a_2-132 \hspace{35pt} S=1 \quad P=-132\\[2pt]
0&=(a_2+11)(a_2-12)\hspace{15pt} (11,-12)\\[2pt]
a_2&=12\\[12pt]
a_2&=a_1^2-a_1\\[2pt]
12&=a_1^2-a_1\\[2pt]
0&=a_1^2-a_1-12\\[2pt]
0&=(a_1-4)(a_1+3)\\[2pt]
a_1&=4
\end{align*}
Choice 1: \hspace{20pt}$a_1=6$\hspace{20pt}false\\
Choice 2: \hspace{20pt}$a_1=5$\hspace{20pt}false\\
Choice 3: \hspace{20pt}$a_1=3$\hspace{20pt}false\\
Choice 4: \hspace{20pt}$a_1=7$\hspace{20pt}false\\
Choice 5: \hspace{20pt}$a_1=4$\hspace{20pt}true\\
\\[4pt]
\noindent\textbf{Question 7}\hspace{20pt}Experience: 35\hspace{20pt}Order: \hspace{20pt}Level: \hspace{20pt}Question-ID: 67\\[2pt]
A sequence is defined by the recurrence relation $U_{n+1}=5U_n-\displaystyle\frac{6}{U_n}$, given that  $U_3 =13, U_2 > 0$, find the value of $U_2$.\\[4pt]
\noindent\textbf{Solution 7}\\[2pt]
\\[-35pt]\begin{align*}
U_3&=5U_2-\displaystyle\frac{6}{U_2}\\[2pt]
13&=5U_2-\displaystyle\frac{6}{U_2}\\[2pt]
0&=5U_2-13 -\displaystyle\frac{6}{U_2}\\[2pt]
0&=5(U_2)^2-13U_2 -6\hspace{20pt}S=-13\quad P=-30\\[2pt]
0&=\left(U_2+\displaystyle\frac{2}{5}\right)(U_2-3)\hspace{15pt}(2,-15)\quad \left(\displaystyle\frac{2}{5},-3\right)\\[2pt]
U_2&=3
\end{align*}
Choice 1: \hspace{20pt}$U_2=4$\hspace{20pt}false\\
Choice 2: \hspace{20pt}$U_2=5$\hspace{20pt}false\\
Choice 3: \hspace{20pt}$U_2=2$\hspace{20pt}false\\
Choice 4: \hspace{20pt}$U_2=1$\hspace{20pt}false\\
Choice 5: \hspace{20pt}$U_2=3$\hspace{20pt}true\\
\\[4pt]
\noindent\textbf{Question 8}\hspace{20pt}Experience: 15\hspace{20pt}Order: \hspace{20pt}Level: \hspace{20pt}Question-ID: 68\\[2pt]
A sequence is defined by the recurrence relation $Y_{n+1}=3Y_n-5$, given that  $Y_3 =7$, find the value of $Y_1$.\\[4pt]
\noindent\textbf{Solution 8}\\[2pt]
\\[-35pt]\begin{align*}
Y_3&=3Y_2-5\\[2pt]
7&=3Y_2-5\\[2pt]
Y_2&=4\\[12pt]
Y_2&=3Y_1-5\\[2pt]
4&=3Y_1-5\\[2pt]
Y_1&=3\\
\end{align*}
Choice 1: \hspace{20pt}$Y_1=5$\hspace{20pt}false\\
Choice 2: \hspace{20pt}$Y_1=4$\hspace{20pt}false\\
Choice 3: \hspace{20pt}$Y_1=1$\hspace{20pt}false\\
Choice 4: \hspace{20pt}$Y_1=2$\hspace{20pt}false\\
Choice 5: \hspace{20pt}$Y_1=3$\hspace{20pt}true\\
\\[4pt]
\noindent\textbf{Question 9}\hspace{20pt}Experience: 40\hspace{20pt}Order: \hspace{20pt}Level: \hspace{20pt}Question-ID: 69\\[2pt]
A sequence is defined by the recurrence relation $a_{n+1}=a_n-\displaystyle\frac{2a_n+6}{a_n+3}$, given that  $a_2 =5$, find the value of $a_1$.\\[4pt]
\noindent\textbf{Solution 9}\\[2pt]
\\[-35pt]\begin{align*}
a_2&=a_1-\displaystyle\frac{2a_1+6}{a_1+3}\\[2pt]
5&=a_1-\displaystyle\frac{2a_1+6}{a_1+3}\\[2pt]
5(a_1+3)&=a_1(a_1+3)-(2a_1+6)\\[2pt]
5a_1+15&=(a_1)^2+3a_1-2a_1-6\\[2pt]
0&=(a_1)^2-4a_1-21\hspace{20pt}S=-4 \quad P=-21\\[2pt]
0&=(a_1+3)(a_1-7)\hspace{20pt}(3,-7)\\[2pt]
a_1&=7
\end{align*}
Choice 1: \hspace{20pt}$a_1=8$\hspace{20pt}false\\
Choice 2: \hspace{20pt}$a_1=4$\hspace{20pt}false\\
Choice 3: \hspace{20pt}$a_1=5$\hspace{20pt}false\\
Choice 4: \hspace{20pt}$a_1=6$\hspace{20pt}false\\
Choice 5: \hspace{20pt}$a_1=7$\hspace{20pt}true\\
\\[4pt]
\noindent\textbf{Question 10}\hspace{20pt}Experience: 25\hspace{20pt}Order: \hspace{20pt}Level: \hspace{20pt}Question-ID: 70\\[2pt]
A sequence is defined by the recurrence relation $X_{n+1}=3(X_n)^2-11$, given that  $X_1 =2$, find $\displaystyle\sum_{r=1}^{4} X_r$.\\[4pt]
\noindent\textbf{Solution 10}\\[2pt]
\\[-35pt]\begin{align*}
X_2&=3(X_1)^2-11\\[2pt]
X_2&=3(2)^2-11\\[2pt]
X_2&=1\\[12pt]
X_3&=3(X_2)^2-11\\[2pt]
X_3&=3(1)^2-11\\[2pt]
X_3&=-8\\[12pt]
X_4&=3(X_3)^2-11\\[2pt]
X_4&=3(-8)^2-11\\[2pt]
X_4&=181\\[12pt]
\displaystyle\sum_{r=1}^{4} X_r&=X_1+X_2+X_3+X_4\\[2pt]
\displaystyle\sum_{r=1}^{4} X_r&=(2)+(1)+(-8)+(181)\\[2pt]
\displaystyle\sum_{r=1}^{4} X_r&=176
\end{align*}
Choice 1: \hspace{20pt}$\displaystyle\sum_{r=1}^{4} X_r=173$\hspace{20pt}false\\
Choice 2: \hspace{20pt}$\displaystyle\sum_{r=1}^{4} X_r=170$\hspace{20pt}false\\
Choice 3: \hspace{20pt}$\displaystyle\sum_{r=1}^{4} X_r=177$\hspace{20pt}false\\
Choice 4: \hspace{20pt}$\displaystyle\sum_{r=1}^{4} X_r=172$\hspace{20pt}false\\
Choice 5: \hspace{20pt}$\displaystyle\sum_{r=1}^{4} X_r=176$\hspace{20pt}true\\
\\[4pt]
\noindent\textbf{Question 11}\hspace{20pt}Experience: 25\hspace{20pt}Order: \hspace{20pt}Level: \hspace{20pt}Question-ID: 71\\[2pt]
A sequence is defined by the recurrence relation $U_{n+2}=3U_{n+1}-U_n+5$, given that  $U_1 =4,U_2=2$, find $\displaystyle\sum_{r=1}^{4} U_r$.\\[4pt]
\noindent\textbf{Solution 11}\\[2pt]
\\[-35pt]\begin{align*}
U_3&=3U_2-U_1+5\\[2pt]
U_3&=3(2)-(4)+5\\[2pt]
U_3&=7\\[12pt]
U_4&=3U_3-U_2+5\\[2pt]
U_4&=3(7)-(2)+5\\[2pt]
U_4&=24\\[12pt]
\displaystyle\sum_{r=1}^{4} U_r&=U_1+U_2+U_3+U_4\\[2pt]
\displaystyle\sum_{r=1}^{4} U_r&=4+2+7+24\\[2pt]
\displaystyle\sum_{r=1}^{4} U_r&=37\\[2pt]
\end{align*}
Choice 1: \hspace{20pt}$\displaystyle\sum_{r=1}^{4} U_r=36$\hspace{20pt}false\\
Choice 2: \hspace{20pt}$\displaystyle\sum_{r=1}^{4} U_r=35$\hspace{20pt}false\\
Choice 3: \hspace{20pt}$\displaystyle\sum_{r=1}^{4} U_r=38$\hspace{20pt}false\\
Choice 4: \hspace{20pt}$\displaystyle\sum_{r=1}^{4} U_r=34$\hspace{20pt}false\\
Choice 5: \hspace{20pt}$\displaystyle\sum_{r=1}^{4} U_r=37$\hspace{20pt}true\\
\\[4pt]
\noindent\textbf{Question 12}\hspace{20pt}Experience: 25\hspace{20pt}Order: \hspace{20pt}Level: \hspace{20pt}Question-ID: 72\\[2pt]
A sequence is defined by the recurrence relation $Y_{n+1}=21-2Y_n$, given that  $Y_1 =5$, find $\displaystyle\sum_{r=2}^{4} Y_r$.\\[4pt]
\noindent\textbf{Solution 12}\\[2pt]
\\[-35pt]\begin{align*}
Y_2&=21-2Y_1\\[2pt]
Y_2&=21-2(5)\\[2pt]
Y_2&=11\\[12pt]
Y_3&=21-2Y_2\\[2pt]
Y_3&=21-2(11)\\[2pt]
Y_3&=-1\\[12pt]
Y_4&=21-2Y_3\\[2pt]
Y_4&=21-2(-1)\\[2pt]
Y_4&=23\\[12pt]
\displaystyle\sum_{r=2}^{4} Y_r&=Y_2+Y_3+Y_4\\[2pt]
\displaystyle\sum_{r=2}^{4} Y_r&=11+(-1)+23\\[2pt]
\displaystyle\sum_{r=2}^{4} Y_r&=33
\end{align*}
Choice 1: \hspace{20pt}$\displaystyle\sum_{r=2}^{4} Y_r=32$\hspace{20pt}false\\
Choice 2: \hspace{20pt}$\displaystyle\sum_{r=2}^{4} Y_r=31$\hspace{20pt}false\\
Choice 3: \hspace{20pt}$\displaystyle\sum_{r=2}^{4} Y_r=30$\hspace{20pt}false\\
Choice 4: \hspace{20pt}$\displaystyle\sum_{r=2}^{4} Y_r=34$\hspace{20pt}false\\
Choice 5: \hspace{20pt}$\displaystyle\sum_{r=2}^{4} Y_r=33$\hspace{20pt}true\\
\\[4pt]
\noindent\textbf{Question 13}\hspace{20pt}Experience: 30\hspace{20pt}Order: \hspace{20pt}Level: \hspace{20pt}Question-ID: 74\\[2pt]
A sequence is defined by the recurrence relation $X_{n+1}=5-X_n$, given that  $X_1 =7$, find $\displaystyle\sum_{r=1}^{20} X_r$.\\[4pt]
\noindent\textbf{Solution 13}\\[2pt]
\\[-35pt]\begin{align*}
X_2&=5-X_1=5-7=-2\\[7pt]
X_3&=5-X_2=5-(-2)=7\\[7pt]
X_4&=5-X_3=5-7=-2\\[7pt]
X_5&=5-X_4=5-(-2)=7\\[7pt]
\displaystyle\sum_{r=1}^{20} X_r &= X_1+X_2+X_3+X_4+...+X_{20}\\[2pt]
\displaystyle\sum_{r=1}^{20} X_r &= -2+7+-2+7+-2+...+7\\[2pt]
\displaystyle\sum_{r=1}^{20} X_r &= 10(-2)+10(7)\\[2pt]
\displaystyle\sum_{r=1}^{20} X_r &= 50\\[2pt]
\end{align*}
Choice 1: \hspace{20pt}$\displaystyle\sum_{r=1}^{20} X_r= 20$\hspace{20pt}false\\
Choice 2: \hspace{20pt}$\displaystyle\sum_{r=1}^{20} X_r= 60$\hspace{20pt}false\\
Choice 3: \hspace{20pt}$\displaystyle\sum_{r=1}^{20} X_r= 30$\hspace{20pt}false\\
Choice 4: \hspace{20pt}$\displaystyle\sum_{r=1}^{20} X_r= 40$\hspace{20pt}false\\
Choice 5: \hspace{20pt}$\displaystyle\sum_{r=1}^{20} X_r= 50$\hspace{20pt}true\\
\\[4pt]
\noindent\textbf{Question 14}\hspace{20pt}Experience: 30\hspace{20pt}Order: \hspace{20pt}Level: \hspace{20pt}Question-ID: 77\\[2pt]
A sequence is defined by the recurrence relation $Y_{n+1}=5+5Y_n-2(Y_n)^3$, given that  $Y_1 =2$, find $Y_{1000}$.\\[4pt]
\noindent\textbf{Solution 14}\\[2pt]
\\[-35pt]\begin{align*}
Y_2&=5+5Y_1-2(Y_1)^3=5+5(2)-2(2)^3=-1\\[7pt]
Y_3&=5+5Y_2-2(Y_2)^3=5+5(-1)-2(-1)^3=2\\[7pt]
Y_4&=5+5Y_3-2(Y_3)^3=5+5(2)-2(2)^3=-1\\[7pt]
\end{align*}
\begin{align*}
&Y_1&&Y_2&\,\,\,&Y_3&&Y_4&\,\,\,&Y_5&&Y_6&\\[2pt]
&-1&&\,\,2&&-1&&\,\,2&&-1&&\,\,2&\\
\end{align*}
We can see that $Y_2=Y_4=Y_6=Y_{8}=...=2$$\\[2pt]$
Every numbered term divisible by $2$ is $2$$\\[2pt]$
Find a numbered term that is close to $Y_{1000}$ that is divisible by 2$\\[2pt]$
$Y_2=2\quad Y_4=2\quad Y_{100}=2\quad Y_{1000}=2$$\\[2pt]$\\[4pt]
Choice 1: \hspace{20pt}$Y_{1000}=1$\hspace{20pt}false\\
Choice 2: \hspace{20pt}$Y_{1000}=0$\hspace{20pt}false\\
Choice 3: \hspace{20pt}$Y_{1000}=3$\hspace{20pt}false\\
Choice 4: \hspace{20pt}$Y_{1000}=4$\hspace{20pt}false\\
Choice 5: \hspace{20pt}$Y_{1000}=2$\hspace{20pt}true\\
\\[4pt]
\noindent\textbf{Question 15}\hspace{20pt}Experience: 15\hspace{20pt}Order: \hspace{20pt}Level: \hspace{20pt}Question-ID: 78\\[2pt]
Given $\displaystyle\sum_{r=1}^{n} x_r = 5n^2-3$, find $\displaystyle\sum_{r=1}^{7} x_r$\\[4pt]
\noindent\textbf{Solution 15}\\[2pt]
\\[-35pt]\begin{align*}
\displaystyle\sum_{r=1}^{7} x_r&=5(7)^2-3=242\\[2pt]
\end{align*}
Choice 1: \hspace{20pt}$\displaystyle\sum_{r=1}^{7} x_r=239$\hspace{20pt}false\\
Choice 2: \hspace{20pt}$\displaystyle\sum_{r=1}^{7} x_r=240$\hspace{20pt}false\\
Choice 3: \hspace{20pt}$\displaystyle\sum_{r=1}^{7} x_r=243$\hspace{20pt}false\\
Choice 4: \hspace{20pt}$\displaystyle\sum_{r=1}^{7} x_r=241$\hspace{20pt}false\\
Choice 5: \hspace{20pt}$\displaystyle\sum_{r=1}^{7} x_r=242$\hspace{20pt}true\\
\\[4pt]
\noindent\textbf{Question 16}\hspace{20pt}Experience: 40\hspace{20pt}Order: \hspace{20pt}Level: \hspace{20pt}Question-ID: 76\\[2pt]
A sequence is defined by the recurrence relation $U_{n+1}=\displaystyle\frac{13-5U_n}{7-3U_n}$, given that  $U_1 =1$, find $U_{50}$.\\[4pt]
\noindent\textbf{Solution 16}\\[2pt]
\\[-35pt]\begin{align*}
U_2&=\displaystyle\frac{13-5U_1}{7-3U_1}=\displaystyle\frac{13-5(1)}{7-3(1)}=\displaystyle\frac{8}{4}=2\\[7pt]
U_3&=\displaystyle\frac{13-5U_2}{7-3U_2}=\displaystyle\frac{13-5(2)}{7-3(2)}=\displaystyle\frac{3}{1}=3\\[7pt]
U_4&=\displaystyle\frac{13-5U_3}{7-3U_3}=\displaystyle\frac{13-5(3)}{7-3(3)}=\displaystyle\frac{-2}{-2}=1\\[7pt]
U_5&=\displaystyle\frac{13-5U_4}{7-3U_4}=\displaystyle\frac{13-5(1)}{7-3(1)}=\displaystyle\frac{8}{4}=2\\
\end{align*}
\begin{align*}
&U_1&&U_2&&U_3&&U_4&&U_5&&U_6&\\[2pt]
&1&&2&&3&&1&&2&&3&\\
\end{align*}
We can see that $U_3=U_6=U_9=U_{12}=...=3$$\\[2pt]$
Every numbered term divisible by $3$ is 3$\\[2pt]$
Find a numbered term that is close to $U_{50}$ that is divisible by 3$\\[2pt]$
$U_3=3\quad U_9=3\quad U_{30}=3\quad U_{51}=3$$\\[2pt]$
$U_{51}=3\quad\Rightarrow \quad U_{50}=2$ since 2 is the term before 3 in the sequence$\\[2pt]$ i.e. $1,2,3,1,2,3,1,$$\\$\\[4pt]
Choice 1: \hspace{20pt}$U_{50}=1$\hspace{20pt}false\\
Choice 2: \hspace{20pt}$U_{50}=3$\hspace{20pt}false\\
Choice 3: \hspace{20pt}$U_{50}=4$\hspace{20pt}false\\
Choice 4: \hspace{20pt}$U_{50}=5$\hspace{20pt}false\\
Choice 5: \hspace{20pt}$U_{50}=2$\hspace{20pt}true\\
\\[4pt]
\noindent\textbf{Question 17}\hspace{20pt}Experience: 30\hspace{20pt}Order: \hspace{20pt}Level: \hspace{20pt}Question-ID: 79\\[2pt]
Given $\displaystyle\sum_{r=1}^{n} a_r = 2n^3+5$, find $a_2$\\[4pt]
\noindent\textbf{Solution 17}\\[2pt]
\\[-35pt]\begin{align*}
\displaystyle\sum_{r=1}^{n} a_r &= 2n^3+5\\[2pt]
a_2&=\displaystyle\sum_{r=1}^{2} a_r - \displaystyle\sum_{r=1}^{1} a_r \\[2pt]
a_2&= 2(2)^3+5- (2(1)^3+5) \\[2pt]
a_2&= 21- 7 \\[2pt]
a_2&= 14 \\[2pt]
\end{align*}
Choice 1: \hspace{20pt}$a_2= 13$\hspace{20pt}false\\
Choice 2: \hspace{20pt}$a_2= 12$\hspace{20pt}false\\
Choice 3: \hspace{20pt}$a_2= 11$\hspace{20pt}false\\
Choice 4: \hspace{20pt}$a_2= 15$\hspace{20pt}false\\
Choice 5: \hspace{20pt}$a_2= 14$\hspace{20pt}true\\
\\[4pt]
\noindent\textbf{Question 18}\hspace{20pt}Experience: 15\hspace{20pt}Order: \hspace{20pt}Level: \hspace{20pt}Question-ID: 80\\[2pt]
Given $\displaystyle\sum_{r=1}^{n} U_r = 6n^2+11$, find $U_1$\\[4pt]
\noindent\textbf{Solution 18}\\[2pt]
\\[-35pt]\begin{align*}
\displaystyle\sum_{r=1}^{1} U_r &=U_1=6(1)^2+11=17
\end{align*}
Choice 1: \hspace{20pt}$U_1=16$\hspace{20pt}false\\
Choice 2: \hspace{20pt}$U_1=15$\hspace{20pt}false\\
Choice 3: \hspace{20pt}$U_1=14$\hspace{20pt}false\\
Choice 4: \hspace{20pt}$U_1=18$\hspace{20pt}false\\
Choice 5: \hspace{20pt}$U_1=17$\hspace{20pt}true\\
\\[4pt]
\noindent\textbf{Question 19}\hspace{20pt}Experience: 15\hspace{20pt}Order: \hspace{20pt}Level: \hspace{20pt}Question-ID: 81\\[2pt]
Given $\displaystyle\sum_{r=1}^{n} u_r = n^3+4$, find $\displaystyle\sum_{r=1}^{5} u_r$\\[4pt]
\noindent\textbf{Solution 19}\\[2pt]
\\[-35pt]\begin{align*}
\displaystyle\sum_{r=1}^{5} u_r &= (5)^3+4=129\\[2pt]
\end{align*}
Choice 1: \hspace{20pt}$\displaystyle\sum_{r=1}^{5} u_r=130$\hspace{20pt}false\\
Choice 2: \hspace{20pt}$\displaystyle\sum_{r=1}^{5} u_r=126$\hspace{20pt}false\\
Choice 3: \hspace{20pt}$\displaystyle\sum_{r=1}^{5} u_r=127$\hspace{20pt}false\\
Choice 4: \hspace{20pt}$\displaystyle\sum_{r=1}^{5} u_r=128$\hspace{20pt}false\\
Choice 5: \hspace{20pt}$\displaystyle\sum_{r=1}^{5} u_r=129$\hspace{20pt}true\\
\\[4pt]
\noindent\textbf{Question 20}\hspace{20pt}Experience: 30\hspace{20pt}Order: \hspace{20pt}Level: \hspace{20pt}Question-ID: 82\\[2pt]
Given $\displaystyle\sum_{r=1}^{n} Y_r = 3n^3-2$, find $Y_3$\\[4pt]
\noindent\textbf{Solution 20}\\[2pt]
\\[-35pt]\begin{align*}
Y_3&=\displaystyle\sum_{r=1}^{3} - \displaystyle\sum_{r=1}^{2}\\[2pt]
Y_3&=3(3)^3-2 - (3(2)^3-2)\\[2pt]
Y_3&=57
\end{align*}
Choice 1: \hspace{20pt}$Y_3=54$\hspace{20pt}false\\
Choice 2: \hspace{20pt}$Y_3=55$\hspace{20pt}false\\
Choice 3: \hspace{20pt}$Y_3=52$\hspace{20pt}false\\
Choice 4: \hspace{20pt}$Y_3=56$\hspace{20pt}false\\
Choice 5: \hspace{20pt}$Y_3=57$\hspace{20pt}true\\
\\[4pt]
\noindent\textbf{Question 21}\hspace{20pt}Experience: 30\hspace{20pt}Order: \hspace{20pt}Level: \hspace{20pt}Question-ID: 83\\[2pt]
Given $\displaystyle\sum_{r=1}^{n} U_r = 3n+7$, find $U_5$\\[4pt]
\noindent\textbf{Solution 21}\\[2pt]
\\[-35pt]\begin{align*}
U_5&=\displaystyle\sum_{r=1}^{5} U_r - \displaystyle\sum_{r=1}^{4} U_r\\[2pt]
U_5&=3(5)+7 - (3(4)+7)\\[2pt]
U_5&=3\\
\end{align*}
Choice 1: \hspace{20pt}$U_5=2$\hspace{20pt}false\\
Choice 2: \hspace{20pt}$U_5=1$\hspace{20pt}false\\
Choice 3: \hspace{20pt}$U_5=4$\hspace{20pt}false\\
Choice 4: \hspace{20pt}$U_5=5$\hspace{20pt}false\\
Choice 5: \hspace{20pt}$U_5=3$\hspace{20pt}true\\
\\[4pt]
\noindent\textbf{Question 22}\hspace{20pt}Experience: 45\hspace{20pt}Order: \hspace{20pt}Level: \hspace{20pt}Question-ID: 55\\[2pt]
A sequence is defined by the recurrence relation $U_{n+1}=kU_n-4, U_1=3, k>0$, given that $U_3=0$ find the value of $k$\\[4pt]
\noindent\textbf{Solution 22}\\[2pt]
\\[-35pt]\begin{align*}
U_2&=kU_1-4\\[2pt]
U_2&=3k-4\\[12pt]
U_3&=kU_2-4\\[2pt]
U_3&=k(3k-4)-4\\[2pt]
U_3&=3k^2-4k-4\quad U_3=0\\[12pt]
0&=3k^2-4k-4\hspace{25pt}S=-4\quad P=-12\\[2pt]
0&=\left(k+\frac{2}{3}\right)(k-2) \hspace{13pt}(2,-6) \quad \Rightarrow \quad \left(\displaystyle\frac{2}{3},-2\right)\\[2pt]
k&=2
\end{align*}
Choice 1: \hspace{20pt}$k=3$\hspace{20pt}false\\
Choice 2: \hspace{20pt}$k=4$\hspace{20pt}false\\
Choice 3: \hspace{20pt}$k=1$\hspace{20pt}false\\
Choice 4: \hspace{20pt}$k=5$\hspace{20pt}false\\
Choice 5: \hspace{20pt}$k=2$\hspace{20pt}true\\
\\[4pt]
\noindent\textbf{Question 23}\hspace{20pt}Experience: 60\hspace{20pt}Order: \hspace{20pt}Level: \hspace{20pt}Question-ID: 56\\[2pt]
A sequence is defined by the recurrence relation $a_{n+1}=\displaystyle\frac{a_n}{k}+3, a_1=3,k>0$, given that $a_3=9$ find the value of $k$\\[4pt]
\noindent\textbf{Solution 23}\\[2pt]
\\[-35pt]\begin{align*}
a_2&=\displaystyle\frac{a_1}{k}+3\\[2pt]
a_2&=\displaystyle\frac{3}{k}+3\\[12pt]
a_3&=\displaystyle\frac{a_2}{k}+3\\[2pt]
a_3&=\displaystyle\frac{\left(\displaystyle\frac{3}{k}+3\right)}{k}+3\\[2pt]
a_3&=\displaystyle\frac{3}{k^2}+\frac{3}{k}+3\quad a_3=9\\[12pt]
9&=\displaystyle\frac{3}{k^2}+\frac{3}{k}+3\\[2pt]
6-\displaystyle\frac{3}{k^2}-\frac{3}{k}&=0\\[2pt]
6k^2-3k-3&=0\\[2pt]
2k^2-k-1&=0\hspace{20pt}S=-1\quad P=-2\\[2pt]
\left(x+\displaystyle\frac{1}{2}\right)(k-1)&=0\hspace{20pt} (1,-2) \quad \Rightarrow \quad \left(\displaystyle\frac{1}{2},-1\right)\\[2pt]
k&=1\\
\end{align*}
Choice 1: \hspace{20pt}$k=2$\hspace{20pt}false\\
Choice 2: \hspace{20pt}$k=3$\hspace{20pt}false\\
Choice 3: \hspace{20pt}$k=4$\hspace{20pt}false\\
Choice 4: \hspace{20pt}$k=5$\hspace{20pt}false\\
Choice 5: \hspace{20pt}$k=1$\hspace{20pt}true\\
\\[4pt]
\noindent\textbf{Question 24}\hspace{20pt}Experience: 60\hspace{20pt}Order: \hspace{20pt}Level: \hspace{20pt}Question-ID: 62\\[2pt]
A sequence is defined by the recurrence relation $u_{n+1}=\sqrt{a}\left(u_n-\displaystyle\frac{1}{b}\right),5 u_1=4$, given that $u_2=7$ and $u_3=13$ find the value of $a$ and $b$ .\\[4pt]
\noindent\textbf{Solution 24}\\[2pt]
\\[-35pt]\begin{align*}
u_2&=\sqrt{a}\left(u_1-\displaystyle\frac{1}{b}\right)\\[2pt]
7&=\sqrt{a}\left(4-\displaystyle\frac{1}{b}\right)\hspace{20pt}(1)\\[2pt]
7&=4\sqrt{a}-\displaystyle\frac{\sqrt{a}}{b}\hspace{20pt}(2)\\[12pt]
u_3&=\sqrt{a}\left(u_2-\displaystyle\frac{1}{b}\right)\\[2pt]
13&=\sqrt{a}\left(7-\displaystyle\frac{1}{b}\right)\\[2pt]
13&=7\sqrt{a}-\displaystyle\frac{\sqrt{a}}{b}\hspace{20pt}(3)\\[12pt]
(3)-(2)\quad13-7&=7\sqrt{a}-\displaystyle\frac{\sqrt{a}}{b}-\left(4\sqrt{a}-\displaystyle\frac{\sqrt{a}}{b}\right)\\[2pt]
6&=3\sqrt{a}\\[2pt]
2&=\sqrt{a}\\[2pt]
a&=4\\[12pt]
\text{Sub into} \,\,(1)\quad 7&=\sqrt{4}\left(4-\displaystyle\frac{1}{b}\right)\\[2pt]
\displaystyle\frac{7}{2}&=4-\displaystyle\frac{1}{b}\\[2pt]
-\displaystyle\frac{1}{2}&=-\displaystyle\frac{1}{b}\\[2pt]
b&=2
\end{align*}
Choice 1: \hspace{20pt}$a=4\quad b=2$\hspace{20pt}true\\
Choice 2: \hspace{20pt}$a=3\quad b=2$\hspace{20pt}false\\
Choice 3: \hspace{20pt}$a=4\quad b=3$\hspace{20pt}false\\
Choice 4: \hspace{20pt}$a=3\quad b=3$\hspace{20pt}false\\
Choice 5: \hspace{20pt}$a=2\quad b=3$\hspace{20pt}false\\
\\[4pt]
\noindent\textbf{Question 25}\hspace{20pt}Experience: 30\hspace{20pt}Order: \hspace{20pt}Level: \hspace{20pt}Question-ID: 73\\[2pt]
A sequence is defined by the recurrence relation $a_{n+1}=3-a_n$, given that  $a_1 =1$, find $\displaystyle\sum_{r=1}^{100} a_r$.\\[4pt]
\noindent\textbf{Solution 25}\\[2pt]
\\[-35pt]\begin{align*}
a_2&=3-a_1=3-1=2\\[7pt]
a_3&=3-a_2=3-2=1\\[7pt]
a_4&=3-a_3=3-1=2\\[7pt]
a_5&=3-a_4=3-2=1\\[7pt]
\displaystyle\sum_{r=1}^{100} a_r &= a_1+a_2+a_3+a_4+...+a_{100}\\[2pt]
\displaystyle\sum_{r=1}^{100} a_r &= 1+2+1+2+1+2+...+2\\[2pt]
\displaystyle\sum_{r=1}^{100} a_r &= 50(2)+50(1)\\[2pt]
\displaystyle\sum_{r=1}^{100} a_r &= 150\\[2pt]
\end{align*}
Choice 1: \hspace{20pt}$\displaystyle\sum_{r=1}^{100} a_r = 100$\hspace{20pt}false\\
Choice 2: \hspace{20pt}$\displaystyle\sum_{r=1}^{100} a_r = 200$\hspace{20pt}false\\
Choice 3: \hspace{20pt}$\displaystyle\sum_{r=1}^{100} a_r = 50$\hspace{20pt}false\\
Choice 4: \hspace{20pt}$\displaystyle\sum_{r=1}^{100} a_r = 250$\hspace{20pt}false\\
Choice 5: \hspace{20pt}$\displaystyle\sum_{r=1}^{100} a_r = 150$\hspace{20pt}true\\
\\[4pt]
\noindent\textbf{Question 26}\hspace{20pt}Experience: 40\hspace{20pt}Order: \hspace{20pt}Level: \hspace{20pt}Question-ID: 75\\[2pt]
A sequence is defined by the recurrence relation $A_{n+1}=\displaystyle\frac{4A_n-16}{3A_n-8}$, given that  $A_1 =0$, find $A_{100}$.\\[4pt]
\noindent\textbf{Solution 26}\\[2pt]
\\[-35pt]\begin{align*}
A_2&=\displaystyle\frac{4A_1-16}{3A_1-8}=\displaystyle\frac{4(0)-16}{3(0)-8}=\displaystyle\frac{-16}{-8}=2\\[7pt]
A_3&=\displaystyle\frac{4A_2-16}{3A_2-8}=\displaystyle\frac{4(2)-16}{3(2)-8}=\displaystyle\frac{-8}{-2}=4\\[7pt]
A_4&=\displaystyle\frac{4A_1-16}{3A_1-8}=\displaystyle\frac{4(4)-16}{3(4)-8}=0\\[7pt]
A_5&=\displaystyle\frac{4A_1-16}{3A_1-8}=\displaystyle\frac{4(0)-16}{3(0)-8}=\displaystyle\frac{-16}{-8}=2\\
\end{align*}
$a_1\quad a_2\quad a_3\quad a_4\quad a_5\quad a_6$ $\\[2pt]$
$0\hspace{16pt} 2\hspace{16pt} 4\hspace{14pt} 0\hspace{16pt} 2\hspace{14pt} 4$  $\\[2pt]$
We can see that $a_3=a_6=a_9=a_{12}=...=4$$\\[2pt]$
Every numbered term divisible by $3$ is 4$\\[2pt]$
Find a numbered term that is close to $a_{100}$ that is divisible by 3$\\[2pt]$
$a_3=4\hspace{9.5pt} a_9=4\hspace{9.5pt} a_{30}=4\hspace{9.5pt} a_{99}=4$$\\[2pt]$
$a_{99}=4\hspace{9.5pt} \Rightarrow \hspace{9.5pt} a_{100}=0$ since 0 is the next term after 4 in the sequence$\\[2pt]$ i.e. $0,2,4,0,2,4,0$$\\$\\[4pt]
Choice 1: \hspace{20pt}$a_{100}=1$\hspace{20pt}false\\
Choice 2: \hspace{20pt}$a_{100}=2$\hspace{20pt}false\\
Choice 3: \hspace{20pt}$a_{100}=3$\hspace{20pt}false\\
Choice 4: \hspace{20pt}$a_{100}=4$\hspace{20pt}false\\
Choice 5: \hspace{20pt}$a_{100}=0$\hspace{20pt}true\\
\\[4pt]
\noindent\large{\textbf{Lesson 4 Arithmetic Sequence 2}}\\[12pt]
\noindent\textbf{Question 1}\hspace{20pt}Experience: 50\hspace{20pt}Order: \hspace{20pt}Level: \hspace{20pt}Question-ID: 123\\[2pt]
$U_n$ is an arithmetic sequence with $S_n$ being the sum of the first n terms of the sequence. Given that $U_4=11$ and $U_7=23$, find $S_{11}$\\[4pt]
\noindent\textbf{Solution 1}\\[2pt]
\\[-35pt]\begin{align*}
U_n&=a+(n-1)d\\[2pt]
U_4&=a+(4-1)d=a+3d=11\quad (1)\\[2pt]
U_7&=a+(7-1)d=a+6d=23\quad (2)\\[2pt]
(2)-(1)\quad a+6d-(a+3d)&=23-11\\[2pt]
3d&=12\\[2pt]
d&=4\\[2pt]
\text{Sub into} \quad (1) \quad a+3(4)&=11\\[2pt]
a&=-1\\[12pt]
S_n&=\displaystyle\frac{n}{2}(2(a)+(n-1)d)\\[2pt]
S_{11}&=\displaystyle\frac{11}{2}(2(-1)+(11-1)4)=209\\
\end{align*}
Choice 1: \hspace{20pt}$S_{11}=208$\hspace{20pt}false\\
Choice 2: \hspace{20pt}$S_{11}=205$\hspace{20pt}false\\
Choice 3: \hspace{20pt}$S_{11}=206$\hspace{20pt}false\\
Choice 4: \hspace{20pt}$S_{11}=207$\hspace{20pt}false\\
Choice 5: \hspace{20pt}$S_{11}=209$\hspace{20pt}true\\
\\[4pt]
\noindent\textbf{Question 2}\hspace{20pt}Experience: 50\hspace{20pt}Order: \hspace{20pt}Level: \hspace{20pt}Question-ID: 124\\[2pt]
$U_n$ is an arithmetic sequence with $S_n$ being the sum of the first n terms of the sequence. Given that $U_3=-5$ and $U_5=-11$, find $S_{7}$\\[4pt]
\noindent\textbf{Solution 2}\\[2pt]
\\[-35pt]\begin{align*}
U_n&=a+(n-1)d\\[2pt]
U_3&=a+(3-1)d=a+2d=-5\quad (1)\\[2pt]
U_5&=a+(5-1)d=a+4d=-11\quad (2)\\[2pt]
(2)-(1)\quad a+4d-(a+2d)&=-11-(-5)\\[2pt]
2d&=-6\\[2pt]
d&=-3\\[2pt]
\text{Sub into} \quad (1) \quad a+2(-3)&=-5\\[2pt]
a&=1\\[12pt]
U_{7}&=1+(7-1)(-3)=-17\\[2pt]
S_{7}&=\displaystyle\frac{7}{2}(1+(-17))=-56\\
\end{align*}
Choice 1: \hspace{20pt}$S_{7}=-52$\hspace{20pt}false\\
Choice 2: \hspace{20pt}$S_{7}=-53$\hspace{20pt}false\\
Choice 3: \hspace{20pt}$S_{7}=-54$\hspace{20pt}false\\
Choice 4: \hspace{20pt}$S_{7}=-55$\hspace{20pt}false\\
Choice 5: \hspace{20pt}$S_{7}=-56$\hspace{20pt}true\\
\\[4pt]
\noindent\textbf{Question 3}\hspace{20pt}Experience: 45\hspace{20pt}Order: \hspace{20pt}Level: \hspace{20pt}Question-ID: 140\\[2pt]
The first three terms of an arithmetic sequence are $60,58,56...$, there exists a $k^{\text{th}}$ term which $=0$, find the value of $k$, hence of otherwise find the maximum value of $S_n$\\[4pt]
\noindent\textbf{Solution 3}\\[2pt]
\\[-35pt]\begin{align*}
U_n&=a+(n-1)d\\[2pt]
U_k&=60+(k-1)(-2)=0\\[2pt]
k-1&=30\\[2pt]
k&=31\\[12pt]
\text{maimum value of} \,\,S_n&=S_k\,\, \text{as any term after}\,\, U_k\,\, \text{is negative}\\[2pt]
S_n&=\displaystyle\frac{n}{2}(a+d)\\[2pt]
S_k&=\displaystyle\frac{31}{2}(60+0)\\[2pt]
S_k&=930\\[-80pt]
\end{align*}
Choice 1: \hspace{20pt}$k=28 \quad S_k=935$\hspace{20pt}false\\
Choice 2: \hspace{20pt}$k=29 \quad S_k=915$\hspace{20pt}false\\
Choice 3: \hspace{20pt}$k=30 \quad S_k=920$\hspace{20pt}false\\
Choice 4: \hspace{20pt}$k=32 \quad S_k=925$\hspace{20pt}false\\
Choice 5: \hspace{20pt}$k=31 \quad S_k=930$\hspace{20pt}true\\
\\[4pt]
\noindent\textbf{Question 4}\hspace{20pt}Experience: 50\hspace{20pt}Order: \hspace{20pt}Level: \hspace{20pt}Question-ID: 126\\[2pt]
$U_n$ is an arithmetic sequence with $S_n$ being the sum of the first n terms of the sequence. Given that $S_{5}=85$ and $S_8=184$, find $U_{6}$\\[4pt]
\noindent\textbf{Solution 4}\\[2pt]
\\[-35pt]\begin{align*}
S_n&=\displaystyle\frac{n}{2}(2a+(n-1)d)\\[2pt]
S_5&=\displaystyle\frac{5}{2}(2a+(5-1)d)=85\\[2pt]
2a+4d&=34\quad (1)\\[2pt]
S_8&=\displaystyle\frac{8}{2}(2a+(8-1)d)=184\\[2pt]
2a+7d&=46 \quad (2)\\[2pt]
(2)-(1) \quad 2a+7d-(2a+4d)&=46-34\\[2pt]
3d&=12\\[2pt]
d&=4\\[2pt]
\text{Sub into}\quad (1) \quad 2a+4(4)&=34\\[2pt]
a&=9\\[12pt]
U_6&=a+(6-1)d=9+5(4)=29
\end{align*}
Choice 1: \hspace{20pt}$U_{6}=31$\hspace{20pt}false\\
Choice 2: \hspace{20pt}$U_{6}=30$\hspace{20pt}false\\
Choice 3: \hspace{20pt}$U_{6}=27$\hspace{20pt}false\\
Choice 4: \hspace{20pt}$U_{6}=28$\hspace{20pt}false\\
Choice 5: \hspace{20pt}$U_{6}=29$\hspace{20pt}true\\
\\[4pt]
\noindent\textbf{Question 5}\hspace{20pt}Experience: 50\hspace{20pt}Order: \hspace{20pt}Level: \hspace{20pt}Question-ID: 127\\[2pt]
$U_n$ is an arithmetic sequence with $S_n$ being the sum of the first n terms of the sequence. Given that $U_{5}=19$ and $S_{10}=170$, find $U_{4}$\\[4pt]
\noindent\textbf{Solution 5}\\[2pt]
\\[-35pt]\begin{align*}
U_n&=a+(n-1)d\\[2pt]
U_5&=a+4d=19 \hspace{62pt} (1)\\[2pt]
S_n&=\displaystyle\frac{n}{2}(2a+(n-1)d)\\[2pt]
S_{10}&=\displaystyle\frac{10}{2}(2a+(10-1)d)=170\\[2pt]
2a+9d&=34\hspace{103pt} (2)\\[2pt]
(2)-2(1) \quad 2a+9d-2(a+4d)&=34-38\\[2pt]
d&=-4\\[2pt]
d&=-4\\[2pt]
\text{Sub into}\quad (1) \quad a+4(-4)&=19\\[2pt]
a&=35\\[12pt]
U_4&=a+(4-1)d=35+(3)(-4)=23
\end{align*}
Choice 1: \hspace{20pt}$U_{4}=19$\hspace{20pt}false\\
Choice 2: \hspace{20pt}$U_{4}=20$\hspace{20pt}false\\
Choice 3: \hspace{20pt}$U_{4}=21$\hspace{20pt}false\\
Choice 4: \hspace{20pt}$U_{4}=22$\hspace{20pt}false\\
Choice 5: \hspace{20pt}$U_{4}=23$\hspace{20pt}true\\
\\[4pt]
\noindent\textbf{Question 6}\hspace{20pt}Experience: 50\hspace{20pt}Order: \hspace{20pt}Level: \hspace{20pt}Question-ID: 128\\[2pt]
$U_n$ is an arithmetic sequence with $S_n$ being the sum of the first n terms of the sequence. Given that $U_{4}=8$ and $S_{12}=0$, find $S_{9}$\\[4pt]
\noindent\textbf{Solution 6}\\[2pt]
\\[-35pt]\begin{align*}
U_n&=a+(n-1)d\\[2pt]
U_4&=a+3d=8 \hspace{62pt} (1)\\[2pt]
S_n&=\displaystyle\frac{n}{2}(2a+(n-1)d)\\[2pt]
S_{12}&=\displaystyle\frac{12}{2}(2a+(12-1)d)=0\\[2pt]
2a+11d&=0\hspace{103pt} (2)\\[2pt]
(2)-2(1) \quad 2a+11d-2(a+3d)&=0-16\\[2pt]
8d&=-16\\[2pt]
d&=-2\\[2pt]
\text{Sub into}\quad (1) \quad a+3(-2)&=8\\[2pt]
a&=14\\[12pt]
S_9&=\displaystyle\frac{9}{2}(2(14)+(9-1)(-2))=54
\end{align*}
Choice 1: \hspace{20pt}$S_{9}=55$\hspace{20pt}false\\
Choice 2: \hspace{20pt}$S_{9}=51$\hspace{20pt}false\\
Choice 3: \hspace{20pt}$S_{9}=52$\hspace{20pt}false\\
Choice 4: \hspace{20pt}$S_{9}=53$\hspace{20pt}false\\
Choice 5: \hspace{20pt}$S_{9}=54$\hspace{20pt}true\\
\\[4pt]
\noindent\textbf{Question 7}\hspace{20pt}Experience: 50\hspace{20pt}Order: \hspace{20pt}Level: \hspace{20pt}Question-ID: 129\\[2pt]
$U_n$ is an arithmetic sequence with $S_n$ being the sum of the first n terms of the sequence. Given that $U_{3}=4$ and $U_{7}=0$, find $S_{10}$\\[4pt]
\noindent\textbf{Solution 7}\\[2pt]
\\[-35pt]\begin{align*}
U_n&=a+(n-1)d\\[2pt]
U_3&=a+2d=4 \quad (1)\\[2pt]
U_7&=a+6d=0 \quad (2)\\[2pt]
(2)-(1) \quad a+6d-(a+2d)&=0-4\\[2pt]
4d&=-4\\[2pt]
d&=-1\\[2pt]
\text{Sub into}\quad (1) \quad a+2(-1)&=4\\[2pt]
a&=6\\[12pt]
S_n&=\displaystyle\frac{n}{2}(2a+(n-1)d)\\[2pt]
S_{10}&=\displaystyle\frac{10}{2}(2(6)+(10-1)(-1))=15
\end{align*}
Choice 1: \hspace{20pt}$S_{10}=14$\hspace{20pt}false\\
Choice 2: \hspace{20pt}$S_{10}=13$\hspace{20pt}false\\
Choice 3: \hspace{20pt}$S_{10}=12$\hspace{20pt}false\\
Choice 4: \hspace{20pt}$S_{10}=16$\hspace{20pt}false\\
Choice 5: \hspace{20pt}$S_{10}=15$\hspace{20pt}true\\
\\[4pt]
\noindent\textbf{Question 8}\hspace{20pt}Experience: 50\hspace{20pt}Order: \hspace{20pt}Level: \hspace{20pt}Question-ID: 130\\[2pt]
$U_n$ is an arithmetic sequence with $S_n$ being the sum of the first n terms of the sequence. Given that $U_{4}=10$ and $S_{6}=57$, find $S_{11}$\\[4pt]
\noindent\textbf{Solution 8}\\[2pt]
\\[-35pt]\begin{align*}
U_n&=a+(n-1)d\\[2pt]
U_4&=a+3d=10 \hspace{85pt} (1)\\[2pt]
S_n&=\displaystyle\frac{n}{2}(2a+(n-1)d)\\[2pt]
S_6&=\displaystyle\frac{6}{2}(2a+(5-1)d)=57\\[2pt]
a+2d&=\displaystyle\frac{19}{2}\hspace{124pt}(2)\\[2pt]
(1)-(2) \quad a+3d-(a+2d)&=10-\displaystyle\frac{19}{2}\\[2pt]
d&=\displaystyle\frac{1}{2}\\[2pt]
\text{Sub into}\quad (1) \quad a+3\left(\displaystyle\frac{1}{2}\right)&=10\\[2pt]
a&=\displaystyle\frac{17}{2}\\[12pt]
S_{11}&=\displaystyle\frac{11}{2}\left(2\left(\displaystyle\frac{17}{2}\right)+(11-1)\left(\displaystyle\frac{1}{2}\right)\right)=121
\end{align*}
Choice 1: \hspace{20pt}$S_{10}=14$\hspace{20pt}false\\
Choice 2: \hspace{20pt}$S_{10}=13$\hspace{20pt}false\\
Choice 3: \hspace{20pt}$S_{10}=12$\hspace{20pt}false\\
Choice 4: \hspace{20pt}$S_{10}=16$\hspace{20pt}false\\
Choice 5: \hspace{20pt}$S_{10}=15$\hspace{20pt}true\\
\\[4pt]
\noindent\textbf{Question 9}\hspace{20pt}Experience: 30\hspace{20pt}Order: \hspace{20pt}Level: \hspace{20pt}Question-ID: 131\\[2pt]
Three consecutive terms in an arithmetic sequence are $3k+2,2k+5,4k+5$, find the value of $k$\\[4pt]
\noindent\textbf{Solution 9}\\[2pt]
\\[-35pt]\begin{align*}
2k+5-(3k+2)=d&=4k+5-(2k+5)\\[2pt]
-k+3&=2k\\[2pt]
3k&=3\\[2pt]
k&=1\\[-70pt]
\end{align*}
Choice 1: \hspace{20pt}$k=5$\hspace{20pt}false\\
Choice 2: \hspace{20pt}$k=4$\hspace{20pt}false\\
Choice 3: \hspace{20pt}$k=3$\hspace{20pt}false\\
Choice 4: \hspace{20pt}$k=2$\hspace{20pt}false\\
Choice 5: \hspace{20pt}$k=1$\hspace{20pt}true\\
\\[4pt]
\noindent\textbf{Question 10}\hspace{20pt}Experience: 30\hspace{20pt}Order: \hspace{20pt}Level: \hspace{20pt}Question-ID: 132\\[2pt]
Three consecutive terms in an arithmetic sequence are $k^2+3,-k,k-1$, find the possible values of $k$\\[4pt]
\noindent\textbf{Solution 10}\\[2pt]
\\[-35pt]\begin{align*}
-k-(k^2+3)=d&=k-1-(-k)\\[2pt]
-k-k^2-3&=2k-1\\[2pt]
0&=k^2+3k+2\\[2pt]
0&=(k+2)(k+1)\\[-70pt]
\end{align*}
Choice 1: \hspace{20pt}$k=-2,-3$\hspace{20pt}false\\
Choice 2: \hspace{20pt}$k=-1,-2$\hspace{20pt}false\\
Choice 3: \hspace{20pt}$k=-3,-1$\hspace{20pt}false\\
Choice 4: \hspace{20pt}$k=-1,-2$\hspace{20pt}false\\
Choice 5: \hspace{20pt}$k=-2,-1$\hspace{20pt}true\\
\\[4pt]
\noindent\textbf{Question 11}\hspace{20pt}Experience: 30\hspace{20pt}Order: \hspace{20pt}Level: \hspace{20pt}Question-ID: 133\\[2pt]
Three consecutive terms in an arithmetic sequence are $k+16,3k+12,7k-2$, find the value of $k$\\[4pt]
\noindent\textbf{Solution 11}\\[2pt]
\\[-35pt]\begin{align*}
3k+12-(k+16)=d&=7k-2-(3k+12)\\[2pt]
2k-4&=4k-14\\[2pt]
2k&=10\\[2pt]
k&=5\\[-70pt]
\end{align*}
Choice 1: \hspace{20pt}$k=6$\hspace{20pt}false\\
Choice 2: \hspace{20pt}$k=2$\hspace{20pt}false\\
Choice 3: \hspace{20pt}$k=3$\hspace{20pt}false\\
Choice 4: \hspace{20pt}$k=4$\hspace{20pt}false\\
Choice 5: \hspace{20pt}$k=5$\hspace{20pt}true\\
\\[4pt]
\noindent\textbf{Question 12}\hspace{20pt}Experience: 45\hspace{20pt}Order: \hspace{20pt}Level: \hspace{20pt}Question-ID: 134\\[2pt]
The first three terms in an arithmetic sequence are $2k,k+9,3k$, find the smallest $n$ such that $S_n > 117$\\[4pt]
\noindent\textbf{Solution 12}\\[2pt]
\\[-35pt]\begin{align*}
k+9-2k&=d=3k-(k+9)\\[2pt]
-k+9&=2k-9\\[2pt]
3k&=18\\[2pt]
k&=6\\[12pt]
\Rightarrow\quad U_1&=12 \quad U_2=15 \quad U_3=18\\[2pt]
S_n&=\displaystyle\frac{n}{2}(2a+(n-1)d)\\[2pt]
\displaystyle\frac{n}{2}(2(12)+(n-1)3)&>117\\[2pt]
n(24+3n-3)&>234\\[2pt]
3n^2+21n-234&>0\\[2pt]
n^2+7n-78&>0\quad P=-78 \quad S=7\\[2pt]
(n+13)(n-6)&>0 \quad (13,-6)\\[2pt]
n&=6
\end{align*}
Choice 1: \hspace{20pt}$n=7$\hspace{20pt}false\\
Choice 2: \hspace{20pt}$n=3$\hspace{20pt}false\\
Choice 3: \hspace{20pt}$n=4$\hspace{20pt}false\\
Choice 4: \hspace{20pt}$n=5$\hspace{20pt}false\\
Choice 5: \hspace{20pt}$n=6$\hspace{20pt}true\\
\\[4pt]
\noindent\textbf{Question 13}\hspace{20pt}Experience: 40\hspace{20pt}Order: \hspace{20pt}Level: \hspace{20pt}Question-ID: 135\\[2pt]
The first three terms of an arithmetic sequence are $99,96,93...$, there exists a $k^{\text{th}}$ term which $=0$, find the value of $k$, hence of otherwise find the maximum value of $S_n$\\[4pt]
\noindent\textbf{Solution 13}\\[2pt]
\\[-35pt]\begin{align*}
U_n&=a+(n-1)d\\[2pt]
U_k&=99+(k-1)(-3)=0\\[2pt]
k-1&=33\\[2pt]
k&=34\\[12pt]
\text{maimum value of} \,\,S_n&=S_k\,\, \text{as any term after}\,\, U_k\,\, \text{is negative}\\[2pt]
S_n&=\displaystyle\frac{n}{2}(a+l)\\[2pt]
S_k&=\displaystyle\frac{34}{2}(99+0)\\[2pt]
S_k&=1683
\end{align*}
Choice 1: \hspace{20pt}$k=33 \quad S_k=1689$\hspace{20pt}false\\
Choice 2: \hspace{20pt}$k=32 \quad S_k=1686$\hspace{20pt}false\\
Choice 3: \hspace{20pt}$k=35 \quad S_k=1677$\hspace{20pt}false\\
Choice 4: \hspace{20pt}$k=36 \quad S_k=1680$\hspace{20pt}false\\
Choice 5: \hspace{20pt}$k=34 \quad S_k=1683$\hspace{20pt}true\\
\\[4pt]
\noindent\textbf{Question 14}\hspace{20pt}Experience: 35\hspace{20pt}Order: \hspace{20pt}Level: \hspace{20pt}Question-ID: 136\\[2pt]
The first three terms in an arithmetic sequence are $5,7,9$, find the smallest $n$ such that $S_n > 252$\\[4pt]
\noindent\textbf{Solution 14}\\[2pt]
\\[-35pt]\begin{align*}
S_n&=\displaystyle\frac{n}{2}(2a+(n-1)d)\\[2pt]
S_n&=\displaystyle\frac{n}{2}(2(5)+(n-1)2)>252\\[2pt]
n(5+n-1)&>252\\[2pt]
n^2+4n-252&>0 \quad P=-252\quad S=4\\[2pt]
(n+18)(n-14)&>0 \quad (18,-14)\\[2pt]
n&=14
\end{align*}
Choice 1: \hspace{20pt}$n=15$\hspace{20pt}false\\
Choice 2: \hspace{20pt}$n=11$\hspace{20pt}false\\
Choice 3: \hspace{20pt}$n=12$\hspace{20pt}false\\
Choice 4: \hspace{20pt}$n=13$\hspace{20pt}false\\
Choice 5: \hspace{20pt}$n=14$\hspace{20pt}true\\
\\[4pt]
\noindent\textbf{Question 15}\hspace{20pt}Experience: 35\hspace{20pt}Order: \hspace{20pt}Level: \hspace{20pt}Question-ID: 137\\[2pt]
The first three terms in an arithmetic sequence are $9,12,15$, find the smallest $n$ such that $S_n > 750$\\[4pt]
\noindent\textbf{Solution 15}\\[2pt]
\\[-35pt]\begin{align*}
S_n&=\displaystyle\frac{n}{2}(2a+(n-1)d)\\[2pt]
S_n&=\displaystyle\frac{n}{2}(2(9)+(n-1)3)>750\\[2pt]
n(18+3n-3)&>1500\\[2pt]
3n(5+n)&>1500\\[2pt]
n(5+n)&>500\\[2pt]
n^2+5n-500&>0 \quad P=-500\quad S=5\\[2pt]
(n+25)(n-20)&>0 \quad (25,-20)\\[2pt]
n&=20
\end{align*}
Choice 1: \hspace{20pt}$n=21$\hspace{20pt}false\\
Choice 2: \hspace{20pt}$n=17$\hspace{20pt}false\\
Choice 3: \hspace{20pt}$n=18$\hspace{20pt}false\\
Choice 4: \hspace{20pt}$n=19$\hspace{20pt}false\\
Choice 5: \hspace{20pt}$n=20$\hspace{20pt}true\\
\\[4pt]
\noindent\textbf{Question 16}\hspace{20pt}Experience: 35\hspace{20pt}Order: \hspace{20pt}Level: \hspace{20pt}Question-ID: 138\\[2pt]
The first three terms in an arithmetic sequence are $12,16,20,24$, find the smallest $n$ such that $S_n > 672$\\[4pt]
\noindent\textbf{Solution 16}\\[2pt]
\\[-35pt]\begin{align*}
S_n&=\displaystyle\frac{n}{2}(2a+(n-1)d)\\[2pt]
S_n&=\displaystyle\frac{n}{2}(2(12)+(n-1)4)>672\\[2pt]
n(12+2n-2)&>672\\[2pt]
2n^2+10n-672&>0\\[2pt]
n^2+5n-336&>0 \quad P=-336\quad S=5\\[2pt]
(n+21)(n-16)&>0 \quad (21,-16)\\[2pt]
n&=16
\end{align*}
Choice 1: \hspace{20pt}$n=15$\hspace{20pt}false\\
Choice 2: \hspace{20pt}$n=19$\hspace{20pt}false\\
Choice 3: \hspace{20pt}$n=18$\hspace{20pt}false\\
Choice 4: \hspace{20pt}$n=17$\hspace{20pt}false\\
Choice 5: \hspace{20pt}$n=16$\hspace{20pt}true\\
\\[4pt]
\noindent\textbf{Question 17}\hspace{20pt}Experience: 50\hspace{20pt}Order: \hspace{20pt}Level: \hspace{20pt}Question-ID: 142\\[2pt]
Judith is playing with 294 sticks, she puts them in rows. The first row has 8 sticks, next row has 10 sticks, subsequent rows have 2 more sticks then the previous row. She has enough for $k$ rows but not enough for $k+1$ rows. Find k.\\[4pt]
\noindent\textbf{Solution 17}\\[2pt]
Sequence goes: 8,10,12,14,18,20....$\\[2pt]$
Not having enough for k+1 rows means that $S_k\leq294$
\begin{align*}
S_n&=\displaystyle\frac{n}{2}(2a+(k-1)d)\\[2pt]
S_k&=\displaystyle\frac{k}{2}(2(8)+(k-1)2)\\[2pt]
S_k&=k(8+k-1)\\[2pt]
S_k&=k(k+7)\\[2pt]
S_k&=k^2+7k \qquad (1)\\[12pt]
S_k&\leq 294 \\[2pt]
(1)\qquad k^2+7k& \leq 294\\[2pt]
k^2+7k-294&\leq 0\qquad P=294 \quad S=7\\[2pt]
(k+21)(k-14)&\leq 0 \qquad (21,-14)\\[2pt]
k&=14\\[-70pt]
\end{align*}
Choice 1: \hspace{20pt}$k=11$\hspace{20pt}false\\
Choice 2: \hspace{20pt}$k=15$\hspace{20pt}false\\
Choice 3: \hspace{20pt}$k=12$\hspace{20pt}false\\
Choice 4: \hspace{20pt}$k=13$\hspace{20pt}false\\
Choice 5: \hspace{20pt}$k=14$\hspace{20pt}true\\
\\[4pt]
\noindent\textbf{Question 18}\hspace{20pt}Experience: 50\hspace{20pt}Order: \hspace{20pt}Level: \hspace{20pt}Question-ID: 125\\[2pt]
$U_n$ is an arithmetic sequence with $S_n$ being the sum of the first n terms of the sequence. Given that $S_{11}=0$ and $U_2=8$, find $U_{6}$\\[4pt]
\noindent\textbf{Solution 18}\\[2pt]
\\[-35pt]\begin{align*}
U_n&=a+(n-1)d\\[2pt]
U_2&=a+(2-1)d=a+d=8\hspace{59pt} (1)\\[2pt]
S_n&=\displaystyle\frac{n}{2}(2a+(n-1)d)\\[2pt]
S_{11}&=\displaystyle\frac{11}{2}(2a+(11-1)d)=0\\[2pt]
S_{11}&=a+5d=0\hspace{119pt} (2)\\[2pt]
(2)-(1)\quad a+5d-(a+d)&=0-8\\[2pt]
4d&=-8\\[2pt]
d&=-2\\[2pt]
\text{Sub into} \quad (1) \hspace{20pt} a+(-2)&=8\\[2pt]
a&=10\\[12pt]
U_{6}&=1+(7-1)(-2)=-11\\[2pt]
\end{align*}
Choice 1: \hspace{20pt}$U_{6}=-13$\hspace{20pt}false\\
Choice 2: \hspace{20pt}$U_{6}=-12$\hspace{20pt}false\\
Choice 3: \hspace{20pt}$U_{6}=-9$\hspace{20pt}false\\
Choice 4: \hspace{20pt}$U_{6}=-10$\hspace{20pt}false\\
Choice 5: \hspace{20pt}$U_{6}=-11$\hspace{20pt}true\\
\\[4pt]
\noindent\textbf{Question 19}\hspace{20pt}Experience: 45\hspace{20pt}Order: \hspace{20pt}Level: \hspace{20pt}Question-ID: 139\\[2pt]
The first three terms of an arithmetic sequence are $44,41,38...$, there exists a $k^{\text{th}}$ term which is the smallest positive term in the sequence, find the value of $k$, hence of otherwise find the maximum value of $S_n$\\[4pt]
\noindent\textbf{Solution 19}\\[2pt]
\\[-35pt]\begin{align*}
U_n&=a+(n-1)d\\[2pt]
U_k&=44+(k-1)(-3)=0\\[2pt]
k-1&=\displaystyle\frac{44}{3}\\[2pt]
k&=\displaystyle\frac{44}{3}+1=15.6\\[2pt]
k&=15\\[12pt]
\text{maimum value of} \,\,S_n&=S_k\,\, \text{as any term after}\,\, U_k\,\, \text{is negative}\\[2pt]
S_n&=\displaystyle\frac{n}{2}(2a+(n-1)d)\\[2pt]
S_k&=\displaystyle\frac{34}{2}(2(99)+(15-1)(-3))\\[2pt]
S_k&=2652\\[-90pt]
\end{align*}
Choice 1: \hspace{20pt}$k=14 \quad S_k=2658$\hspace{20pt}false\\
Choice 2: \hspace{20pt}$k=13 \quad S_k=2656$\hspace{20pt}false\\
Choice 3: \hspace{20pt}$k=12 \quad S_k=2654$\hspace{20pt}false\\
Choice 4: \hspace{20pt}$k=16 \quad S_k=2650$\hspace{20pt}false\\
Choice 5: \hspace{20pt}$k=15 \quad S_k=2652$\hspace{20pt}true\\
\\[4pt]
\noindent\textbf{Question 20}\hspace{20pt}Experience: 50\hspace{20pt}Order: \hspace{20pt}Level: \hspace{20pt}Question-ID: 141\\[2pt]
At the start of the year 2000, Tony the farmer has $50m^2$ of land, he buys $7m^2$ of land at the end of each year. At the beginning of this year, Tony owns $141m^2$ of land. What year is it?\\[4pt]
\noindent\textbf{Solution 20}\\[2pt]
Sequence goes from the start of every year: 50,57,64,71,78,85....
\begin{align*}
U_n&=a+(n-1)d\\[2pt]
U_n&=141 \quad a=50 \quad d=7\\[2pt]
141&=50+(n-1)7\\[2pt]
n-1&=13\\[2pt]
n&=14\\[2pt]
\text{Year}&=2000+14=2014\\[-70pt]
\end{align*}
Choice 1: \hspace{20pt}Year$\,$$=2015$\hspace{20pt}false\\
Choice 2: \hspace{20pt}Year$\,$$=2011$\hspace{20pt}false\\
Choice 3: \hspace{20pt}Year$\,$$=2012$\hspace{20pt}false\\
Choice 4: \hspace{20pt}Year$\,$$=2013$\hspace{20pt}false\\
Choice 5: \hspace{20pt}Year$\,$$=2014$\hspace{20pt}true\\
\\[4pt]
\\[2pt]
\noindent\large{\textbf{End of Chapter Questions}}\\[15pt]
\noindent\huge{\textbf{Unit 2 Core 2}}\\[18pt]
\noindent\huge{\textbf{Chapter 1 Logarithms}}\\[15pt]
\noindent\large{\textbf{Lesson 1 Basic logarithms}}\\[12pt]
\noindent\textbf{Question 1}\hspace{20pt}Experience: 20\hspace{20pt}Order: f2\hspace{20pt}Level: f2\hspace{20pt}Question-ID: 196\\[2pt]
Express $\log_{9}30$ in terms of $\ln$\\[4pt]
\noindent\textbf{Solution 1}\\[2pt]
\\[-35pt]\begin{align*}
&\log_{9}30\\[2pt]
=&\displaystyle\frac{\log_{e}30}{\log_{e}9}\\[2pt]
=&\displaystyle\frac{\ln30}{\ln9}\\[-100pt]
\end{align*}
Choice 1: \hspace{20pt}$\displaystyle\frac{\ln10}{\ln3}$\hspace{20pt}false\\
Choice 2: \hspace{20pt}$\ln{21}$\hspace{20pt}false\\
Choice 3: \hspace{20pt}$\ln{39}$\hspace{20pt}false\\
Choice 4: \hspace{20pt}$\ln{\displaystyle\frac{10}{3}}$\hspace{20pt}false\\
Choice 5: \hspace{20pt}$\displaystyle\frac{\ln30}{\ln9}$\hspace{20pt}true\\
\\[4pt]
\noindent\textbf{Question 2}\hspace{20pt}Experience: 25\hspace{20pt}Order: g1\hspace{20pt}Level: g1\hspace{20pt}Question-ID: 197\\[2pt]
Solve $4^x=16$ for $x$\\[4pt]
\noindent\textbf{Solution 2}\\[2pt]
\\[-35pt]\begin{align*}
4^x&=16\\[2pt]
x&=\log_{4}16\\[2pt]
x&=\displaystyle\frac{\log16}{\log4}\\[2pt]
x&=2\\[-105pt]
\end{align*}
Answer part 1: \hspace{10pt}Label\hspace{10pt}$x=$\hspace{10pt}Solution\hspace{10pt}2\\
Answer part 1 hint: \hspace{15pt}$x$ is an integer value\\
\\[4pt]
\noindent\textbf{Question 3}\hspace{20pt}Experience: 10\hspace{20pt}Order: a2\hspace{20pt}Level: a2\hspace{20pt}Question-ID: 149\\[2pt]
Express $\log_{x+5}10=4$ in power form\\[4pt]
\noindent\textbf{Solution 3}\\[2pt]
\\[-35pt]\begin{align*}
\log_{x+5}10&=4\\[2pt]
(x+5)^4&=10
\end{align*}
Choice 1: \hspace{20pt}$4^{x+5}=10$\hspace{20pt}false\\
Choice 2: \hspace{20pt}$(x+5)^{10}=4$\hspace{20pt}false\\
Choice 3: \hspace{20pt}$10^{x+5}=4$\hspace{20pt}false\\
Choice 4: \hspace{20pt}$(x+5)^{10}=4$\hspace{20pt}false\\
Choice 5: \hspace{20pt}$(x+5)^4=10$\hspace{20pt}true\\
\\[4pt]
\noindent\textbf{Question 4}\hspace{20pt}Experience: 10\hspace{20pt}Order: a2\hspace{20pt}Level: a2\hspace{20pt}Question-ID: 150\\[2pt]
Express $\log_{a+b}6=c$ in power form\\[4pt]
\noindent\textbf{Solution 4}\\[2pt]
\\[-35pt]\begin{align*}
\log_{a+b}6&=c\\[2pt]
(a+b)^c&=6
\end{align*}
Choice 1: \hspace{20pt}$(a+b)^6=c$\hspace{20pt}false\\
Choice 2: \hspace{20pt}$6^c=a+b$\hspace{20pt}false\\
Choice 3: \hspace{20pt}$(a+b)^c=6$\hspace{20pt}false\\
Choice 4: \hspace{20pt}$6^{a+b}=6$\hspace{20pt}false\\
Choice 5: \hspace{20pt}$(a+b)^c=6$\hspace{20pt}true\\
\\[4pt]
\noindent\textbf{Question 5}\hspace{20pt}Experience: 10\hspace{20pt}Order: a2\hspace{20pt}Level: a2\hspace{20pt}Question-ID: 152\\[2pt]
Express $\log_{xy}3=2$ in power form\\[4pt]
\noindent\textbf{Solution 5}\\[2pt]
\\[-35pt]\begin{align*}
\log_{xy}3&=2\\[2pt]
(xy)^2&=3
\end{align*}
Choice 1: \hspace{20pt}$2^{xy}=3$\hspace{20pt}false\\
Choice 2: \hspace{20pt}$3^2=xy$\hspace{20pt}false\\
Choice 3: \hspace{20pt}$xy^{3}=2$\hspace{20pt}false\\
Choice 4: \hspace{20pt}$(3)^{xy}=2$\hspace{20pt}false\\
Choice 5: \hspace{20pt}$(xy)^2=3$\hspace{20pt}true\\
\\[4pt]
\noindent\textbf{Question 6}\hspace{20pt}Experience: 10\hspace{20pt}Order: b1\hspace{20pt}Level: b1\hspace{20pt}Question-ID: 154\\[2pt]
Express $a^b=c$ in log form\\[4pt]
\noindent\textbf{Solution 6}\\[2pt]
\\[-35pt]\begin{align*}
a^b&=c\\[2pt]
\log_ac&=b
\end{align*}
Choice 1: \hspace{20pt}$\log_ca=b$\hspace{20pt}false\\
Choice 2: \hspace{20pt}$\log_bc=a$\hspace{20pt}false\\
Choice 3: \hspace{20pt}$\log_ba=c$\hspace{20pt}false\\
Choice 4: \hspace{20pt}$\log_ab=c$\hspace{20pt}false\\
Choice 5: \hspace{20pt}$\log_ac=b$\hspace{20pt}true\\
\\[4pt]
\noindent\textbf{Question 7}\hspace{20pt}Experience: 10\hspace{20pt}Order: b1\hspace{20pt}Level: b1\hspace{20pt}Question-ID: 157\\[2pt]
Express $5^2=25$ in log form\\[4pt]
\noindent\textbf{Solution 7}\\[2pt]
\\[-35pt]\begin{align*}
5^2&=25\\[2pt]
\log_{5}25&=2
\end{align*}
Choice 1: \hspace{20pt}$\log_{5}2=25$\hspace{20pt}false\\
Choice 2: \hspace{20pt}$\log_{25}2=5$\hspace{20pt}false\\
Choice 3: \hspace{20pt}$\log_{25}5=2$\hspace{20pt}false\\
Choice 4: \hspace{20pt}$\log_{2}25=5$\hspace{20pt}false\\
Choice 5: \hspace{20pt}$\log_{5}25=2$\hspace{20pt}true\\
\\[4pt]
\noindent\textbf{Question 8}\hspace{20pt}Experience: 10\hspace{20pt}Order: b2\hspace{20pt}Level: b2\hspace{20pt}Question-ID: 156\\[2pt]
Express $(xy)^5=20$ in log form\\[4pt]
\noindent\textbf{Solution 8}\\[2pt]
\\[-35pt]\begin{align*}
(xy)^5&=20\\[2pt]
\log_{xy}20&=5
\end{align*}
Choice 1: \hspace{20pt}$\log_{5}20=xy$\hspace{20pt}false\\
Choice 2: \hspace{20pt}$\log_{xy}5=20$\hspace{20pt}false\\
Choice 3: \hspace{20pt}$\log_{20}5=xy$\hspace{20pt}false\\
Choice 4: \hspace{20pt}$\log_{20}xy=5$\hspace{20pt}false\\
Choice 5: \hspace{20pt}$\log_{xy}20=5$\hspace{20pt}true\\
\\[4pt]
\noindent\textbf{Question 9}\hspace{20pt}Experience: 15\hspace{20pt}Order: c2\hspace{20pt}Level: c2\hspace{20pt}Question-ID: 162\\[2pt]
Express $\log_{2}(x^2y)-\log_{2}x$ as a single logarithm\\[4pt]
\noindent\textbf{Solution 9}\\[2pt]
\\[-35pt]\begin{align*}
&\log_{2}(x^2y)-\log_{2}x\\[2pt]
=&\log_{2}((x^2y) \div x)\\[2pt]
=&\log_{2}xy
\end{align*}
Choice 1: \hspace{20pt}$2\log_{x^2y}1$\hspace{20pt}false\\
Choice 2: \hspace{20pt}$\log_{x^2y}2$\hspace{20pt}false\\
Choice 3: \hspace{20pt}$\log_{2}x^2y$\hspace{20pt}false\\
Choice 4: \hspace{20pt}$2x\log_{2}y$\hspace{20pt}false\\
Choice 5: \hspace{20pt}$\log_{2}xy$\hspace{20pt}true\\
\\[4pt]
\noindent\textbf{Question 10}\hspace{20pt}Experience: 10\hspace{20pt}Order: b1\hspace{20pt}Level: b1\hspace{20pt}Question-ID: 159\\[2pt]
Express $a^{bc}=6$ in log form\\[4pt]
\noindent\textbf{Solution 10}\\[2pt]
\\[-35pt]\begin{align*}
a^{bc}&=6\\[2pt]
\log_{a}6&=bc
\end{align*}
Choice 1: \hspace{20pt}$\log_{6}ab=c$\hspace{20pt}false\\
Choice 2: \hspace{20pt}$\log_{bc}a=6$\hspace{20pt}false\\
Choice 3: \hspace{20pt}$\log_{bc}6=a$\hspace{20pt}false\\
Choice 4: \hspace{20pt}$\log_{a}bc=6$\hspace{20pt}false\\
Choice 5: \hspace{20pt}$\log_{a}6=bc$\hspace{20pt}true\\
\\[4pt]
\noindent\textbf{Question 11}\hspace{20pt}Experience: 10\hspace{20pt}Order: b2\hspace{20pt}Level: b2\hspace{20pt}Question-ID: 155\\[2pt]
Express $(a+b)^4=15$ in log form\\[4pt]
\noindent\textbf{Solution 11}\\[2pt]
\\[-35pt]\begin{align*}
(a+b)^4&=15\\[2pt]
\log_{(a+b)}15&=4
\end{align*}
Choice 1: \hspace{20pt}$\log_{4}15=a+b$\hspace{20pt}false\\
Choice 2: \hspace{20pt}$\log_{15}(a+b)=4$\hspace{20pt}false\\
Choice 3: \hspace{20pt}$\log_{15}4=a+b$\hspace{20pt}false\\
Choice 4: \hspace{20pt}$\log_{4}(a+b)=15$\hspace{20pt}false\\
Choice 5: \hspace{20pt}$\log_{(a+b)}15=4$\hspace{20pt}true\\
\\[4pt]
\noindent\textbf{Question 12}\hspace{20pt}Experience: 10\hspace{20pt}Order: b2\hspace{20pt}Level: b2\hspace{20pt}Question-ID: 158\\[2pt]
Express $(x+4)^4=5$ in log form\\[4pt]
\noindent\textbf{Solution 12}\\[2pt]
\\[-35pt]\begin{align*}
(x+4)^4&=5\\[2pt]
\log_{(x+4)}5&=4
\end{align*}
Choice 1: \hspace{20pt}$\log_{4}(x+4)=5$\hspace{20pt}false\\
Choice 2: \hspace{20pt}$\log_{5}4=x+4$\hspace{20pt}false\\
Choice 3: \hspace{20pt}$\log_{(x+4)}4=5$\hspace{20pt}false\\
Choice 4: \hspace{20pt}$\log_{5}(x+4)=5$\hspace{20pt}false\\
Choice 5: \hspace{20pt}$\log_{(x+4)}5=4$\hspace{20pt}true\\
\\[4pt]
\noindent\textbf{Question 13}\hspace{20pt}Experience: 15\hspace{20pt}Order: c1\hspace{20pt}Level: c1\hspace{20pt}Question-ID: 161\\[2pt]
Express $\log_{4}(x+y)+\log_{4}6$ as a single logarithm\\[4pt]
\noindent\textbf{Solution 13}\\[2pt]
\\[-35pt]\begin{align*}
&\log_{4}(x+y)+\log_{4}6\\[2pt]
=&\log_{4}((x+y) \,\, \text{x} \,\, 6)\\[2pt]
=&\log_{4}6(x+y)
\end{align*}
Choice 1: \hspace{20pt}$4\log_{(x+y)}6$\hspace{20pt}false\\
Choice 2: \hspace{20pt}$\log_{(x+y)}24$\hspace{20pt}false\\
Choice 3: \hspace{20pt}$4\log_{6}x+y$\hspace{20pt}false\\
Choice 4: \hspace{20pt}$6\log_{4}x+y$\hspace{20pt}false\\
Choice 5: \hspace{20pt}$\log_{4}6(x+y)$\hspace{20pt}true\\
\\[4pt]
\noindent\textbf{Question 14}\hspace{20pt}Experience: 10\hspace{20pt}Order: a1\hspace{20pt}Level: a1\hspace{20pt}Question-ID: 148\\[2pt]
Express $\log_x9=2$ in power form\\[4pt]
\noindent\textbf{Solution 14}\\[2pt]
\\[-35pt]\begin{align*}
\log_x9&=2\\[2pt]
x^2&=9
\end{align*}
Choice 1: \hspace{20pt}$x^9=2$\hspace{20pt}false\\
Choice 2: \hspace{20pt}$x^2=2$\hspace{20pt}false\\
Choice 3: \hspace{20pt}$x^9=9$\hspace{20pt}false\\
Choice 4: \hspace{20pt}$x^2=7$\hspace{20pt}false\\
Choice 5: \hspace{20pt}$x^2=9$\hspace{20pt}true\\
\\[4pt]
\noindent\textbf{Question 15}\hspace{20pt}Experience: 15\hspace{20pt}Order: c1\hspace{20pt}Level: c1\hspace{20pt}Question-ID: 163\\[2pt]
Express $3\log_{3}(a+b)+\log_{3}4$ as a single logarithm\\[4pt]
\noindent\textbf{Solution 15}\\[2pt]
\\[-35pt]\begin{align*}
&3\log_{3}(a+b)+\log_{3}4\\[2pt]
=&\log_{3}(a+b)^3+\log_{3}4\\[2pt]
=&\log_{3}((a+b)^3 \,\, \text{x} \,\, 4)\\[2pt]
=&\log_{3}4(a+b)^3\\[-80pt]
\end{align*}
Choice 1: \hspace{20pt}$3\log_{a+b}(4)$\hspace{20pt}false\\
Choice 2: \hspace{20pt}$\log_{a+b}(12)$\hspace{20pt}false\\
Choice 3: \hspace{20pt}$12\log_{3}a+b$\hspace{20pt}false\\
Choice 4: \hspace{20pt}$4\log_{3}(a+b)^3$\hspace{20pt}false\\
Choice 5: \hspace{20pt}$\log_{3}4(a+b)^3$\hspace{20pt}true\\
\\[4pt]
\noindent\textbf{Question 16}\hspace{20pt}Experience: 15\hspace{20pt}Order: c2\hspace{20pt}Level: c2\hspace{20pt}Question-ID: 164\\[2pt]
Express $\log_{4}(a^2-b^2)-2\log_{4}(a+b)$ as a single logarithm\\[4pt]
\noindent\textbf{Solution 16}\\[2pt]
\\[-35pt]\begin{align*}
&\log_{4}(a^2-b^2)-2\log_{4}a+b\\[2pt]
=&\log_{4}(a^2-b^2)-\log_{4}(a+b)^2\\[2pt]
=&\log_{3}((a^2-b^2) \,\, \div \,\, (a+b)^2)\\[2pt]
=&\log_{3}\left(\displaystyle\frac{(a+b)(a-b)}{(a+b)^2}\right)\\[2pt]
=&\log_{3}\displaystyle\frac{a-b}{a+b}\\[-130pt]
\end{align*}
Choice 1: \hspace{20pt}$\log_{3}\displaystyle\frac{a-b}{(a+b)^2}$\hspace{20pt}false\\
Choice 2: \hspace{20pt}$\log_{3}\displaystyle\frac{a+b}{a-b}$\hspace{20pt}false\\
Choice 3: \hspace{20pt}$\log_{3}\displaystyle\frac{a^2+b^2}{a^2-b^2}$\hspace{20pt}false\\
Choice 4: \hspace{20pt}$\log_{3}\displaystyle\frac{a^2-b^2}{a^2+b^2}$\hspace{20pt}false\\
Choice 5: \hspace{20pt}$\log_{3}\displaystyle\frac{a-b}{a+b}$\hspace{20pt}true\\
\\[4pt]
\noindent\textbf{Question 17}\hspace{20pt}Experience: 15\hspace{20pt}Order: c2\hspace{20pt}Level: c2\hspace{20pt}Question-ID: 165\\[2pt]
Express $\log_{x}(4a-6b)+\log_{x}\displaystyle\frac{1}{2}$ as a single logarithm\\[4pt]
\noindent\textbf{Solution 17}\\[2pt]
\\[-35pt]\begin{align*}
&\log_{x}(4a-6b)+\log_{x}\displaystyle\frac{1}{2}\\[2pt]
=&\log_{x}\displaystyle\frac{1}{2}(4a-6b)\\[2pt]
=&\log_{x}(2a-3b)
\end{align*}
Choice 1: \hspace{20pt}$\log_{(4a-6b)}\displaystyle\frac{1}{2}x$\hspace{20pt}false\\
Choice 2: \hspace{20pt}$\log_{x}(4a-6b)$\hspace{20pt}false\\
Choice 3: \hspace{20pt}$\displaystyle\frac{1}{2}\log_{(4a-6b)}x$\hspace{20pt}false\\
Choice 4: \hspace{20pt}$\displaystyle\frac{1}{2}\log_{x}(2a-3b)$\hspace{20pt}false\\
Choice 5: \hspace{20pt}$\log_{x}(2a-3b)$\hspace{20pt}true\\
\\[4pt]
\noindent\textbf{Question 18}\hspace{20pt}Experience: 15\hspace{20pt}Order: d1\hspace{20pt}Level: d1\hspace{20pt}Question-ID: 166\\[2pt]
Express $\log_{4}(6a)-\log_{4}(2a)$ as a single logarithm\\[4pt]
\noindent\textbf{Solution 18}\\[2pt]
\\[-35pt]\begin{align*}
&\log_{4}(6a)-\log_{4}(2a)\\[2pt]
=&\log_{4}(6a \,\, \div \,\, 2a)\\[2pt]
=&\log_{4}3
\end{align*}
Choice 1: \hspace{20pt}$\log_{a}2$\hspace{20pt}false\\
Choice 2: \hspace{20pt}$\log_{a}3$\hspace{20pt}false\\
Choice 3: \hspace{20pt}$\log_{4}3a$\hspace{20pt}false\\
Choice 4: \hspace{20pt}$\log_{4}12a^2$\hspace{20pt}false\\
Choice 5: \hspace{20pt}$\log_{4}3$\hspace{20pt}true\\
\\[4pt]
\noindent\textbf{Question 19}\hspace{20pt}Experience: 15\hspace{20pt}Order: d1\hspace{20pt}Level: d1\hspace{20pt}Question-ID: 167\\[2pt]
Express $\log_{10}(15)-\log_{10}(3)$ as a single logarithm\\[4pt]
\noindent\textbf{Solution 19}\\[2pt]
\\[-35pt]\begin{align*}
&\log_{10}(15)-\log_{10}(3)\\[2pt]
=&\log_{10}(15 \,\, \div \,\, 3)\\[2pt]
=&\log_{10}5
\end{align*}
Choice 1: \hspace{20pt}$\log_{10}3$\hspace{20pt}false\\
Choice 2: \hspace{20pt}$\log_{5}45$\hspace{20pt}false\\
Choice 3: \hspace{20pt}$\log_{10}45$\hspace{20pt}false\\
Choice 4: \hspace{20pt}$\log_{5}10$\hspace{20pt}false\\
Choice 5: \hspace{20pt}$\log_{10}5$\hspace{20pt}true\\
\\[4pt]
\noindent\textbf{Question 20}\hspace{20pt}Experience: 15\hspace{20pt}Order: d2\hspace{20pt}Level: d2\hspace{20pt}Question-ID: 168\\[2pt]
Express $3\log_{y}(5)+\log_{y}(4)$ as a single logarithm\\[4pt]
\noindent\textbf{Solution 20}\\[2pt]
\\[-35pt]\begin{align*}
&3\log_{y}(5)+\log_{y}(4)\\[2pt]
=&\log_{y}5^3+\log_{y}4\\[2pt]
=&\log_{y}(5^3 \,\, \text{x} \,\, 4)\\[2pt]
=&\log_{y}500
\end{align*}
Choice 1: \hspace{20pt}$\log_{y}100$\hspace{20pt}false\\
Choice 2: \hspace{20pt}$\log_{y}8000$\hspace{20pt}false\\
Choice 3: \hspace{20pt}$4\log_{y}125$\hspace{20pt}false\\
Choice 4: \hspace{20pt}$4\log_{y}50$\hspace{20pt}false\\
Choice 5: \hspace{20pt}$\log_{y}500$\hspace{20pt}true\\
\\[4pt]
\noindent\textbf{Question 21}\hspace{20pt}Experience: 15\hspace{20pt}Order: d2\hspace{20pt}Level: d2\hspace{20pt}Question-ID: 169\\[2pt]
Express $3\log_{a}(4)-4\log_{a}(2)$ as a single logarithm\\[4pt]
\noindent\textbf{Solution 21}\\[2pt]
\\[-35pt]\begin{align*}
&\log_{a}(4^3)-\log_{a}(2^4)\\[2pt]
=&\log_{a}(64)-\log_{a}(16)\\[2pt]
=&\log_{y}(64 \div 16)\\[2pt]
=&\log_{y}4
\end{align*}
Choice 1: \hspace{20pt}$\log_{4}a^2$\hspace{20pt}false\\
Choice 2: \hspace{20pt}$\log_{4}16$\hspace{20pt}false\\
Choice 3: \hspace{20pt}$\log_{4}64$\hspace{20pt}false\\
Choice 4: \hspace{20pt}$\log_{y}16$\hspace{20pt}false\\
Choice 5: \hspace{20pt}$\log_{y}4$\hspace{20pt}true\\
\\[4pt]
\noindent\textbf{Question 22}\hspace{20pt}Experience: 10\hspace{20pt}Order: a1\hspace{20pt}Level: a1\hspace{20pt}Question-ID: 153\\[2pt]
Express $\log_{3}7=a+b^2$ in power form\\[4pt]
\noindent\textbf{Solution 22}\\[2pt]
\\[-35pt]\begin{align*}
\log_{3}7&=a+b^2\\[2pt]
3^{a+b^2}&=7
\end{align*}
Choice 1: \hspace{20pt}$7^{a+b^2}=3$\hspace{20pt}false\\
Choice 2: \hspace{20pt}$3^7=7a+b^2$\hspace{20pt}false\\
Choice 3: \hspace{20pt}$(a+b^2)^3=7$\hspace{20pt}false\\
Choice 4: \hspace{20pt}$3^{7}=a+b^2$\hspace{20pt}false\\
Choice 5: \hspace{20pt}$3^{a+b^2}=7$\hspace{20pt}true\\
\\[4pt]
\noindent\textbf{Question 23}\hspace{20pt}Experience: 30\hspace{20pt}Order: d3\hspace{20pt}Level: d3\hspace{20pt}Question-ID: 171\\[2pt]
Express $4\log_{9}5-2\log_{3}(15)$ as a single logarithm\\[4pt]
\noindent\textbf{Solution 23}\\[2pt]
\\[-35pt]\begin{align*}
&4\log_{9}5-2\log_{3}(9)\\[2pt]
=&4\left(\displaystyle\frac{\log_{3}5}{\log_{3}9}\right)-2\log_{3}(15)\\[2pt]
=&\left(\displaystyle\frac{4\log_{3}5}{2}\right)-\log_{3}(15^2)\\[2pt]
=&2\log_{3}5-\log_{3}(15^2)\\[2pt]
=&\log_{3}(5^2 \div 15^2)\\[2pt]
=&\log_{3}\displaystyle\frac{25}{225}\\[2pt]
=&\log_{3}\displaystyle\frac{1}{9}
\end{align*}
Choice 1: \hspace{20pt}$\log_{9}225$\hspace{20pt}false\\
Choice 2: \hspace{20pt}$\log_{9}25$\hspace{20pt}false\\
Choice 3: \hspace{20pt}$\log_{9}\displaystyle\frac{1}{3}$\hspace{20pt}false\\
Choice 4: \hspace{20pt}$\log_{3}9$\hspace{20pt}false\\
Choice 5: \hspace{20pt}$\log_{3}\displaystyle\frac{1}{9}$\hspace{20pt}true\\
\\[4pt]
\noindent\textbf{Question 24}\hspace{20pt}Experience: 15\hspace{20pt}Order: c1\hspace{20pt}Level: c1\hspace{20pt}Question-ID: 160\\[2pt]
Express $\log_{a}4+\log_{a}5$ as a single logarithm\\[4pt]
\noindent\textbf{Solution 24}\\[2pt]
\\[-35pt]\begin{align*}
&\log_{a}4+\log_{a}5\\[2pt]
=&\log_{a}(4 \,\, \text{x} \,\, 5)\\[2pt]
=&\log_{a}20
\end{align*}
Choice 1: \hspace{20pt}$\log_{4}5a$\hspace{20pt}false\\
Choice 2: \hspace{20pt}$a\log_{4}5$\hspace{20pt}false\\
Choice 3: \hspace{20pt}$4\log_{a}5$\hspace{20pt}false\\
Choice 4: \hspace{20pt}$5\log_{a}4$\hspace{20pt}false\\
Choice 5: \hspace{20pt}$\log_{a}20$\hspace{20pt}true\\
\\[4pt]
\noindent\textbf{Question 25}\hspace{20pt}Experience: 20\hspace{20pt}Order: e2\hspace{20pt}Level: e2\hspace{20pt}Question-ID: 188\\[2pt]
Express $\log_{a}\displaystyle\frac{a^2b^5}{c^3}$ as a linear combination\\[4pt]
\noindent\textbf{Solution 25}\\[2pt]
\\[-35pt]\begin{align*}
&\log_{a}\displaystyle\frac{a^2b^5}{c^3}\\[2pt]
=&\log_{a}a^2+\log_{a}b^5-\log_{a}c^3\\[2pt]
=&2+5\log_{a}b-3\log_{a}c\\[-50pt]
\end{align*}
Choice 1: \hspace{20pt}$3+5\log_{a}b-2\log_{a}c$\hspace{20pt}false\\
Choice 2: \hspace{20pt}$5+2\log_{a}b-3\log_{a}c$\hspace{20pt}false\\
Choice 3: \hspace{20pt}$5+3\log_{a}b-2\log_{a}c$\hspace{20pt}false\\
Choice 4: \hspace{20pt}$2+3\log_{a}b-5\log_{a}c$\hspace{20pt}false\\
Choice 5: \hspace{20pt}$2+5\log_{a}b-3\log_{a}c$\hspace{20pt}true\\
\\[4pt]
\noindent\textbf{Question 26}\hspace{20pt}Experience: 20\hspace{20pt}Order: e2\hspace{20pt}Level: e2\hspace{20pt}Question-ID: 189\\[2pt]
Express $\log_{c}\displaystyle\frac{a^4}{b^2c^6}$ as a linear combination\\[4pt]
\noindent\textbf{Solution 26}\\[2pt]
\\[-35pt]\begin{align*}
&\log_{c}\displaystyle\frac{a^4}{b^2c^6}\\[2pt]
=&\log_{c}a^4-\log_{c}b^2-\log_{c}c^6\\[2pt]
=&4\log_{c}a-2\log_{c}b-6\\[-30pt]
\end{align*}
Choice 1: \hspace{20pt}$6\log_{c}a-4\log_{c}b-2$\hspace{20pt}false\\
Choice 2: \hspace{20pt}$2\log_{c}a-6\log_{c}b-4$\hspace{20pt}false\\
Choice 3: \hspace{20pt}$2\log_{c}a-4\log_{c}b-6$\hspace{20pt}false\\
Choice 4: \hspace{20pt}$4\log_{c}a-6\log_{c}b-2$\hspace{20pt}false\\
Choice 5: \hspace{20pt}$4\log_{c}a-2\log_{c}b-6$\hspace{20pt}true\\
\\[4pt]
\noindent\textbf{Question 27}\hspace{20pt}Experience: 10\hspace{20pt}Order: a1\hspace{20pt}Level: a1\hspace{20pt}Question-ID: 180\\[2pt]
Express $\log_{a}b-4=7$ in power form\\[4pt]
\noindent\textbf{Solution 27}\\[2pt]
\\[-35pt]\begin{align*}
\log_{a}b-4&=7\\[2pt]
a^7&=b-4
\end{align*}
Choice 1: \hspace{20pt}$a^{b-4}=7$\hspace{20pt}false\\
Choice 2: \hspace{20pt}$(b-4)^7=a$\hspace{20pt}false\\
Choice 3: \hspace{20pt}$7^a=b-4$\hspace{20pt}false\\
Choice 4: \hspace{20pt}$(b-4)^a=7$\hspace{20pt}false\\
Choice 5: \hspace{20pt}$a^7=b-4$\hspace{20pt}true\\
\\[4pt]
\noindent\textbf{Question 28}\hspace{20pt}Experience: 25\hspace{20pt}Order: g1\hspace{20pt}Level: g1\hspace{20pt}Question-ID: 198\\[2pt]
Solve $3^x=27$ for $x$\\[4pt]
\noindent\textbf{Solution 28}\\[2pt]
\\[-35pt]\begin{align*}
3^x&=27\\[2pt]
x&=\log_{3}27\\[2pt]
x&=\displaystyle\frac{\log27}{\log3}\\[2pt]
x&=3
\end{align*}
Answer part 1: \hspace{10pt}Label\hspace{10pt}$x=$\hspace{10pt}Solution\hspace{10pt}3\\
Answer part 1 hint: \hspace{15pt}$x$ is an integer value\\
\\[4pt]
\noindent\textbf{Question 29}\hspace{20pt}Experience: 30\hspace{20pt}Order: d3\hspace{20pt}Level: d3\hspace{20pt}Question-ID: 170\\[2pt]
Express $2\log_{16}8-4\log_{4}(2)$ as a single logarithm\\[4pt]
\noindent\textbf{Solution 29}\\[2pt]
\\[-35pt]\begin{align*}
&2\log_{16}8-4\log_{4}(2)\\[2pt]
=&2\left(\displaystyle\frac{\log_{4}8}{\log_416}\right)-4\log_{4}(2)\\[2pt]
=&\left(\displaystyle\frac{2\log_{4}8}{2}\right)-\log_{4}(16)\\[2pt]
=&\log_{4}(8 \div 16)\\[2pt]
=&\log_{4}\displaystyle\frac{1}{2}
\end{align*}
Choice 1: \hspace{20pt}$\log_{8}4$\hspace{20pt}false\\
Choice 2: \hspace{20pt}$\log_{8}2$\hspace{20pt}false\\
Choice 3: \hspace{20pt}$\log_{8}\displaystyle\frac{1}{2}$\hspace{20pt}false\\
Choice 4: \hspace{20pt}$\log_{4}2$\hspace{20pt}false\\
Choice 5: \hspace{20pt}$\log_{4}\displaystyle\frac{1}{2}$\hspace{20pt}true\\
\\[4pt]
\noindent\textbf{Question 30}\hspace{20pt}Experience: 30\hspace{20pt}Order: d4\hspace{20pt}Level: d4\hspace{20pt}Question-ID: 181\\[2pt]
Express $3\log_{4}5+4\log_{16}(3)$ as a single logarithm.\\[4pt]
\noindent\textbf{Solution 30}\\[2pt]
\\[-35pt]\begin{align*}
&3\log_{4}5+4\log_{16}(3)\\[2pt]
=&3\log_{4}5+4\left(\displaystyle\frac{\log_{4}3}{\log_{4}16}\right)\\[2pt]
=&\log_{4}5^3+4\left(\displaystyle\frac{\log_{4}3}{2}\right)\\[2pt]
=&\log_{4}5^3+\log_{4}3^2\\[2pt]
=&\log_{4}1125\\[-120pt]
\end{align*}
Choice 1: \hspace{20pt}$\log_{4}225$\hspace{20pt}false\\
Choice 2: \hspace{20pt}$\log_{4}25$\hspace{20pt}false\\
Choice 3: \hspace{20pt}$\log_{16}25$\hspace{20pt}false\\
Choice 4: \hspace{20pt}$\log_{16}225$\hspace{20pt}false\\
Choice 5: \hspace{20pt}$\log_{4}1125$\hspace{20pt}true\\
\\[4pt]
\noindent\textbf{Question 31}\hspace{20pt}Experience: 30\hspace{20pt}Order: d4\hspace{20pt}Level: d4\hspace{20pt}Question-ID: 184\\[2pt]
Express $8\log_{4}3-2\log_{2}(4)$ as a single logarithm\\[4pt]
\noindent\textbf{Solution 31}\\[2pt]
\\[-35pt]\begin{align*}
&8\log_{4}3-2\log_{2}4\\[2pt]
=&8\left(\displaystyle\frac{\log_{2}3}{\log_{2}4}\right)-2\log_{2}4\\[2pt]
=&4\log_{2}3-2\log_{2}4\\[2pt]
=&\log_{2}3^4-\log_{2}4^2\\[2pt]
=&\log_{2}\displaystyle\frac{3^4}{4^2}\\[2pt]
=&\log_{2}\displaystyle\frac{81}{16}\\[-87pt]
\end{align*}
Choice 1: \hspace{20pt}$\log_{4}65$\hspace{20pt}false\\
Choice 2: \hspace{20pt}$\log_{4}1296$\hspace{20pt}false\\
Choice 3: \hspace{20pt}$\log_{2}\displaystyle\frac{27}{4}$\hspace{20pt}false\\
Choice 4: \hspace{20pt}$\log_{2}65$\hspace{20pt}false\\
Choice 5: \hspace{20pt}$\log_{2}\displaystyle\frac{81}{16}$\hspace{20pt}true\\
\\[4pt]
\noindent\textbf{Question 32}\hspace{20pt}Experience: 20\hspace{20pt}Order: e1\hspace{20pt}Level: e1\hspace{20pt}Question-ID: 185\\[2pt]
Express $\log_{x}a^2b^3c^4$ as a linear combination\\[4pt]
\noindent\textbf{Solution 32}\\[2pt]
\\[-35pt]\begin{align*}
&\log_{x}a^2b^3c^4\\[2pt]
=&\log_{x}a^2+\log_{x}b^3+\log_{x}c^4\\[2pt]
=&2\log_{x}a+3\log_{x}b+4\log_{x}c\\[-30pt]
\end{align*}
Choice 1: \hspace{20pt}$4\log_{x}a+2\log_{x}b+3\log_{x}c$\hspace{20pt}false\\
Choice 2: \hspace{20pt}$4\log_{x}a+3\log_{x}b+2\log_{x}c$\hspace{20pt}false\\
Choice 3: \hspace{20pt}$2\log_{x}a+4\log_{x}b+3\log_{x}c$\hspace{20pt}false\\
Choice 4: \hspace{20pt}$3\log_{x}a+2\log_{x}b+4\log_{x}c$\hspace{20pt}false\\
Choice 5: \hspace{20pt}$2\log_{x}a+3\log_{x}b+4\log_{x}c$\hspace{20pt}true\\
\\[4pt]
\noindent\textbf{Question 33}\hspace{20pt}Experience: 20\hspace{20pt}Order: e1\hspace{20pt}Level: e1\hspace{20pt}Question-ID: 186\\[2pt]
Express $\log_{a}x^5y^4z^6$ as a linear combination\\[4pt]
\noindent\textbf{Solution 33}\\[2pt]
\\[-35pt]\begin{align*}
&\log_{a}x^5y^4z^6\\[2pt]
=&\log_{a}x^5+\log_{a}y^4+\log_{a}z^6\\[2pt]
=&5\log_{a}x+4\log_{a}y+6\log_{a}z\\[-30pt]
\end{align*}
Choice 1: \hspace{20pt}$5\log_{a}x+6\log_{a}y+4\log_{a}z$\hspace{20pt}false\\
Choice 2: \hspace{20pt}$6\log_{a}x+4\log_{a}y+5\log_{a}z$\hspace{20pt}false\\
Choice 3: \hspace{20pt}$4\log_{a}x+5\log_{a}y+6\log_{a}z$\hspace{20pt}false\\
Choice 4: \hspace{20pt}$6\log_{a}x+5\log_{a}y+4\log_{a}z$\hspace{20pt}false\\
Choice 5: \hspace{20pt}$5\log_{a}x+4\log_{a}y+6\log_{a}z$\hspace{20pt}true\\
\\[4pt]
\noindent\textbf{Question 34}\hspace{20pt}Experience: 20\hspace{20pt}Order: e1\hspace{20pt}Level: e1\hspace{20pt}Question-ID: 187\\[2pt]
Express $\log_{b}a^7b^2c^5$ as a linear combination\\[4pt]
\noindent\textbf{Solution 34}\\[2pt]
\\[-35pt]\begin{align*}
&\log_{b}a^7b^2c^5\\[2pt]
=&\log_{b}a^7+\log_{b}b^2+\log_{b}c^5\\[2pt]
=&7\log_{b}a+2+5\log_{b}c\\[-30pt]
\end{align*}
Choice 1: \hspace{20pt}$2\log_{b}a+5+7\log_{b}c$\hspace{20pt}false\\
Choice 2: \hspace{20pt}$2\log_{b}a+7+5\log_{b}c$\hspace{20pt}false\\
Choice 3: \hspace{20pt}$7\log_{b}a+5+2\log_{b}c$\hspace{20pt}false\\
Choice 4: \hspace{20pt}$5\log_{b}a+2+7\log_{b}c$\hspace{20pt}false\\
Choice 5: \hspace{20pt}$7\log_{b}a+2+5\log_{b}c$\hspace{20pt}true\\
\\[4pt]
\noindent\textbf{Question 35}\hspace{20pt}Experience: 20\hspace{20pt}Order: e2\hspace{20pt}Level: e2\hspace{20pt}Question-ID: 190\\[2pt]
Express $\log_{x}\displaystyle\frac{x^3y^8}{z^2}$ as a linear combination\\[4pt]
\noindent\textbf{Solution 35}\\[2pt]
\\[-35pt]\begin{align*}
&\log_{x}\displaystyle\frac{x^3y^8}{z^2}\\[2pt]
=&\log_{x}x^3+\log_{x}y^8-\log_{x}z^2\\[2pt]
=&3+8\log_{x}y-2\log_{x}z\\[-30pt]
\end{align*}
Choice 1: \hspace{20pt}$8+2\log_{x}y-3\log_{x}z$\hspace{20pt}false\\
Choice 2: \hspace{20pt}$2+8\log_{x}y-3\log_{x}z$\hspace{20pt}false\\
Choice 3: \hspace{20pt}$2+3\log_{x}y-8\log_{x}z$\hspace{20pt}false\\
Choice 4: \hspace{20pt}$3+2\log_{x}y-8\log_{x}z$\hspace{20pt}false\\
Choice 5: \hspace{20pt}$3+8\log_{x}y-2\log_{x}z$\hspace{20pt}true\\
\\[4pt]
\noindent\textbf{Question 36}\hspace{20pt}Experience: 20\hspace{20pt}Order: f1\hspace{20pt}Level: f1\hspace{20pt}Question-ID: 191\\[2pt]
Express $\log_{4}20$ in terms of $\ln$\\[4pt]
\noindent\textbf{Solution 36}\\[2pt]
\\[-35pt]\begin{align*}
&\log_{4}20\\[2pt]
=&\displaystyle\frac{\log_{e}20}{\log_{e}4}\\[2pt]
=&\displaystyle\frac{\ln20}{\ln4}\\[-100pt]
\end{align*}
Choice 1: \hspace{20pt}$\displaystyle\frac{\ln20}{\ln80}$\hspace{20pt}false\\
Choice 2: \hspace{20pt}$\ln{24}$\hspace{20pt}false\\
Choice 3: \hspace{20pt}$\ln{80}$\hspace{20pt}false\\
Choice 4: \hspace{20pt}$\ln{5}$\hspace{20pt}false\\
Choice 5: \hspace{20pt}$\displaystyle\frac{\ln20}{\ln4}$\hspace{20pt}false\\
\\[4pt]
\noindent\textbf{Question 37}\hspace{20pt}Experience: 20\hspace{20pt}Order: f1\hspace{20pt}Level: f1\hspace{20pt}Question-ID: 192\\[2pt]
Express $\log_{8}24$ in terms of $\ln$\\[4pt]
\noindent\textbf{Solution 37}\\[2pt]
\\[-35pt]\begin{align*}
&\log_{8}24\\[2pt]
=&\displaystyle\frac{\log_{e}24}{\log_{e}8}\\[2pt]
=&\displaystyle\frac{\ln24}{\ln8}\\[-100pt]
\end{align*}
Choice 1: \hspace{20pt}$\displaystyle\frac{\ln24}{\ln192}$\hspace{20pt}false\\
Choice 2: \hspace{20pt}$\ln{192}$\hspace{20pt}false\\
Choice 3: \hspace{20pt}$\ln{32}$\hspace{20pt}false\\
Choice 4: \hspace{20pt}$\ln{3}$\hspace{20pt}false\\
Choice 5: \hspace{20pt}$\displaystyle\frac{\ln24}{\ln8}$\hspace{20pt}false\\
\\[4pt]
\noindent\textbf{Question 38}\hspace{20pt}Experience: 20\hspace{20pt}Order: f1\hspace{20pt}Level: f1\hspace{20pt}Question-ID: 193\\[2pt]
Express $\log_{6}18$ in terms of $\ln$\\[4pt]
\noindent\textbf{Solution 38}\\[2pt]
\\[-35pt]\begin{align*}
&\log_{6}18\\[2pt]
=&\displaystyle\frac{\log_{e}18}{\log_{e}6}\\[2pt]
=&\displaystyle\frac{\ln18}{\ln6}\\[-100pt]
\end{align*}
Choice 1: \hspace{20pt}$\displaystyle\frac{\ln6}{\ln2}$\hspace{20pt}false\\
Choice 2: \hspace{20pt}$\ln{108}$\hspace{20pt}false\\
Choice 3: \hspace{20pt}$\ln{24}$\hspace{20pt}false\\
Choice 4: \hspace{20pt}$\ln{3}$\hspace{20pt}false\\
Choice 5: \hspace{20pt}$\displaystyle\frac{\ln18}{\ln6}$\hspace{20pt}true\\
\\[4pt]
\noindent\textbf{Question 39}\hspace{20pt}Experience: 20\hspace{20pt}Order: f2\hspace{20pt}Level: f2\hspace{20pt}Question-ID: 194\\[2pt]
Express $\log_{2}8$ in terms of $\ln$\\[4pt]
\noindent\textbf{Solution 39}\\[2pt]
\\[-35pt]\begin{align*}
&\log_{2}8\\[2pt]
=&\displaystyle\frac{\log_{e}8}{\log_{e}2}\\[2pt]
=&\displaystyle\frac{\ln8}{\ln2}\\[-100pt]
\end{align*}
Choice 1: \hspace{20pt}$\displaystyle\frac{\ln8}{\ln16}$\hspace{20pt}false\\
Choice 2: \hspace{20pt}$\ln{10}$\hspace{20pt}false\\
Choice 3: \hspace{20pt}$\ln{16}$\hspace{20pt}false\\
Choice 4: \hspace{20pt}$\ln{4}$\hspace{20pt}false\\
Choice 5: \hspace{20pt}$\displaystyle\frac{\ln8}{\ln2}$\hspace{20pt}true\\
\\[4pt]
\noindent\textbf{Question 40}\hspace{20pt}Experience: 20\hspace{20pt}Order: f2\hspace{20pt}Level: f2\hspace{20pt}Question-ID: 195\\[2pt]
Express $\log_{6}15$ in terms of $\ln$\\[4pt]
\noindent\textbf{Solution 40}\\[2pt]
\\[-35pt]\begin{align*}
&\log_{6}15\\[2pt]
=&\displaystyle\frac{\log_{e}15}{\log_{e}6}\\[2pt]
=&\displaystyle\frac{\ln15}{\ln6}\\[-100pt]
\end{align*}
Choice 1: \hspace{20pt}$\displaystyle\frac{\ln5}{\ln2}$\hspace{20pt}false\\
Choice 2: \hspace{20pt}$\ln{9}$\hspace{20pt}false\\
Choice 3: \hspace{20pt}$\ln{21}$\hspace{20pt}false\\
Choice 4: \hspace{20pt}$\ln{\displaystyle\frac{5}{2}}$\hspace{20pt}false\\
Choice 5: \hspace{20pt}$\displaystyle\frac{\ln15}{\ln6}$\hspace{20pt}true\\
\\[4pt]
\noindent\textbf{Question 41}\hspace{20pt}Experience: 25\hspace{20pt}Order: g1\hspace{20pt}Level: g1\hspace{20pt}Question-ID: 199\\[2pt]
Solve $4^x=64$ for $x$\\[4pt]
\noindent\textbf{Solution 41}\\[2pt]
\\[-35pt]\begin{align*}
4^x&=64\\[2pt]
x&=\log_{4}64\\[2pt]
x&=\displaystyle\frac{\log64}{\log4}\\[2pt]
x&=3
\end{align*}
Answer part 1: \hspace{10pt}Label\hspace{10pt}$x=$\hspace{10pt}Solution\hspace{10pt}3\\
Answer part 1 hint: \hspace{15pt}$x$ is an integer value\\
\\[4pt]
\noindent\textbf{Question 42}\hspace{20pt}Experience: 25\hspace{20pt}Order: h1\hspace{20pt}Level: h1\hspace{20pt}Question-ID: 200\\[2pt]
Solve $5^x=35$ for $x$\\[4pt]
\noindent\textbf{Solution 42}\\[2pt]
\\[-35pt]\begin{align*}
5^x&=35\\[2pt]
x&=\log_{5}35\\[2pt]
x&=\displaystyle\frac{\log35}{\log5}\\[2pt]
x&=2.21
\end{align*}
Answer part 1: \hspace{10pt}Label\hspace{10pt}$x=$\hspace{10pt}Solution\hspace{10pt}2.21\\
Answer part 1 hint: \hspace{15pt}Write the answer to 2 d.p.\\
\\[4pt]
\noindent\textbf{Question 43}\hspace{20pt}Experience: 25\hspace{20pt}Order: h1\hspace{20pt}Level: h1\hspace{20pt}Question-ID: 201\\[2pt]
Solve $4^x=20$ for $x$\\[4pt]
\noindent\textbf{Solution 43}\\[2pt]
\\[-35pt]\begin{align*}
4^x&=20\\[2pt]
x&=\log_{4}20\\[2pt]
x&=\displaystyle\frac{\log20}{\log4}\\[2pt]
x&=2.16
\end{align*}
Answer part 1: \hspace{10pt}Label\hspace{10pt}$x=$\hspace{10pt}Solution\hspace{10pt}2.16\\
Answer part 1 hint: \hspace{15pt}Write the answer to 2 d.p.\\
\\[4pt]
\noindent\textbf{Question 44}\hspace{20pt}Experience: 25\hspace{20pt}Order: h1\hspace{20pt}Level: h1\hspace{20pt}Question-ID: 202\\[2pt]
Solve $5^x=75$ for $x$\\[4pt]
\noindent\textbf{Solution 44}\\[2pt]
\\[-35pt]\begin{align*}
5^x&=75\\[2pt]
x&=\log_{5}75\\[2pt]
x&=\displaystyle\frac{\log75}{\log5}\\[2pt]
x&=2.68
\end{align*}
Answer part 1: \hspace{10pt}Label\hspace{10pt}$x=$\hspace{10pt}Solution\hspace{10pt}2.68\\
Answer part 1 hint: \hspace{15pt}Write the answer to 2 d.p.\\
\\[4pt]
\noindent\textbf{Question 45}\hspace{20pt}Experience: 40\hspace{20pt}Order: i1\hspace{20pt}Level: i1\hspace{20pt}Question-ID: 203\\[2pt]
Solve $3^{5x+3}=5^{3x-1}$ for $x$\\[4pt]
\noindent\textbf{Solution 45}\\[2pt]
\\[-35pt]\begin{align*}
3^{5x+3}&=5^{3x-1}\\[2pt]
\log(3^{5x+3})&\log(5^{3x-1})\\[2pt]
(5x+3)\log3&=(3x-1)\log5\\[2pt]
5x\log3+3\log3&=3x\log5-\log5\\[2pt]
5x\log3-3x\log5&=-\log5-3\log3\\[2pt]
x(5\log3-3\log5)&=-(\log5+3\log3)\\[2pt]
x\log\displaystyle\frac{3^5}{5^3}&=-\log(5 \,\, \text{x} \,\, 3^3)\\[2pt]
0.289x&=-2.13\\[2pt]
x&=-7.38
\end{align*}
Answer part 1: \hspace{10pt}Label\hspace{10pt}$x=$\hspace{10pt}Solution\hspace{10pt}-7.38\\
Answer part 1 hint: \hspace{15pt}Write the answer to 2 d.p.\\
\\[4pt]
\noindent\textbf{Question 46}\hspace{20pt}Experience: 40\hspace{20pt}Order: i1\hspace{20pt}Level: i1\hspace{20pt}Question-ID: 204\\[2pt]
Solve $5^{2x-5}=3^{6x+7}$ for $x$\\[4pt]
\noindent\textbf{Solution 46}\\[2pt]
\\[-35pt]\begin{align*}
5^{2x-5}&=3^{6x+7}\\[2pt]
\log(5^{2x-5})&=\log(3^{6x+7})\\[2pt]
(2x-5)\log5&=(6x+7)\log3\\[2pt]
2x\log5-5\log5&=6x\log3+7\log3\\[2pt]
2x\log5-6x\log3&=5\log5+7\log3\\[2pt]
x(\log5^2-\log3^6)&=\log5^5+\log3^7\\[2pt]
x\left(\log\displaystyle\frac{5^2}{3^6}\right)&=\log(5^5\,\,\text{x}\,\,3^7)\\[2pt]
-1.46x&=6.83\\[2pt]
x&=-4.67
\end{align*}
Answer part 1: \hspace{10pt}Label\hspace{10pt}$x=$\hspace{10pt}Solution\hspace{10pt}-4.67\\
Answer part 1 hint: \hspace{15pt}Write the answer to 2 d.p.\\
\\[4pt]
\noindent\textbf{Question 47}\hspace{20pt}Experience: 40\hspace{20pt}Order: i2\hspace{20pt}Level: i2\hspace{20pt}Question-ID: 205\\[2pt]
Solve $2^{4x-3}=5^{7x+4}$ for $x$\\[4pt]
\noindent\textbf{Solution 47}\\[2pt]
\\[-35pt]\begin{align*}
2^{4x-3}&=5^{7x+4}\\[2pt]
\log(2^{4x-3})&=\log(5^{7x+4})\\[2pt]
(4x-3)\log2&=(7x+4)\log5\\[2pt]
4x\log2-3\log2&=7x\log5+4\log5\\[2pt]
4x\log2-7x\log5&=3\log2+4\log5\\[2pt]
x(\log2^4-\log5^7)&=\log2^3+\log5^4\\[2pt]
x\left(\log\displaystyle\frac{2^4}{5^7}\right)&=\log(2^3\,\,\text{x}\,\,5^4)\\[2pt]
-3.69x&=3.70\\[2pt]
x&=-1.00
\end{align*}
Answer part 1: \hspace{10pt}Label\hspace{10pt}$x=$\hspace{10pt}Solution\hspace{10pt}-1.00\\
Answer part 1 hint: \hspace{15pt}Write the answer to 2 d.p.\\
\\[4pt]
\noindent\textbf{Question 48}\hspace{20pt}Experience: 40\hspace{20pt}Order: i2\hspace{20pt}Level: i2\hspace{20pt}Question-ID: 206\\[2pt]
Solve $7^{6x+2}=4^{4x-5}$ for $x$\\[4pt]
\noindent\textbf{Solution 48}\\[2pt]
\\[-35pt]\begin{align*}
7^{6x+2}&=4^{4x-5}\\[2pt]
\log(7^{6x+2})&=\log(4^{4x-5})\\[2pt]
(6x+2)\log7&=(4x-5)\log4\\[2pt]
6x\log7+2\log7&=4x\log4-5\log4\\[2pt]
6x\log7-4x\log4&=-2\log7-5\log4\\[2pt]
x(6\log7-4\log4)&=-(2\log7+5\log4)\\[2pt]
x(\log7^6-\log4^4)&=-(\log7^2+\log4^5)\\[2pt]
x\left(\log\displaystyle\frac{7^6}{4^4}\right)&=-\log(7^2 \,\, \text{x} \,\, 4^5)\\[2pt]
2.66x&=-4.70\\[2pt]
x&=-1.77
\end{align*}
Answer part 1: \hspace{10pt}Label\hspace{10pt}$x=$\hspace{10pt}Solution\hspace{10pt}-1.77\\
Answer part 1 hint: \hspace{15pt}Write the answer to 2 d.p.\\
\\[4pt]
\noindent\textbf{Question 49}\hspace{20pt}Experience: 40\hspace{20pt}Order: i1\hspace{20pt}Level: i1\hspace{20pt}Question-ID: 234\\[2pt]
Solve $5^{2-3x}=4^{3+2x}$ for $x$\\[4pt]
\noindent\textbf{Solution 49}\\[2pt]
\\[-35pt]\begin{align*}
5^{2-3x}&=4^{3+2x}\\[2pt]
\log(5^{2-3x})&=\log(4^{3+2x})\\[2pt]
(2-3x)\log5&=(3+2x)\log4\\[2pt]
2\log5-3x\log5&=3\log4+2x\log4\\[2pt]
-3x\log5-2x\log4&=3\log4-2\log5\\[2pt]
3x\log5+2x\log4&=-3\log4+2\log5\\[2pt]
x(\log5^3+\log4^2)&=2\log5-3\log4\\[2pt]
x\log{(5^3\,\,\text{x}\,\,4^2)}&=\log\left(\displaystyle\frac{5^2}{4^3}\right)\\[2pt]
3.30x&=-0.408\\[2pt]
x&=-0.12
\end{align*}
Answer part 1: \hspace{10pt}Label\hspace{10pt}$x=$\hspace{10pt}Solution\hspace{10pt}-0.12\\
Answer part 1 hint: \hspace{15pt}Write the answer to 2 d.p.\\
\\[4pt]
\noindent\textbf{Question 50}\hspace{20pt}Experience: 40\hspace{20pt}Order: i2\hspace{20pt}Level: i2\hspace{20pt}Question-ID: 235\\[2pt]
Solve $3^{2x+1}=7^{2x-5}$ for $x$\\[4pt]
\noindent\textbf{Solution 50}\\[2pt]
\\[-35pt]\begin{align*}
3^{2x+1}&=7^{2x-5}\\[2pt]
\log(3^{2x+1})&=\log(7^{2x-5})\\[2pt]
(2x+1)\log3&=(2x-5)\log7\\[2pt]
2x\log3+\log3&=2x\log7-5\log7\\[2pt]
2x\log3-2x\log7&=-5\log7-\log3\\[2pt]
x(2\log3-2\log7)&=-(5\log7+\log3)\\[2pt]
x(\log3^2-\log7^2)&=-(\log7^5+\log3)\\[2pt]
x\left(\log\displaystyle\frac{3^2}{7^2}\right)&=-\log(7^5 \,\, \text{x} \,\, 3)\\[2pt]
-0.736x&=-4.70\\[2pt]
x&=6.39
\end{align*}
Answer part 1: \hspace{10pt}Label\hspace{10pt}$x=$\hspace{10pt}Solution\hspace{10pt}6.39\\
Answer part 1 hint: \hspace{15pt}Write the answer to 2 d.p.\\
\\[4pt]
\noindent\large{\textbf{Lesson 2 Applications of logarithms}}\\[12pt]
\noindent\textbf{Question 1}\hspace{20pt}Experience: 50\hspace{20pt}Order: j1\hspace{20pt}Level: j1\hspace{20pt}Question-ID: 207\\[2pt]
Solve $(5^x)^2-8(5^x)+15=0$ for $x$\\[4pt]
\noindent\textbf{Solution 1}\\[2pt]
\\[-35pt]\begin{align*}
(5^x)^2-8(5^x)+15&=0\qquad y=5^x\\[2pt]
y^2-8y+15&=0\qquad P=15 \quad S=-8\\[2pt]
(y-5)(y-3)&=0\qquad (-5,-3)\\[2pt]
y=5 \qquad y&=3\\[2pt]
5^x=5\hspace{16pt} 5^x&=3\\[2pt]
x=1 \qquad x&=0.68
\end{align*}
Answer part 1: \hspace{10pt}Label\hspace{10pt}$x=$\hspace{10pt}Solution\hspace{10pt}0.68,1\\
Answer part 1 hint: \hspace{15pt} Give all solutions correct to 2 d.p, exact integer solutions are also allowed, separate solutions with comma(s). (e.g: 2,-3 or 1.5,0.6)\\
\\[4pt]
\noindent\textbf{Question 2}\hspace{20pt}Experience: 50\hspace{20pt}Order: j1\hspace{20pt}Level: j1\hspace{20pt}Question-ID: 208\\[2pt]
Solve $(2^x)^2-9(2^x)+14=0$ for $x$\\[4pt]
\noindent\textbf{Solution 2}\\[2pt]
\\[-35pt]\begin{align*}
(2^x)^2-9(2^x)+14&=0\qquad y=2^x\\[2pt]
y^2-9y+14&=0\qquad P=14 \quad S=-9\\[2pt]
(y-7)(y-2)&=0\qquad (-7,-2)\\[2pt]
y=7 \qquad y&=2\\[2pt]
2^x=7\hspace{16pt} 2^x&=2\\[2pt]
x=2.81 \hspace{7pt} x&=1
\end{align*}
Answer part 1: \hspace{10pt}Label\hspace{10pt}$x=$\hspace{10pt}Solution\hspace{10pt}1,2.81\\
Answer part 1 hint: \hspace{15pt} Give all solutions correct to 2 d.p, exact integer solutions are also allowed, separate solutions with comma(s). (e.g: 2,-3 or 1.5,0.6)\\
\\[4pt]
\noindent\textbf{Question 3}\hspace{20pt}Experience: 50\hspace{20pt}Order: j2\hspace{20pt}Level: j2\hspace{20pt}Question-ID: 209\\[2pt]
Solve $2^{2x+1}-10(2^x)+12=0$ for $x$\\[4pt]
\noindent\textbf{Solution 3}\\[2pt]
\\[-35pt]\begin{align*}
2^{2x+1}-10(2^x)+12&=0\\[2pt]
2(2^x)^2-10(2^x)+12&=0\qquad y=2^x\\[2pt]
2y^2-10y+12&=0\qquad P=24 \quad S=-10\\[2pt]
(y-2)(y-3)&=0\qquad (-4,-6)\quad (-2,-3)\\[2pt]
y=2 \qquad y&=3\\[2pt]
2^x=2\hspace{16pt} 2^x&=3\\[2pt]
x=1 \hspace{20pt} x&=1.58
\end{align*}
Answer part 1: \hspace{10pt}Label\hspace{10pt}$x=$\hspace{10pt}Solution\hspace{10pt}1,1.58\\
Answer part 1 hint: \hspace{15pt} Give all solutions correct to 2 d.p, exact integer solutions are also allowed, separate solutions with comma(s). (e.g: 2,-3 or 1.5,0.6)\\
\\[4pt]
\noindent\textbf{Question 4}\hspace{20pt}Experience: 50\hspace{20pt}Order: j2\hspace{20pt}Level: j2\hspace{20pt}Question-ID: 210\\[2pt]
Solve $3^{2x+1}-17(3^x)+20=0$ for $x$\\[4pt]
\noindent\textbf{Solution 4}\\[2pt]
\\[-35pt]\begin{align*}
3^{2x+1}-17(3^x)+20&=0\\[2pt]
3(3^x)^2-17(3^x)+20&=0\qquad y=3^x\\[2pt]
3y^2-17y+20&=0\qquad P=60 \quad S=-17\\[2pt]
(y-4)\left(y-\displaystyle\frac{5}{3}\right)&=0\qquad (-12,-5)\quad \left(-4,-\displaystyle\frac{5}{3}\right)\\[2pt]
y=4 \qquad y&=\displaystyle\frac{5}{3}\\[2pt]
3^x=4\hspace{16pt} 3^x&=\displaystyle\frac{5}{3}\\[2pt]
x=1.26 \hspace{6pt} x&=0.46
\end{align*}
Answer part 1: \hspace{10pt}Label\hspace{10pt}$x=$\hspace{10pt}Solution\hspace{10pt}0.46,1.26\\
Answer part 1 hint: \hspace{15pt} Give all solutions correct to 2 d.p, exact integer solutions are also allowed, separate solutions with comma(s). (e.g: 2,-3 or 1.5,0.6)\\
\\[4pt]
\noindent\textbf{Question 5}\hspace{20pt}Experience: 50\hspace{20pt}Order: k1\hspace{20pt}Level: k1\hspace{20pt}Question-ID: 212\\[2pt]
$3\log_{4}x-4\log_{16}x=2$ for $x$\\[4pt]
\noindent\textbf{Solution 5}\\[2pt]
\\[-35pt]\begin{align*}
3\log_{4}x-4\log_{16}x&=2\\[2pt]
3\log_{4}x-4\left(\displaystyle\frac{\log_{4}x}{\log_{4}16}\right)&=2\\[2pt]
3\log_{4}x-4\left(\displaystyle\frac{\log_{4}x}{2}\right)&=2\\[2pt]
3\log_{4}x-2\log_{4}x&=2\\[2pt]
\log_{4}x^3-\log_{4}x^2&=2\\[2pt]
\log_{4}\displaystyle\frac{x^3}{x^2}&=2\\[2pt]
\log_{4} x&=2\\[2pt]
x&=4^2\\[2pt]
x&=16
\end{align*}
Answer part 1: \hspace{10pt}Label\hspace{10pt}$x=$\hspace{10pt}Solution\hspace{10pt}16\\
Answer part 1 hint: \hspace{15pt}\\
\\[4pt]
\noindent\textbf{Question 6}\hspace{20pt}Experience: 50\hspace{20pt}Order: k2\hspace{20pt}Level: k2\hspace{20pt}Question-ID: 213\\[2pt]
Solve $5\log_{2}x-9\log_{8}x=10$ for $x$\\[4pt]
\noindent\textbf{Solution 6}\\[2pt]
\\[-35pt]\begin{align*}
5\log_{2}x-9\log_{8}x&=10\\[2pt]
5\log_{2}x-9\left(\displaystyle\frac{\log_{2}x}{\log_{2}8}\right)&=10\\[2pt]
5\log_{2}x-9\left(\displaystyle\frac{\log_{4}x}{3}\right)&=10\\[2pt]
\log_{2}x^5-\log_{2}x^3&=10\\[2pt]
\log_{2}\displaystyle\frac{x^5}{x^3}&=10\\[2pt]
x^2&=2^{10}\\[2pt]
x&=32\\[2pt]
\end{align*}
Answer part 1: \hspace{10pt}Label\hspace{10pt}$x=$\hspace{10pt}Solution\hspace{10pt}32\\
Answer part 1 hint: \hspace{15pt}\\
\\[4pt]
\noindent\textbf{Question 7}\hspace{20pt}Experience: 50\hspace{20pt}Order: k2\hspace{20pt}Level: k2\hspace{20pt}Question-ID: 214\\[2pt]
Solve $2\log_{2}x+5\log_{4}x=18$ for $x$\\[4pt]
\noindent\textbf{Solution 7}\\[2pt]
\\[-35pt]\begin{align*}
2\log_{2}x+5\log_{4}x&=16\\[2pt]
2\log_{2}x+5\left(\displaystyle\frac{\log_{2}x}{\log_{2}4}\right)&=18\\[2pt]
2\log_{2}x+5\left(\displaystyle\frac{\log_{2}x}{2}\right)&=18\\[2pt]
\log_{2}x^2+\log_{2}x^{\frac{5}{2}}&=18\\[2pt]
\log_{2}x^{\frac{9}{2}}&=18\\[2pt]
\displaystyle\frac{9}{2}\log_{2}x&=18\\[2pt]
\log_{2}x&=4\\[2pt]
x&=2^{4}\\[2pt]
x&=16
\end{align*}
Answer part 1: \hspace{10pt}Label\hspace{10pt}$x=$\hspace{10pt}Solution\hspace{10pt}16\\
Answer part 1 hint: \hspace{15pt}\\
\\[4pt]
\noindent\textbf{Question 8}\hspace{20pt}Experience: 35\hspace{20pt}Order: k3\hspace{20pt}Level: k3\hspace{20pt}Question-ID: 216\\[2pt]
Solve $\log_{2}(3x+2)-\log_{2}(x-8)=4$ for $x$\\[4pt]
\noindent\textbf{Solution 8}\\[2pt]
\\[-35pt]\begin{align*}
\log_{2}(3x+2)-\log_{2}(x-8)&=4\\[2pt]
\log_{2}\displaystyle\frac{3x+2}{x-8}&=4\\[2pt]
\displaystyle\frac{3x+2}{x-8}&=16\\[2pt]
3x+2&=16(x-8)\\[2pt]
3x+2&=16x-128\\[2pt]
13x&=130\\[2pt]
x&=10\\[-50pt]
\end{align*}
Answer part 1: \hspace{10pt}Label\hspace{10pt}$x=$\hspace{10pt}Solution\hspace{10pt}10\\
Answer part 1 hint: \hspace{15pt}\\
\\[4pt]
\noindent\textbf{Question 9}\hspace{20pt}Experience: 50\hspace{20pt}Order: k3\hspace{20pt}Level: k3\hspace{20pt}Question-ID: 217\\[2pt]
Solve $\log_{4}(6x+2)-\log_{4}(x-3)=2$ for $x$\\[4pt]
\noindent\textbf{Solution 9}\\[2pt]
\\[-35pt]\begin{align*}
\log_{4}(6x+2)-\log_{4}(x-3)&=2\\[2pt]
\log_{4}\displaystyle\frac{6x+2}{x-3}&=2\\[2pt]
\displaystyle\frac{6x+2}{x-3}&=16\\[2pt]
6x+2&=16(x-3)\\[2pt]
6x+2&=16x-48\\[2pt]
10x&=50\\[2pt]
x&=5
\end{align*}
Answer part 1: \hspace{10pt}Label\hspace{10pt}$x=$\hspace{10pt}Solution\hspace{10pt}5\\
Answer part 1 hint: \hspace{15pt}\\
\\[4pt]
\noindent\textbf{Question 10}\hspace{20pt}Experience: 45\hspace{20pt}Order: l2\hspace{20pt}Level: l2\hspace{20pt}Question-ID: 220\\[2pt]
Solve $\log_{5}x=8-\displaystyle\frac{15}{\log_5{x}}$ for all values of$x$\\[4pt]
\noindent\textbf{Solution 10}\\[2pt]
\\[-35pt]\begin{align*}
\log_{5}x=8-\displaystyle\frac{15}{\log_5{x}}\\[2pt]
(\log_{5}x)^2&=8\log_{5}x-15\\[2pt]
(\log_{5}x)^2-8\log_{5}x+15&=0\qquad P=15 \qquad S=-8\\[2pt]
(\log_{5}x-5)(\log_{5}x-3)&=0\qquad (-5,-3)\\[2pt]
\log_{5}x=5\qquad\log_{5}x&=3\\[2pt]
x=5^5\hspace{37pt} x&=5^3\\[2pt]
x=3125\hspace{28pt}x&=125
\end{align*}
Answer part 1: \hspace{10pt}Label\hspace{10pt}$x=$\hspace{10pt}Solution\hspace{10pt}125,3125\\
Answer part 1 hint: \hspace{15pt} Give all solutions correct to 2 d.p, exact integer solutions are also allowed, separate solutions with comma(s). (e.g: 2,-3 or 1.5,0.6)\\
\\[4pt]
\noindent\textbf{Question 11}\hspace{20pt}Experience: 45\hspace{20pt}Order: l4\hspace{20pt}Level: l4\hspace{20pt}Question-ID: 225\\[2pt]
Solve $\log_{2}(x-6)=\log_{4}(5x)+\log_{4}(2x+3)$ for all values of $x$\\[4pt]
\noindent\textbf{Solution 11}\\[2pt]
\\[-35pt]\begin{align*}
\log_{2}(x-6)&=\log_{4}(5x)+\log_{4}(2x+3)\\[2pt]
\log_{2}(x-6)&=\log_{4}5x(2x+3)\\[2pt]
\log_{2}(x-6)&=\displaystyle\frac{\log_{2}5x(2x+3)}{\log_{2}4}\\[2pt]
\log_{2}(x-6)&=\displaystyle\frac{\log_{2}5x(2x+3)}{2}\\[2pt]
2\log_{2}(x-6)&=\log_{2}5x(2x+3)\\[2pt]
\log_{2}(x-6)^2&=\log_{2}5x(2x+3)\\[2pt]
(x-6)^2&=5x(2x+3)\\[2pt]
x^2-12x+36&=10x^2+15x\\[2pt]
9x^2+27x-36&=0\\[2pt]
x^2+3x-4&=0\qquad P=-4 \qquad S=3\\[2pt]
(x+4)(x-1)&=0 \qquad (4,-1)\\[2pt]
x=-4 \qquad x&=1
\end{align*}
Answer part 1: \hspace{10pt}Label\hspace{10pt}$x=$\hspace{10pt}Solution\hspace{10pt}-4,1\\
Answer part 1 hint: \hspace{15pt} Give all solutions correct to 2 d.p, exact integer solutions are also allowed, separate solutions with comma(s). (e.g: 2,-3 or 1.5,0.6)\\
\\[4pt]
\noindent\textbf{Question 12}\hspace{20pt}Experience: 45\hspace{20pt}Order: l3\hspace{20pt}Level: l3\hspace{20pt}Question-ID: 223\\[2pt]
Solve $\log_{2}x+\log_{2}(x+1)=\log_{2}(x+28)-\log_{2}2$ for all values of $x$.\\[4pt]
\noindent\textbf{Solution 12}\\[2pt]
\\[-35pt]\begin{align*}
\log_{2}x+\log_{2}(x+1)&=\log_{2}(x+28)-\log_{2}2\\[2pt]
\log_{2}x(x+1)&=\log_{2}\displaystyle\frac{x+28}{2}\\[2pt]
x(x+1)&=\displaystyle\frac{x+28}{2}\\[2pt]
2x^2+2x&=x+28\\[2pt]
2x^2+x-28&=0\qquad P=-56 \qquad S=1\\[2pt]
\left(x-\displaystyle\frac{7}{2}\right)(x-4)&=0\qquad (-7,8) \qquad \left(-\displaystyle\frac{7}{2},4\right)\\[2pt]
x=\displaystyle\frac{7}{2}\qquad x&=4\\[2pt]
x=3.5\hspace{22pt}&
\end{align*}
Answer part 1: \hspace{10pt}Label\hspace{10pt}$x=$\hspace{10pt}Solution\hspace{10pt}3.5,4\\
Answer part 1 hint: \hspace{15pt} Give all solutions correct to 2 d.p, exact integer solutions are also allowed, separate solutions with comma(s). (e.g: 2,-3 or 1.5,0.6)\\
\\[4pt]
\noindent\textbf{Question 13}\hspace{20pt}Experience: 45\hspace{20pt}Order: l3\hspace{20pt}Level: l3\hspace{20pt}Question-ID: 222\\[2pt]
Solve $\log_{4}3-\log_{4}(x+4)=\log_{4}2-2\log_{4}x$ for all values of $x$.\\[4pt]
\noindent\textbf{Solution 13}\\[2pt]
\\[-35pt]\begin{align*}
\log_{4}3-\log_{4}(x+4)&=\log_{4}2-2\log_{4}x\\[2pt]
\log_{4}\displaystyle\frac{3}{x+4}&=\log_{4}\displaystyle\frac{2}{x^2}\\[2pt]
\displaystyle\frac{3}{x+4}&=\displaystyle\frac{2}{x^2}\\[2pt]
3x^2&=2(x+4)\\[2pt]
3x^2-2x-8&=0\qquad P=-24\qquad S=-2\\[2pt]
(x-2)\left(x+\displaystyle\frac{4}{3}\right)&=0\qquad (-6,4)\qquad \left(-2,\displaystyle\frac{4}{3}\right)\\[2pt]
x=2\qquad x&=-\displaystyle\frac{4}{3}\\[2pt]
x&=-1.33
\end{align*}
Answer part 1: \hspace{10pt}Label\hspace{10pt}$x=$\hspace{10pt}Solution\hspace{10pt}-1.33,2\\
Answer part 1 hint: \hspace{15pt} Give all solutions correct to 2 d.p, exact integer solutions are also allowed, separate solutions with comma(s). (e.g: 2,-3 or 1.5,0.6)\\
\\[4pt]
\noindent\textbf{Question 14}\hspace{20pt}Experience: 45\hspace{20pt}Order: l4\hspace{20pt}Level: l4\hspace{20pt}Question-ID: 224\\[2pt]
Solve $\log_{9}(6x^2-12x+1)=\log_{3}(x-5)$ for all values of $x$\\[4pt]
\noindent\textbf{Solution 14}\\[2pt]
\\[-35pt]\begin{align*}
\log_{9}(6x^2-12x+1)&=\log_{3}(x-5)\\[2pt]
\displaystyle\frac{\log_{3}(6x^2-12x+1)}{\log_{3}9}&=\log_{3}(x-5)\\[2pt]
\log_{3}6x^2-12x+1&=2\log_{3}(x-5)\\[2pt]
\log_{3}6x^2-12x+1&=\log_{3}(x-5)^2\\[2pt]
6x^2-12x+1&=(x-5)^2\\[2pt]
6x^2-12x+1&=x^2-10x+25\\[2pt]
5x^2-2x-24&=0\qquad P=-120 \qquad S=-2\\[2pt]
(x+2)\left(x-\displaystyle\frac{12}{5}\right)&=0\qquad (10,-12) \qquad \left(2,-\displaystyle\frac{12}{5}\right)\\[2pt]
x=-2\qquad x&=-\displaystyle\frac{12}{5}\\[2pt]
x&=-2.4
\end{align*}
Answer part 1: \hspace{10pt}Label\hspace{10pt}$x=$\hspace{10pt}Solution\hspace{10pt}-2.4,-2\\
Answer part 1 hint: \hspace{15pt} Give all solutions correct to 2 d.p, exact integer solutions are also allowed, separate solutions with comma(s). (e.g: 2,-3 or 1.5,0.6)\\
\\[4pt]
\noindent\textbf{Question 15}\hspace{20pt}Experience: 45\hspace{20pt}Order: l1\hspace{20pt}Level: l1\hspace{20pt}Question-ID: 218\\[2pt]
Solve $\log_{4}(x+9)+\log_{4}(x+3)=2$ for all values of $x$\\[4pt]
\noindent\textbf{Solution 15}\\[2pt]
\\[-35pt]\begin{align*}
\log_{4}(x+9)+\log_{4}(x+3)&=2\\[2pt]
\log_{4}(x+9)(x+3)&=2\\[2pt]
(x+9)(x+3)&=4^2\\[2pt]
x^2+12x+27&=16\\[2pt]
x^2+12x+11&=0\qquad P=11 \qquad S=12\\[2pt]
(x+11)(x+1)&=0\qquad (11,1)\\[2pt]
x&=-1
\end{align*}
Answer part 1: \hspace{10pt}Label\hspace{10pt}$x=$\hspace{10pt}Solution\hspace{10pt}-1\\
Answer part 1 hint: \hspace{15pt}\\
\\[4pt]
\noindent\textbf{Question 16}\hspace{20pt}Experience: 45\hspace{20pt}Order: l1\hspace{20pt}Level: l1\hspace{20pt}Question-ID: 219\\[2pt]
Solve $\log_{3}(x+12)+\log_{3}(x+4)=2$ for all values of $x$\\[4pt]
\noindent\textbf{Solution 16}\\[2pt]
\\[-35pt]\begin{align*}
\log_{3}(x+12)+\log_{3}(x+4)&=2\\[2pt]
\log_{3}(x+12)(x+4)&=2\\[2pt]
(x+12)(x+4)&=3^2\\[2pt]
x^2+16x+48&=9\\[2pt]
x^2+16x+39&=0\qquad P=39 \qquad S=16\\[2pt]
(x+13)(x+3)&=0\qquad (13,3)\\[2pt]
x&=-3
\end{align*}
Answer part 1: \hspace{10pt}Label\hspace{10pt}$x=$\hspace{10pt}Solution\hspace{10pt}-3\\
Answer part 1 hint: \hspace{15pt} \\
\\[4pt]
\noindent\textbf{Question 17}\hspace{20pt}Experience: 45\hspace{20pt}Order: l1\hspace{20pt}Level: l1\hspace{20pt}Question-ID: 221\\[2pt]
Solve $\log_{10}(x+10)+\log_{10}(x+4)=2$ for all values of $x$\\[4pt]
\noindent\textbf{Solution 17}\\[2pt]
\\[-35pt]\begin{align*}
\log_{10}(x+10)+\log_{10}(x+4)&=2\\[2pt]
\log_{10}(x+10)(x+4)&=2\\[2pt]
(x+10)(x+4)&=10^2\\[2pt]
x^2+14x+140&=100\\[2pt]
x^2+14x+40&=0\qquad P=40 \qquad S=14\\[2pt]
(x+10)(x+4)&=0\qquad (10,4)\\[2pt]
x&=-4
\end{align*}
Answer part 1: \hspace{10pt}Label\hspace{10pt}$x=$\hspace{10pt}Solution\hspace{10pt}-4\\
Answer part 1 hint: \hspace{15pt}\\
\\[4pt]
\noindent\textbf{Question 18}\hspace{20pt}Experience: 50\hspace{20pt}Order: j1\hspace{20pt}Level: j1\hspace{20pt}Question-ID: 236\\[2pt]
Solve $3^{2x}-12(3^x)+27=0$ for $x$\\[4pt]
\noindent\textbf{Solution 18}\\[2pt]
\\[-35pt]\begin{align*}
3^{2x}-12(3^x)+27&=0\\[2pt]
(3^x)^2-12(3^x)+27&=0\qquad y=3^x\\[2pt]
y^2-12y+27&=0\qquad P=27 \quad S=-12\\[2pt]
(y-9)(y-3)&=0\qquad (-9,-3)\\[2pt]
y=9 \qquad y&=3\\[2pt]
3^x=9\hspace{16pt} 3^x&=3\\[2pt]
x=2 \hspace{21pt} x&=1
\end{align*}
Answer part 1: \hspace{10pt}Label\hspace{10pt}$x=$\hspace{10pt}Solution\hspace{10pt}1,2\\
Answer part 1 hint: \hspace{15pt}Give all solutions correct to 2 d.p, exact integer solutions are also allowed, separate solutions with comma(s).  (e.g:  2,-3 or 1.5,0.6)  \\
\\[4pt]
\noindent\textbf{Question 19}\hspace{20pt}Experience: 50\hspace{20pt}Order: j2\hspace{20pt}Level: j2\hspace{20pt}Question-ID: 237\\[2pt]
Solve $2(5^{2x+1})-17(5^x)+3=0$ for $x$\\[4pt]
\noindent\textbf{Solution 19}\\[2pt]
\\[-35pt]\begin{align*}
2(5^{2x+1})-17(5^x)+3&=0\\[2pt]
2\!\cdot\!5\!\cdot\!(5^{2x})-17(5^x)+3&=0\\[2pt]
10(5^{x})^2-17(5^x)+3&=0\qquad y=5^x\\[2pt]
10y^2-17y+3&=0\qquad P=30 \quad S=-17\\[2pt]
\left(y-\displaystyle\frac{3}{2}\right)\left(y-\displaystyle\frac{1}{5}\right)&=0\qquad (-15,-2)\quad \left(-\displaystyle\frac{3}{2},-\displaystyle\frac{1}{5}\right)\\[2pt]
y=\displaystyle\frac{3}{2} \qquad y&=\displaystyle\frac{1}{5}\\[2pt]
5^x=\displaystyle\frac{3}{2}\hspace{16pt} 5^x&=\displaystyle\frac{1}{5}\\[2pt]
x=0.25 \hspace{6pt} x&=-1
\end{align*}
Answer part 1: \hspace{10pt}Label\hspace{10pt}$x=$\hspace{10pt}Solution\hspace{10pt}0.25,-1\\
Answer part 1 hint: \hspace{15pt}Give all solutions correct to 2 d.p, exact integer solutions are also allowed, separate solutions with comma(s).  (e.g:  2,-3 or 1.5,0.6)  \\
\\[4pt]
\noindent\textbf{Question 20}\hspace{20pt}Experience: 50\hspace{20pt}Order: k1\hspace{20pt}Level: k1\hspace{20pt}Question-ID: 211\\[2pt]
Solve $2\log_{3}x-12\log_{9}x=4$ for $x$\\[4pt]
\noindent\textbf{Solution 20}\\[2pt]
\\[-35pt]\begin{align*}
\log_{3}x-6\log_{9}x&=4\\[2pt]
\log_{3}x-6\left(\displaystyle\frac{\log_{3}x}{\log_{3}9}\right)&=4\\[2pt]
\log_{3}x-6\left(\displaystyle\frac{\log_{3}x}{2}\right)&=4\\[2pt]
\log_{3}x-3\log_{3}x&=4\\[2pt]
\log_{3}x-\log_{3}x^3&=4\\[2pt]
\log_{3}\displaystyle\frac{x}{x^3}&=4\\[2pt]
\displaystyle\frac{1}{x^2}&=3^4\\[2pt]
\displaystyle\frac{1}{x}&=3^2\\[2pt]
x&=\displaystyle\frac{1}{9}\\[2pt]
x&=0.11
\end{align*}
Answer part 1: \hspace{10pt}Label\hspace{10pt}$x=$\hspace{10pt}Solution\hspace{10pt}0.11\\
Answer part 1 hint: \hspace{15pt}Write the answer to 2 d.p.\\
\\[4pt]
\\[2pt]
\noindent\large{\textbf{End of Chapter Questions}}\\[15pt]
\noindent\Huge{\textbf{OCR A Level Maths}}\\[5pt]
\noindent\large{To be added ......}\\[20pt]
\noindent\Huge{\textbf{AQA A Level Maths}}\\[5pt]
\noindent\large{To be added ......}\\[20pt]
\noindent\Huge{\textbf{MEI A Level Maths}}\\[5pt]
\noindent\large{To be added .....}\\[20pt]
\noindent\huge{\textbf{Unit 1 C 1}}\\[18pt]
\noindent\huge{\textbf{Chapter 1 asdfasfd}}\\[15pt]
\\[2pt]
\noindent\large{\textbf{End of Chapter Questions}}\\[15pt]
\noindent\Huge{\textbf{Edexcel A Level Further Maths}}\\[5pt]
\noindent\large{To be added ......}\\[20pt]
\noindent\Huge{\textbf{MEI A Level Further Maths}}\\[5pt]
\noindent\large{To be added ......}\\[20pt]
\noindent\Huge{\textbf{Unused Questions}}\\[10pt]
\noindent\large{}\\\noindent\textbf{Question 1}\hspace{20pt}Experience: 50\hspace{20pt}Order: \hspace{20pt}Level: \hspace{20pt}Question-ID: 44\\[2pt]
A sequence is defined by $2U_{n+4}=3U_{n+3}-U_n, \quad U_1=2,U_5=5, U_2=2U_4$, find the values of $U_2,U_4$ and $U_6$.\\[4pt]
\noindent\textbf{Solution 1}\\[2pt]
\\[-35pt]\begin{align*}
2U_5&=3U_4-U_1\\[2pt]
2(5)&=3U_4-2\\[2pt]
3U_4&=12\\[2pt]
U_4&=4\\[12pt]
U_2&=2U_4\\[2pt]
U_2&=2(4)\\[2pt]
U_2&=8\\[12pt]
2U_6&=3U_5-U_2\\[2pt]
2U_6&=3(5)-8\\[2pt]
U_6&=\displaystyle\frac{7}{2}\\
\end{align*}
Choice 1: \hspace{20pt}$U_2=10 \,\, U_4=3 \,\, U_6=2 $\hspace{20pt}false\\
Choice 2: \hspace{20pt}$U_2=8 \,\, U_4=3 \,\, U_6=\displaystyle\frac{7}{2} $\hspace{20pt}false\\
Choice 3: \hspace{20pt}$U_2=6 \,\, U_4=4 \,\, U_6=2 $\hspace{20pt}false\\
Choice 4: \hspace{20pt}$U_2=6 \,\, U_4=3 \,\, U_6=\displaystyle\frac{7}{2} $\hspace{20pt}false\\
Choice 5: \hspace{20pt}$U_2=8 \,\, U_4=4 \,\, U_6=\displaystyle\frac{7}{2} $\hspace{20pt}true\\
\\[4pt]
\noindent\textbf{Question 2}\hspace{20pt}Experience: 50\hspace{20pt}Order: \hspace{20pt}Level: \hspace{20pt}Question-ID: 45\\[2pt]
A sequence is defined by $3x_{n+4}=\displaystyle\frac{2x_{n+3}}{x_n}, \quad x_1=3,x_4=9, \displaystyle\frac{x_6}{x_3}=6$, find the value of $x_5$ and $x_7$.\\[4pt]
\noindent\textbf{Solution 2}\\[2pt]
\\[-35pt]\begin{align*}
3x_5&=\displaystyle\frac{2x_{4}}{x_1}\\[2pt]
3x_5&=\displaystyle\frac{2(9)}{3}\\[2pt]
x_5&=2\\[12pt]
3x_7&=\displaystyle\frac{2x_6}{x_3}\\[2pt]
3x_7&=2(6)\\[2pt]
x_7&=4
\end{align*}
Choice 1: \hspace{20pt}$x_5=2 \,\, x_7=8$\hspace{20pt}false\\
Choice 2: \hspace{20pt}$x_5=6 \,\, x_7=4$\hspace{20pt}false\\
Choice 3: \hspace{20pt}$x_5=6 \,\, x_7=8$\hspace{20pt}false\\
Choice 4: \hspace{20pt}$x_5=6 \,\, x_7=2$\hspace{20pt}false\\
Choice 5: \hspace{20pt}$x_5=2 \,\, x_7=4$\hspace{20pt}true\\
\\[4pt]
\noindent\textbf{Question 3}\hspace{20pt}Experience: 60\hspace{20pt}Order: \hspace{20pt}Level: \hspace{20pt}Question-ID: 61\\[2pt]
A sequence is defined by the recurrence relation $Y_{n+1}=\displaystyle\frac{a^2}{Y_n}+b, Y_1=3, a,b \in \mathbb{N}$, given that $Y_2=7$ and $Y_3=\displaystyle\frac{37}{7}$ find the value of $a$ and $b$.\\[4pt]
\noindent\textbf{Solution 3}\\[2pt]
\\[-35pt]\begin{align*}
Y_2&=\displaystyle\frac{a^2}{Y_1}+b\\[2pt]
7&=\displaystyle\frac{a^2}{3}+b\hspace{20pt}(1)\\[12pt]
Y_3&=\displaystyle\frac{a^2}{Y_2}+b\\[2pt]
\displaystyle\frac{37}{7}&=\displaystyle\frac{a^2}{7}+b\hspace{20pt}(2)\\[2pt]
(1)-(2)\quad 7-\displaystyle\frac{37}{7}&=\frac{a^2}{3}+b-\left(\frac{a^2}{7}+b\right)\\[2pt]
\displaystyle\frac{12}{7}&=\displaystyle\frac{4a^2}{21}\\[2pt]
a^2&=9\\[2pt]
a&=3\\[12pt]
\text{Sub into} (1)\quad 7&=\displaystyle\frac{3^2}{3}+b\\[2pt]
b&=4
\end{align*}
Choice 1: \hspace{20pt}$a=3\quad b=3$\hspace{20pt}false\\
Choice 2: \hspace{20pt}$a=2\quad b=3$\hspace{20pt}false\\
Choice 3: \hspace{20pt}$a=2\quad b=4$\hspace{20pt}false\\
Choice 4: \hspace{20pt}$a=4\quad b=2$\hspace{20pt}false\\
Choice 5: \hspace{20pt}$a=3\quad b=4$\hspace{20pt}true\\
\\[4pt]
\noindent\textbf{Question 4}\hspace{20pt}Experience: 60\hspace{20pt}Order: \hspace{20pt}Level: \hspace{20pt}Question-ID: 60\\[2pt]
A sequence is defined by the recurrence relation $u_{n+1}=\sqrt{a}\left(u_n-\displaystyle\frac{1}{b}\right),5 u_1=4$, given that $u_2=7$ and $u_3=13$ find the value of $a$ and $b$ .\\[4pt]
\noindent\textbf{Solution 4}\\[2pt]
\\[-35pt]\begin{align*}
u_2&=\sqrt{a}\left(u_1-\displaystyle\frac{1}{b}\right)\\[2pt]
7&=\sqrt{a}\left(4-\displaystyle\frac{1}{b}\right)\hspace{20pt}(1)\\[2pt]
7&=4\sqrt{a}-\displaystyle\frac{\sqrt{a}}{b}\hspace{20pt}(2)\\[12pt]
u_3&=\sqrt{a}\left(u_2-\displaystyle\frac{1}{b}\right)\\[2pt]
13&=\sqrt{a}\left(7-\displaystyle\frac{1}{b}\right)\\[2pt]
13&=7\sqrt{a}-\displaystyle\frac{\sqrt{a}}{b}\hspace{20pt}(3)\\[12pt]
(3)-(2)\quad13-7&=7\sqrt{a}-\displaystyle\frac{\sqrt{a}}{b}-\left(4\sqrt{a}-\displaystyle\frac{\sqrt{a}}{b}\right)\\[2pt]
6&=3\sqrt{a}\\[2pt]
2&=\sqrt{a}\\[2pt]
a&=4\\[12pt]
\text{Sub into} \,\,(1)\quad 7&=\sqrt{4}\left(4-\displaystyle\frac{1}{b}\right)\\[2pt]
\displaystyle\frac{7}{2}&=4-\displaystyle\frac{1}{b}\\[2pt]
-\displaystyle\frac{1}{2}&=-\displaystyle\frac{1}{b}\\[2pt]
b&=2
\end{align*}
Choice 1: \hspace{20pt}$a=3\quad b=2$\hspace{20pt}false\\
Choice 2: \hspace{20pt}$a=4\quad b=3$\hspace{20pt}false\\
Choice 3: \hspace{20pt}$a=3\quad b=3$\hspace{20pt}false\\
Choice 4: \hspace{20pt}$a=2\quad b=3$\hspace{20pt}false\\
Choice 5: \hspace{20pt}$a=4\quad b=2$\hspace{20pt}true\\
\\[4pt]
\noindent\textbf{Question 5}\hspace{20pt}Experience: 50\hspace{20pt}Order: \hspace{20pt}Level: \hspace{20pt}Question-ID: 59\\[2pt]
A sequence is defined by the recurrence relation $Y_{n+1}=\displaystyle\frac{a^2}{Y_n}+b, Y_1=3, a,b>0$, given that $Y_2=7$ and $Y_3=\displaystyle\frac{37}{7}$ find the value of $a$ and $b$.\\[4pt]
\noindent\textbf{Solution 5}\\[2pt]
\\[-35pt]\begin{align*}
Y_2&=\displaystyle\frac{a^2}{Y_1}+b\\[2pt]
7&=\displaystyle\frac{a^2}{3}+b\hspace{20pt}(1)\\[12pt]
Y_3&=\displaystyle\frac{a^2}{Y_2}+b\\[2pt]
\displaystyle\frac{37}{7}&=\displaystyle\frac{a^2}{7}+b\hspace{20pt}(2)\\[2pt]
(1)-(2)\quad 7-\displaystyle\frac{37}{7}&=\frac{a^2}{3}+b-\left(\frac{a^2}{7}+b\right)\\[2pt]
\displaystyle\frac{12}{7}&=\displaystyle\frac{4a^2}{21}\\[2pt]
a^2&=9\\[2pt]
a&=3\\[12pt]
\text{Sub into} (1)\quad 7&=\displaystyle\frac{3^2}{3}+b\\[2pt]
b&=4
\end{align*}
Choice 1: \hspace{20pt}$a=3\quad b=3$\hspace{20pt}false\\
Choice 2: \hspace{20pt}$a=2\quad b=3$\hspace{20pt}false\\
Choice 3: \hspace{20pt}$a=2\quad b=4$\hspace{20pt}false\\
Choice 4: \hspace{20pt}$a=4\quad b=2$\hspace{20pt}false\\
Choice 5: \hspace{20pt}$a=3\quad b=4$\hspace{20pt}true\\
\\[4pt]
\noindent\textbf{Question 6}\hspace{20pt}Experience: 100\hspace{20pt}Order: \hspace{20pt}Level: \hspace{20pt}Question-ID: 38\\[2pt]
A sequence is defined by $X_n=\displaystyle\frac{a+1}{n}+b$, given the Sum of the first three terms is $\displaystyle\frac{2}{3}$ and the fifth term is $-\displaystyle\frac{3}{5}$, find the values of $a$ and $b$.\\[4pt]
\noindent\textbf{Solution 6}\\[2pt]
\\[-35pt]\begin{align*}
S_3&=\left(\displaystyle\frac{a+1}{(1)}+b\right)+\left(\displaystyle\frac{a+1}{(2)}+b\right)+\left(\displaystyle\frac{a+1}{(3)}+b\right)\\[2pt]
S_3&=a+\displaystyle\frac{a}{2}+\displaystyle\frac{a}{3}+3b+1+\displaystyle\frac{1}{2}+\frac{1}{3}\\[2pt]
S_3&=\displaystyle\frac{11}{6}a+3b+\frac{11}{6}\quad S_3=\frac{2}{3}\\[2pt]
\displaystyle\frac{11}{6}a+3b+\frac{11}{6}&=\frac{2}{3}\\[2pt]
11a+18b+11&=4\quad(1)\\[12pt]
X_5&=\displaystyle\frac{a+1}{5}+b\hspace{20pt}X_5=-\frac{3}{5}\\[2pt]
\displaystyle\frac{a+1}{5}+b&=-\frac{3}{5}\\[2pt]
a+1+5b&=-3 \quad (2)\\[2pt]
11a+11+55b&=-33\quad (3)\\[12pt]
(3)-(1)\quad 11a+11+55b-(11a+18b+11)&=-33-4\\[2pt]
37b&=-37\\[2pt]
b&=-1\\[12pt]
\text{sub into}\quad (2) \quad a+1+5(-1)&=-3\\[2pt]
a-4&=-3\\[2pt]
a&=1
\end{align*}
Choice 1: \hspace{20pt}$a=1\quad b=2$\hspace{20pt}false\\
Choice 2: \hspace{20pt}$a=3 \quad b=-1$\hspace{20pt}false\\
Choice 3: \hspace{20pt}$a=3 \quad b=2$\hspace{20pt}false\\
Choice 4: \hspace{20pt}$a=2 \quad b=-1$\hspace{20pt}false\\
Choice 5: \hspace{20pt}$a=1 \quad b=-1$\hspace{20pt}true\\
\\[4pt]
\noindent\textbf{Question 7}\hspace{20pt}Experience: 35\hspace{20pt}Order: z\hspace{20pt}Level: z\hspace{20pt}Question-ID: 215\\[2pt]
Solve $\log_{3}(x+6)-\log_{3}\left(\displaystyle\frac{x}{2}-4\right)=2$ for $x$\\[4pt]
\noindent\textbf{Solution 7}\\[2pt]
\\[-35pt]\begin{align*}
\log_{3}(x+6)-\log_{3}\left(\displaystyle\frac{x}{2}-4\right)&=2\\[2pt]
\log_{3}\displaystyle\frac{x+6}{\left(\displaystyle\frac{x}{2}-4\right)}&=2\\[2pt]
\displaystyle\frac{2(x+6)}{x-8}&=9\\[2pt]
2(x+6)&=9(x-8)\\[2pt]
2x+12&=9x-72\\[2pt]
7x&=84\\[2pt]
x&=12
\end{align*}
Answer part 1: \hspace{10pt}Label\hspace{10pt}$x=$\hspace{10pt}Solution\hspace{10pt}12\\
Answer part 1 hint: \hspace{15pt}\\
\\[4pt]
\noindent\textbf{Question 8}\hspace{20pt}Experience: 45\hspace{20pt}Order: l1\hspace{20pt}Level: l1\hspace{20pt}Question-ID: 241\\[2pt]
Solve $\log_{2}(x-3)+\log_{2}(3x-7)=4$ for all values of $x$\\[4pt]
\noindent\textbf{Solution 8}\\[2pt]
\\[-35pt]\begin{align*}
\log_{2}(x-3)+\log_{2}(3x-7)&=4\\[2pt]
\log_{2}(x-3)(3x-7)&=4\\[2pt]
(x-3)(3x-7)&=2^4\\[2pt]
3x^2-16x+21&=16\\[2pt]
3x^2-16x+5&=0\qquad P=15 \qquad S=-16\\[2pt]
(x-5)\left(x-\displaystyle\frac{1}{5}\right)&=0\qquad (-15,-1) \qquad \left(-5,-\displaystyle\frac{1}{5}\right)\\[2pt]
x&=5
\end{align*}
Answer part 1: \hspace{10pt}Label\hspace{10pt}$x=$\hspace{10pt}Solution\hspace{10pt}5\\
Answer part 1 hint: \hspace{15pt}\\
\\[4pt]
\noindent\textbf{Question 9}\hspace{20pt}Experience: 50\hspace{20pt}Order: k1\hspace{20pt}Level: k1\hspace{20pt}Question-ID: 238\\[2pt]
Solve $4\log_{2}x+3\log_{8}x=10$ for $x$\\[4pt]
\noindent\textbf{Solution 9}\\[2pt]
\\[-35pt]\begin{align*}
4\log_{2}x+3\log_{8}x&=10\\[2pt]
4\log_{2}x+3\left(\displaystyle\frac{\log_{2}x}{\log_{2}8}\right)&=10\\[2pt]
4\log_{2}x+3\left(\displaystyle\frac{\log_{2}x}{3}\right)&=10\\[2pt]
4\log_{2}x+\log_{2}x&=10\\[2pt]
\log_{2}x^4+\log_{2}x&=10\\[2pt]
\log_{2}x^4\,\,\text{x}\,\,x&=10\\[2pt]
x^5&=2^{10}\\[2pt]
x&=2^2\\[2pt]
x&=4
\end{align*}
Answer part 1: \hspace{10pt}Label\hspace{10pt}$x=$\hspace{10pt}Solution\hspace{10pt}4\\
Answer part 1 hint: \hspace{15pt}\\
\\[4pt]
\noindent\textbf{Question 10}\hspace{20pt}Experience: 50\hspace{20pt}Order: k2\hspace{20pt}Level: k2\hspace{20pt}Question-ID: 239\\[2pt]
Solve $\log_{3}x-8\log_{9}x=6$ for $x$\\[4pt]
\noindent\textbf{Solution 10}\\[2pt]
\\[-35pt]\begin{align*}
\log_{3}x-8\log_{9}x&=6\\[2pt]
\log_{3}x-8\left(\displaystyle\frac{\log_{3}x}{\log_{3}9}\right)&=6\\[2pt]
\log_{3}x-8\left(\displaystyle\frac{\log_{3}x}{2}\right)&=6\\[2pt]
\log_{3}x-4\log_{3}x&=6\\[2pt]
\log_{3}x-\log_{3}x^4&=6\\[2pt]
\log_{3}\displaystyle\frac{x}{x^4}&=6\\[2pt]
\displaystyle\frac{1}{x^3}&=3^6\\[2pt]
\displaystyle\frac{1}{x}&=3^2\\[2pt]
x&=\displaystyle\frac{1}{9}\\[2pt]
x&=0.11
\end{align*}
Answer part 1: \hspace{10pt}Label\hspace{10pt}$x=$\hspace{10pt}Solution\hspace{10pt}0.11\\
Answer part 1 hint: \hspace{15pt}Write the answer to 2 d.p.\\
\\[4pt]
\noindent\textbf{Question 11}\hspace{20pt}Experience: 50\hspace{20pt}Order: z\hspace{20pt}Level: z\hspace{20pt}Question-ID: 229\\[2pt]
Find values for $x$ and $y$ from the following, $x>0 , y>0$:
\begin{align*}
3\log_{y}x^2&=4\\[2pt]
y^2&=x^2+12x\\[-22pt]
\end{align*}
\noindent\textbf{Solution 11}\\[2pt]
\\[-35pt]\begin{align*}
3\log_{y}x^2&=4\hspace{53pt}(1)\\[2pt]
y^2&=x^2+12x\qquad (2)\\[2pt]
\text{Simplifying}\,\,(1)\qquad \log_{y}x^2&=\displaystyle\frac{4}{3}\\[2pt]
x^2&=y^{\displaystyle\frac{4}{3}}\\[2pt]
x^6&=y^4\\[2pt]
x^3&=y^2\hspace{48pt}(3)\\[2pt]
\text{Sub}\,\,(3)\,\,\text{into}\,\,(2)\hspace{38pt} x^3&=x^2+12x\\[2pt]
x^3-x^2-12x&=0\\[2pt]
x(x^2-x-12)&=0\qquad P=-12\qquad S=-1\\[2pt]
x(x-4)(x+3)&=0\qquad (-4,3)\\[2pt]
x&=4\\[12pt]
\text{Sub}\,\,x=4\,\,\text{into}\,\,(3)\hspace{27pt} 4^3&=y^2\\[2pt]
y^2&=64\\[2pt]
y&=8
\end{align*}
Answer part 1: \hspace{10pt}Label\hspace{10pt}$x=$\hspace{10pt}Solution\hspace{10pt}4\\
Answer part 1 hint: \hspace{15pt}\\
Answer part 2: \hspace{10pt}Label\hspace{10pt}$y=$\hspace{10pt}Solution\hspace{10pt}8\\
Answer part 2 hint: \hspace{15pt}\\
\\[4pt]
\noindent\textbf{Question 12}\hspace{20pt}Experience: 50\hspace{20pt}Order: z\hspace{20pt}Level: z\hspace{20pt}Question-ID: 228\\[2pt]
Find values for $x$ and $y$ from the following, $x>0$:
\begin{align*}
3\log_{x}y-\log_{x}125&=2\\[2pt]
125y&=x^2\\[-22pt]
\end{align*}
\noindent\textbf{Solution 12}\\[2pt]
\\[-35pt]\begin{align*}
3\log_{x}y-\log_{x}125&=2\hspace{25pt} (1)\\[2pt]
125y&=x^2\qquad (2)\\[2pt]
\text{Simplifying} \,\, (1)\qquad 3\log_{x}y-3\log_{x}5&=2\\[2pt]
\log_{x}y-\log_{x}5&=\displaystyle\frac{2}{3}\\[2pt]
\log_{x}\displaystyle\frac{y}{5}&=\displaystyle\frac{2}{3}\\[2pt]
\displaystyle\frac{y}{5}&=x^{\displaystyle\frac{2}{3}}\\[2pt]
y&=5x^{\displaystyle\frac{2}{3}}\hspace{14pt}(3)\\[2pt]
\text{Simplifying}\,\,(2)\qquad y&=\displaystyle\frac{x^2}{125}\hspace{15pt}(4)\\[2pt]
(3) = (4)\qquad 5x^{\displaystyle\frac{2}{3}}&=\displaystyle\frac{x^2}{125}\\[2pt]
x^{\displaystyle\frac{4}{3}}&=625\\[2pt]
x&=625^{\displaystyle\frac{3}{4}}\\[2pt]
x&=125\\[12pt]
\text{Sub}\,\,x=125\,\,\text{into}\,\,(2)\qquad 125y&=(125)^2\\[2pt]
y&=125
\end{align*}
Answer part 1: \hspace{10pt}Label\hspace{10pt}$x=$\hspace{10pt}Solution\hspace{10pt}125\\
Answer part 1 hint: \hspace{15pt}\\
Answer part 2: \hspace{10pt}Label\hspace{10pt}$y=$\hspace{10pt}Solution\hspace{10pt}125\\
Answer part 2 hint: \hspace{15pt}\\
\\[4pt]
\noindent\textbf{Question 13}\hspace{20pt}Experience: 50\hspace{20pt}Order: z\hspace{20pt}Level: z\hspace{20pt}Question-ID: 227\\[2pt]
Find values for $x$ and $y$ from the following:
\begin{align*}
\log_{4}4x+\log_{5}5y&=4\\[2pt]
\log_{2}2x+6\log_{25}y&=6\\[-22pt]
\end{align*}
\noindent\textbf{Solution 13}\\[2pt]
\\[-35pt]\begin{align*}
\log_{4}4x+\log_{5}5y&=4\qquad (1)\\[2pt]
\log_{2}2x+6\log_{25}y&=6\qquad (2)\\[2pt]
\text{Simplifying}\,\,(1)\qquad \log_{4}4+\log_{4}x+\log_{5}5+\log_{5}y&=4\\[2pt]
1+\displaystyle\frac{\log_{2}x}{\log_{2}4}+1+\log_{5}y&=4\\[2pt]
\displaystyle\frac{\log_{2}x}{2}+\log_{5}y&=2\\[2pt]
\log_{2}x+2\log_{5}y&=4\qquad (3)\\[2pt]
\text{Simplifying}\,\,(2)\hspace{50pt}\log_{2}2+\log_{2}x+\displaystyle\frac{6\log_{5}y}{\log_{5}25}&=6\\[2pt]
1+\log_{2}x+\displaystyle\frac{6\log_{5}y}{2}&=6\\[2pt]
\log_{2}x+3\log_{5}y&=5\qquad (4)\\[2pt]
(4)\,\,-\,\,(3)\qquad\log_{2}x+3\log_{5}y-(\log_{2}x+2\log_{5}y)&=5-4\\[2pt]
\log_{5}y&=1\\[2pt]
y&=5\\[12pt]
\text{Sub} \,\,y=5\,\,\text{into}\,\,(3)\hspace{73pt} \log_{2}x+2\log_{5}5&=4\\[2pt]
\log_{2}x+2&=4\\[2pt]
\log_{2}x&=2\\[2pt]
x&=2^2\\[2pt]
x&=4
\end{align*}
Answer part 1: \hspace{10pt}Label\hspace{10pt}$x=$\hspace{10pt}Solution\hspace{10pt}4\\
Answer part 1 hint: \hspace{15pt}\\
Answer part 2: \hspace{10pt}Label\hspace{10pt}$y=$\hspace{10pt}Solution\hspace{10pt}5\\
Answer part 2 hint: \hspace{15pt}\\
\\[4pt]
\noindent\textbf{Question 14}\hspace{20pt}Experience: 50\hspace{20pt}Order: z\hspace{20pt}Level: z\hspace{20pt}Question-ID: 226\\[2pt]
Find values for $x$ and $y$ from the following:
\begin{align*}
5\log_{3}x-2\log_{5}y&=8\\[2pt]
3\log_{27}x&=4\log_{25}y\\[-22pt]
\end{align*}
\noindent\textbf{Solution 14}\\[2pt]
\\[-35pt]\begin{align*}
5\log_{3}x-2\log_{5}y&=8\hspace{50pt} (1)\\[2pt]
3\log_{27}x&=4\log_{25}y\qquad (2)\\[2pt]
\text{Simplifying} \,\,(2) \hspace{77pt} \displaystyle\frac{3\log_{3}x}{\log_{3}27}&=\displaystyle\frac{4\log_{5}y}{\log_{5}25}\\[2pt]
\displaystyle\frac{3\log_{3}x}{3}&=\displaystyle\frac{4\log_{5}y}{2}\\[2pt]
\log_{3}x&=2\log_{5}y\hspace{26pt} (3)\\[2pt]
\text{Sub}\,\,(3)\,\, \text{into}\,\,(1)\qquad 5(2\log_{5}y)-2\log_{5}y&=8\\[2pt]
10\log_{5}y-2\log_{5}y&=8\\[2pt]
8\log_{5}y&=8\\[2pt]
\log_{5}y&=1\\[2pt]
y&=5\\[12pt]
\text{Sub}\,\,y=5\,\,\text{into}\,\,(1)\hspace{23pt} 5\log_{3}x-2\log_{5}5&=8\\[2pt]
5\log_{3}x-2&=8\\[2pt]
\log_{3}x&=2\\[2pt]
x&=3^2\\[2pt]
x&=9
\end{align*}
Answer part 1: \hspace{10pt}Label\hspace{10pt}$x=$\hspace{10pt}Solution\hspace{10pt}9\\
Answer part 1 hint: \hspace{15pt}\\
Answer part 2: \hspace{10pt}Label\hspace{10pt}$y=$\hspace{10pt}Solution\hspace{10pt}5\\
Answer part 2 hint: \hspace{15pt}\\
\\[4pt]
\noindent\textbf{Question 15}\hspace{20pt}Experience: 35\hspace{20pt}Order: k3\hspace{20pt}Level: k3\hspace{20pt}Question-ID: 240\\[2pt]
Solve $\log_{3}(4x-5)-\log_{3}(x-5)=2$ for $x$\\[4pt]
\noindent\textbf{Solution 15}\\[2pt]
\\[-35pt]\begin{align*}
\log_{3}(4x-5)-\log_{3}(x-5)&=2\\[2pt]
\log_{3}\displaystyle\frac{4x-5}{x-5}&=2\\[2pt]
\displaystyle\frac{4x-5}{x-5}&=9\\[2pt]
4x-5&=9(x-5)\\[2pt]
4x-5&=9x-45\\[2pt]
5x&=40\\[2pt]
x&=8
\end{align*}
Answer part 1: \hspace{10pt}Label\hspace{10pt}$x=$\hspace{10pt}Solution\hspace{10pt}8\\
Answer part 1 hint: \hspace{15pt}\\
\\[4pt]
\end{document}
