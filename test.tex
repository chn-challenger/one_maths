\documentclass{article}
\usepackage[math]{iwona}
\usepackage[fleqn]{amsmath}
\usepackage{scrextend}
\usepackage{amsmath,amssymb}
\changefontsizes[12pt]{8pt}
\usepackage[a4paper, left=0.7in,right=0.7in,top=1in,bottom=1in]{geometry}
\pagenumbering{gobble}
\usepackage{fancyhdr}
\renewcommand{\headrulewidth}{0pt}
\pagestyle{fancy}
\lfoot{\textcopyright\, One Maths Limited}
\rfoot{}
\begin{document}
\noindent\Huge{\textbf{Edexcel A Level Maths}}\\[5pt]
\noindent\large{Core 1, Core 2, Core 3, Core 4 and two units from (Decision 1, Decision 2, Mechanics 1, Mechanics 2, Statistics 1 and Statistics 2).  Each unit is out of 100 UMS, giving a total of 600 UMS.  480 and above is grade A.  To obtain an A*, total score must be 480 or more, and total of C3 and C4 must be 180 or more.}\\[20pt]
\noindent\huge{\textbf{Unit 1 Core 1}}\\[18pt]
\noindent\huge{\textbf{Chapter 1 Sequence and Series 1}}\\[15pt]
\noindent\large{\textbf{Lesson 1 Sequence and Summation}}\\[12pt]
\noindent\textbf{Question 1}\\
A sequence is defined by $a_n=3n^2-4$, find the value of $a_2$.\\
\noindent\textbf{Solution 1}\\
\begin{align*}
a_2=3(2)^2-4=8\\[2pt]
\end{align*}
\noindent\textbf{Question 2}\\[2pt]
A sequence is defined by $x_n=3n^2-5n+2$, find the value of $n$ such that $x_n=14$.\\[4pt]
\noindent\textbf{Solution 2}\\[2pt]
\begin{align*}
x_n=3n^2-5n+2&=14\\[2pt]
3n^2-5n-12&=0 \hspace{20pt} S=-5 \quad P=-36\\[2pt]
\left(n+\displaystyle\frac{4}{3}\right)(n-3)&=0 \hspace{19pt} (4,-9) \quad \left(\displaystyle\frac{4}{3},-3\right)\\[2pt]
n&=3\\
\end{align*}
\noindent\textbf{Question 3}\\[2pt]
A sequence is defined by $x_n=6n-3$, find the value of $x_3$ and $x_5$.\\[4pt]
\noindent\textbf{Solution 3}\\[2pt]
\begin{align*}
x_3=6(3)-3=15\\[2pt]
x_5=6(5)-3=27\\[2pt]
\end{align*}
\noindent\textbf{Question 4}\\[2pt]
A sequence is defined by $X_n=2n-1$, find the value of $n$ such that $a_n=15$.\\[4pt]
\noindent\textbf{Solution 4}\\[2pt]
\begin{align*}
15&=2n-1\\[2pt]
n&=8\\[10pt]
\end{align*}
\noindent\textbf{Question 5}\\[2pt]
A sequence is defined by $u_n=an-b$, find the sum of the first four terms in terms of $a$ and $b$.\\[4pt]
\noindent\textbf{Solution 5}\\[2pt]
\begin{align*}
u_1+u_2+u_3+u_4=(a-b)+(2a-b)+(3a-b)+(4a-b)=10a-4b
\end{align*}
\noindent\textbf{Question 6}\\[2pt]
A sequence is defined by $x_n=an^2-4$, find the sum of the first three terms in terms of $a$.\\[4pt]
\noindent\textbf{Solution 6}\\[2pt]
\begin{align*}
x_1+x_2+x_3=(a-4)+(4a-4)+(9a-4)=14a-12
\end{align*}
\noindent\textbf{Question 7}\\[2pt]
A sequence is defined by $y_n=an^2+bn+c$, find the sum of the first three terms in terms of $a,b$ and $c$.\\[4pt]
\noindent\textbf{Solution 7}\\[2pt]
\begin{align*}
y_1+y_2+y_3=(a+b+c)+(4a+2b+c)+(9a+3b+c)=14a+6b+3c
\end{align*}
wrong choice\\[4pt]
\noindent\textbf{Question 8}\\[2pt]
A sequence is defined by $x_n=4n-b$, find the third term in terms of $b$ .\\[4pt]
\noindent\textbf{Solution 8}\\[2pt]
\begin{align*}
x_3&=4(3)-b\\
x_3&=12-b
\end{align*}
\noindent\textbf{Question 9}\\[2pt]
A sequence is defined by $U_n=\displaystyle\frac{a}{n}+b$, find the fourth term in terms of $a$ and $b$ .\\[4pt]
\noindent\textbf{Solution 9}\\[2pt]
\begin{align*}
U_4=\displaystyle\frac{a}{4}+b
\end{align*}
\noindent\textbf{Question 10}\\[2pt]
A sequence is defined by $y_n=\displaystyle\frac{a-3b}{n^2}$, find the fifth term in terms of $a$ and $b$ .\\[4pt]
\noindent\textbf{Solution 10}\\[2pt]
\begin{align*}
y_5&=\displaystyle\frac{a-3b}{(5)^2}\\[2pt]
y_5&=\displaystyle\frac{a-3b}{25}\\[2pt]
\end{align*}
\noindent\textbf{Question 11}\\[2pt]
A sequence is defined by $U_n=an+2b$, given the Sum of the first four terms is $26$ and the fifth term is $9$, find the values of $a$ and $b$.\\[4pt]
\noindent\textbf{Solution 11}\\[2pt]
\begin{align*}
S_4&=(a+2b)+(2a+2b)+(3a+2b)+(4a+2b)\\[2pt]
S_4&=10a+8b\hspace{20pt} S_4=26\\[2pt]
10a+8b&=26\\[2pt]
5a+4b&=13\quad (1)\\[12pt]
U_5&=5a+2b\hspace{20pt}U_5=9\\[2pt]
5a+2b&=9\quad (2)\\[12pt]
(1)-(2)\quad 5a+4b-(5a+2b)&=13-9\\[2pt]
2b&=4\\[2pt]
b&=2\\[12pt]
\text{sub into}\quad (2) \quad 5a+2(2)&=9\\[2pt]
5a&=5\\[2pt]
a&=1
\end{align*}
\noindent\textbf{Question 12}\\[2pt]
A sequence is defined by $U_{n+1}=U_n-4, \quad U_1=20$, find the values of $U_2,U_3$ and $U_4$.\\[4pt]
\noindent\textbf{Solution 12}\\[2pt]
\begin{align*}
U_2&=U_1-4=20-4=16\\[2pt]
U_3&=U_2-4=16-4=12\\[2pt]
U_4&=U_3-4=12-4=8
\end{align*}
\noindent\textbf{Question 13}\\[2pt]
A sequence is defined by $X_{n+1}=X_n+5, \quad X_4=17$, find the values of $X_1,X_2$ and $X_3$.\\[4pt]
\noindent\textbf{Solution 13}\\[2pt]
\begin{align*}
X_4&=X_3+5\\[2pt]
17&=X_3+5\\[2pt]
X_3&=12\\[12pt]
X_3&=X_2+5\\[2pt]
12&=X_2+5\\[2pt]
X_2&=7\\[12pt]
X_2&=X_1+5\\[2pt]
7&=X_1+5\\[2pt]
X_1&=2\\[12pt]
\end{align*}
\noindent\textbf{Question 14}\\[2pt]
A sequence is defined by $a_{n+1}=(a_n)^2-4, \quad a_1=2$, find the values of $a_2,a_3$ and $a_4$.\\[4pt]
\noindent\textbf{Solution 14}\\[2pt]
\begin{align*}
a_2&=(a_1)^2-4=4-4=0\\[2pt]
a_3&=(a_2)^2-4=0-4=-4\\[2pt]
a_4&=(a_3)^2-4=(-4)^2-4=16-4=12
\end{align*}
\noindent\textbf{Question 15}\\[2pt]
A sequence is defined by $y_{n+2}=3y_{n+1}-y_n, \quad y_1=3,y_2=2$, find the values of $y_3,y_4$ and $y_5$.\\[4pt]
\noindent\textbf{Solution 15}\\[2pt]
\begin{align*}
y_3&=3(y_2)-y_1=3(2)-3=3\\[2pt]
y_4&=3(y_3)-y_2=3(3)-2=7\\[2pt]
y_5&=3(y_4)-y_3=3(7)-3=18\\[-30pt]
\end{align*}
\noindent\textbf{Question 16}\\[2pt]
Calculate the following sum:
\begin{align*}
\sum_{r=2}^{5} (r-1)
\end{align*}
\noindent\textbf{Solution 16}\\[2pt]
\begin{align*}
\sum_{r=2}^{5} (r-1)&=(2-1)+(3-1)+(4-1)+(5-1)\\[2pt]
&=1+2+3+4\\[2pt]
&=10
\end{align*}
\noindent\textbf{Question 17}\\[2pt]
Calculate the following sum:
\begin{align*}
\sum_{r=4}^{8} (r^2-2r+1)
\end{align*}
\noindent\textbf{Solution 17}\\[2pt]
\begin{align*}
&\sum_{r=4}^{8} (r^2-2r+1)\\[2pt]
=\,\,&\sum_{r=4}^{8} (r-1)^2\\[2pt]
=\,\,&(4-1)^2+(5-1)^2+(6-1)^2+(7-1)^2+(8-1)^2\\[2pt]
=\,\,&9+16+25+36+49\\[2pt]
=\,\,&135
\end{align*}
\noindent\textbf{Question 18}\\[2pt]
Calculate the following sum:
\begin{align*}
\sum_{r=5}^{9} U_r \hspace{20pt}U_r=3r^2+4
\end{align*}
\noindent\textbf{Solution 18}\\[2pt]
\begin{align*}
&\sum_{r=5}^{9} U_r\\[2pt]
=\,\,&\sum_{r=5}^{9} 3r^2+4\\[2pt]
=\,\,&(3(5)^2+4)+(3(6)^2+4)+(3(7)^2+4)+(3(8)^2+4)+(3(9)^2+4) \\[2pt]
=\,\,&785
\end{align*}
\noindent\textbf{Question 19}\\[2pt]
Calculate the following sum:
\begin{align*}
\sum_{r=1}^{3} a_r \hspace{20pt}a_r=4r-1
\end{align*}
\noindent\textbf{Solution 19}\\[2pt]
\begin{align*}
&\sum_{r=1}^{3} a_r\\[2pt]
=\,\,&\sum_{r=1}^{3} 4r-1\\[2pt]
=\,\,&(4(1)-1)+(4(2)-1)+(4(3)-1) \\[2pt]
=\,\,&21
\end{align*}
\noindent\textbf{Question 20}\\[2pt]
A sequence is defined by $U_{n+1}=3(U_n -1), U_1=2,$ find the following sum: $\displaystyle\sum_{2}^{4} (U_r+2)^2$\\[4pt]
\noindent\textbf{Solution 20}\\[2pt]
\begin{align*}
U_2&=3(2-1)\\[2pt]
U_2&=3\\[12pt]
U_3&=3(3-1)\\[2pt]
U_3&=6\\[12pt]
U_4&=3(6-1)\\[2pt]
U_4&=15\\[12pt]
\displaystyle\sum_{2}^{4} (U_r+2)^2&=(U_2+2)^2+(U_3+2)^2+(U_4+2)^2\\[2pt]
&=(3+2)^2+(6+2)^2+(15+2)^2\\[2pt]
&=5^2+8^2+17^2\\[2pt]
&=378
\end{align*}
\noindent\textbf{Question 21}\\[2pt]
Calculate the following sum:
\begin{align*}
\sum_{r=1}^{4} (2r+4)
\end{align*}
\noindent\textbf{Solution 21}\\[2pt]
\begin{align*}
&\sum_{r=1}^{4} (2r+4)\\[2pt]
=\,\,&\sum_{r=1}^{4} 2(r+2)\\[2pt]
=\,\,&2\sum_{r=1}^{4} (r+2)\\[2pt]
=\,\,&2[(1+2)+(2+2)+(3+2)+(4+2)]\\[2pt]
=\,\,&2(3+4+5+6)\\[2pt]
=\,\,&36
\end{align*}
\noindent\textbf{Question 22}\\[2pt]
Calculate the following sum:
\begin{align*}
\sum_{r=3}^{6} (r^2-1)
\end{align*}
\noindent\textbf{Solution 22}\\[2pt]
\begin{align*}
&\sum_{r=3}^{6} (r^2-1)\\[2pt]
=\,\,&(3^2-1)+(4^2-1)+(5^2-1)+(6^2-1)\\[2pt]
=\,\,&8+15+24+35\\[2pt]
=\,\,&82\\[-20pt]
\end{align*}
\noindent\textbf{Question 23}\\[2pt]
Calculate the following sum:
\begin{align*}
\sum_{r=1}^{45} 2
\end{align*}
\noindent\textbf{Solution 23}\\[2pt]
\begin{align*}
&\sum_{r=1}^{45} 2\\[2pt]
=\,\,&2+2+2+2+2+...+2\\[2pt]
=\,\,&2 \, \text{x} \, 45\\[2pt]
=\,\,&90
\end{align*}
\noindent\textbf{Question 24}\\[2pt]
Calculate the following sum:
\begin{align*}
\sum_{r=1}^{100} 5
\end{align*}
\noindent\textbf{Solution 24}\\[2pt]
\begin{align*}
&\sum_{r=1}^{100} 5\\[2pt]
=\,\,&5+5+5+5+5+5+...+5\\[2pt]
=\,\,&5 \, \text{x} \, 100\\[2pt]
=\,\,&500
\end{align*}
\noindent\textbf{Question 25}\\[2pt]
A sequence is defined by $U_n=2n+3$, find the value of $U_2, U_4$ and $U_5$.\\[4pt]
\noindent\textbf{Solution 25}\\[2pt]
\begin{align*}
U_2&=2(2)+3=7\\[2pt]
U_4&=2(4)+3=11\\[2pt]
U_5&=2(5)+3=13\\[2pt]
\end{align*}
\noindent\textbf{Question 26}\\[2pt]
A sequence is defined by $u_n=2n^2-5n-3$, find the value of $n$ such that $u_n=9$.\\[4pt]
\noindent\textbf{Solution 26}\\[2pt]
\begin{align*}
u_n=2n^2-5n-3&=9\\[2pt]
2n^2-5n-12&=0\hspace{20pt}S=-5 \quad P=-24\\[2pt]
\left(n+\displaystyle\frac{3}{2}\right)(n-4)&=0\hspace{20pt} (3,-8)\quad\left(\displaystyle\frac{3}{2},-4\right)\\[2pt]
n&=4
\end{align*}
\noindent\textbf{Question 27}\\[2pt]
A sequence is defined by $a_n=an^2+b$, given the Sum of the first five terms is $-5$ and the sixth term is $4$, find the values of $a$ and $b$.\\[4pt]
\noindent\textbf{Solution 27}\\[2pt]
\begin{align*}
S_5&=(a+b)+(4a+b)+(9a+b)+(16a+b)\\[2pt]
S_5&=30a+5b\hspace{20pt} S_5=-5\\[2pt]
30a+5b&=-5\\[2pt]
6a+b&=-1\quad (1)\\[12pt]
a_6&=25a+b\hspace{20pt}a_6=4\\[2pt]
36a+b&=4\quad (2)\\[12pt]
(2)-(1)\quad 36a+b-(6a+b)&=4-(-1)\\[2pt]
30a&=5\\[2pt]
a&=\displaystyle\frac{1}{6}\\[12pt]
\text{sub into}\quad (1) \quad 6\left(\displaystyle\frac{1}{6}\right)+b&=-1\\[2pt]
b&=-2\\[2pt]
\end{align*}
\noindent\large{\textbf{Lesson 2 Arithmetic Sequence 1}}\\[12pt]
\noindent\textbf{Question 1}\\[2pt]
Evaluate $\displaystyle\sum_{r=1}^{15} (5r+2)$\\[4pt]
\noindent\textbf{Solution 1}\\[2pt]
\begin{align*}
\displaystyle\sum_{r=1}^{15} (5r+2)&=7+12+17+22+...+77\\[2pt]
a&=7\quad l=77 \quad n=15\\[2pt]
\displaystyle\sum_{r=1}^{15} (5r+2)&=\displaystyle\frac{15}{2}(7+77)\\[2pt]
&=630
\end{align*}
\noindent\textbf{Question 2}\\[2pt]
How many terms are there in the arithmetic sequence 19,21,23,...,87\\[4pt]
\noindent\textbf{Solution 2}\\[2pt]
\begin{align*}
a&=19 \quad d=2\\[2pt]
U_n&=a+(n-1)d\\[2pt]
87&=19+(n-1)2\\[2pt]
n-1&=34\\[2pt]
n&=35
\end{align*}
\noindent\textbf{Question 3}\\[2pt]
How many terms are there in the arithmetic sequence 21,26,31,...,256\\[4pt]
\noindent\textbf{Solution 3}\\[2pt]
\begin{align*}
a&=21 \quad d=5\\[2pt]
U_n&=a+(n-1)d\\[2pt]
256&=21+(n-1)5\\[2pt]
n-1&=47\\[2pt]
n&=48
\end{align*}
\noindent\textbf{Question 4}\\[2pt]
How many terms are there in the arithmetic sequence 88,86,84,...,22\\[4pt]
\noindent\textbf{Solution 4}\\[2pt]
Reverse the order of the sequence$\quad$ 22,24,26,28...88
\begin{align*}
a&=88 \quad d=2\\[2pt]
U_n&=a+(n-1)d\\[2pt]
88&=22+(n-1)2\\[2pt]
n-1&=33\\[2pt]
n&=34\\[-100pt]
\end{align*}
\noindent\textbf{Question 5}\\[2pt]
Evaluate $S=1+2+3+4+...+50$\\[4pt]
\noindent\textbf{Solution 5}\\[2pt]
\begin{align*}
S&=1+2+3+4+...+50\\[2pt]
S&=50+49+48+47+...+1\\[2pt]
2S&=51\,\,\text{x}\,\,50\\[2pt]
S&=\displaystyle\frac{51\,\,\text{x}\,\,50}{2}\\[2pt]
S&=1275
\end{align*}
\noindent\textbf{Question 6}\\[2pt]
Evaluate $T=2+4+6+8+...+100$\\[4pt]
\noindent\textbf{Solution 6}\\[2pt]
\begin{align*}
T&=2+4+6+8+...+100\\[2pt]
T&=100+98+96+94+...+2\\[2pt]
2T&=102\,\,\text{x}\,\,50\\[2pt]
T&=\displaystyle\frac{102\,\,\text{x}\,\,50}{2}\\[2pt]
T&=2550
\end{align*}
\noindent\textbf{Question 7}\\[2pt]
Evaluate $R=1+3+5+7+...+99$\\[4pt]
\noindent\textbf{Solution 7}\\[2pt]
\begin{align*}
R&=1+3+5+7+...+99\\[2pt]
R&=99+97+95+93+...+1\\[2pt]
2R&=100\,\,\text{x}\,\,100\\[2pt]
R&=\displaystyle\frac{100\,\,\text{x}\,\,100}{2}\\[2pt]
R&=5000
\end{align*}
\noindent\textbf{Question 8}\\[2pt]
Evaluate $S=1+2+3+4+...+200$\\[4pt]
\noindent\textbf{Solution 8}\\[2pt]
\begin{align*}
S&=1+2+3+4+...+200\\[2pt]
S&=200+199+198+197+...+1\\[2pt]
2S&=201\,\,\text{x}\,\,200\\[2pt]
S&=\displaystyle\frac{201\,\,\text{x}\,\,200}{2}\\[2pt]
S&=20100
\end{align*}
\noindent\textbf{Question 9}\\[2pt]
Evaluate $T=102+104+106+108+...+200$\\[4pt]
\noindent\textbf{Solution 9}\\[2pt]
\begin{align*}
T&=102+104+106+108+...+200\\[2pt]
T&=200+198+196+194+...+102\\[2pt]
2T&=302\,\,\text{x}\,\,50\\[2pt]
T&=\displaystyle\frac{302\,\,\text{x}\,\,50}{2}\\[2pt]
T&=7550
\end{align*}
\noindent\textbf{Question 10}\\[2pt]
Find the sum of all numbers divisible by 5 between $1$ and $300$\\[4pt]
\noindent\textbf{Solution 10}\\[2pt]
\begin{align*}
&300 \div 5 = 60\\[2pt]
&\Rightarrow \quad \text{last term }= 300\\[2pt]
S&=5+10+15+...+300\\[12pt]
a&=5\quad d=5 \quad U_n=300\\[2pt]
U_n&=a+(n-1)d\\[2pt]
300&=5+(n-1)5\\[2pt]
n&=60\\[12pt]
S&=\displaystyle\frac{n}{2}(a+l)\\[2pt]
S&=\displaystyle\frac{60}{2}(5+300)\\[2pt]
S&=9150\\[-50pt]
\end{align*}
\noindent\textbf{Question 11}\\[2pt]
Find the sum of all numbers divisible by 7 between $1$ and $200$\\[4pt]
\noindent\textbf{Solution 11}\\[2pt]
\begin{align*}
&200 \div 7 = 28 \,\, \text{remainder}\,\, 4\\[2pt]
&\Rightarrow \quad \text{last term }= 7 \,\,\text{x}\,\, 28=196\\[2pt]
S&=7+14+21+...+196\\[12pt]
a&=7\quad d=7 \quad U_n=196\\[2pt]
U_n&=a+(n-1)d\\[2pt]
196&=7+(n-1)7\\[2pt]
n&=28\\[12pt]
S&=\displaystyle\frac{n}{2}(a+l)\\[2pt]
S&=\displaystyle\frac{28}{2}(7+196)\\[2pt]
S&=2842\\[-140pt]
\end{align*}
\noindent\textbf{Question 12}\\[2pt]
Evaluate $S=27+31+35+39+...+107$\\[4pt]
\noindent\textbf{Solution 12}\\[2pt]
\begin{align*}
S&=27+31+35+39+...+107\\[2pt]
a&=27\quad d=4 \quad U_n=107\\[2pt]
U_n&=a+(n-1)d\\[2pt]
107&=27+(n-1)4\\[2pt]
n&=21\\[12pt]
S&=\displaystyle\frac{n}{2}(a+l)\\[2pt]
S&=\displaystyle\frac{21}{2}(27+107)\\[2pt]
S&=1407\\[-140pt]
\end{align*}
\noindent\textbf{Question 13}\\[2pt]
Evaluate $T=31+33+35+37+...+81$\\[4pt]
\noindent\textbf{Solution 13}\\[2pt]
\begin{align*}
T&=31+33+35+37+...+81\\[2pt]
a&=31\quad d=2 \quad U_n=81\\[2pt]
U_n&=a+(n-1)d\\[2pt]
81&=31+(n-1)2\\[2pt]
n&=26\\[12pt]
S&=\displaystyle\frac{n}{2}(a+l)\\[2pt]
S&=\displaystyle\frac{26}{2}(31+81)\\[2pt]
S&=1456\\[-140pt]
\end{align*}
\noindent\textbf{Question 14}\\[2pt]
Evaluate $R=97+92+87+82+...+22$\\[4pt]
\noindent\textbf{Solution 14}\\[2pt]
Reverse the sequence: $R=22+27+32+27+...+97$
\begin{align*}
R&=22+27+32+27+...+97\\[2pt]
a&=22\quad d=5 \quad U_n=97\\[2pt]
U_n&=a+(n-1)d\\[2pt]
97&=22+(n-1)5\\[2pt]
n&=16\\[12pt]
S&=\displaystyle\frac{n}{2}(a+l)\\[2pt]
S&=\displaystyle\frac{16}{2}(22+97)\\[2pt]
S&=952\\
\end{align*}
\noindent\textbf{Question 15}\\[2pt]
Evaluate $\displaystyle\sum_{r=9}^{35} (3r-1)$\\[4pt]
\noindent\textbf{Solution 15}\\[2pt]
\begin{align*}
\displaystyle\sum_{r=9}^{35} (3r-1)&=26+29+32+35+...+104\\[2pt]
a&=26\quad l=104 \quad n=27\\[2pt]
\displaystyle\sum_{r=9}^{35} (3r-1)&=\displaystyle\frac{27}{2}(26+104)\\[2pt]
&=1755
\end{align*}
\noindent\textbf{Question 16}\\[2pt]
Evaluate $\displaystyle\sum_{r=1}^{20} (3r-1)$\\[4pt]
\noindent\textbf{Solution 16}\\[2pt]
\begin{align*}
\displaystyle\sum_{r=1}^{20} (3r-1)&=2+5+8+11+...+59\\[2pt]
a&=2\quad l=59 \quad n=20\\[2pt]
\displaystyle\sum_{r=1}^{20} (3r-1)&=\displaystyle\frac{20}{2}(2+59)\\[2pt]
&=610
\end{align*}
\noindent\textbf{Question 17}\\[2pt]
Evaluate $\displaystyle\sum_{r=21}^{45} (2r-25)$\\[4pt]
\noindent\textbf{Solution 17}\\[2pt]
\begin{align*}
\displaystyle\sum_{r=21}^{45} (2r-25)&=17+19+21+23+...+65\\[2pt]
a&=17\quad l=65 \quad n=25\\[2pt]
\displaystyle\sum_{r=21}^{45} (2r-25)&=\displaystyle\frac{25}{2}(17+65)\\[2pt]
&=1025
\end{align*}
\noindent\textbf{Question 18}\\[2pt]
The first three terms of an arithmetic sequence are 3,5,7, find $U_{10}$\\[4pt]
\noindent\textbf{Solution 18}\\[2pt]
\begin{align*}
a&=3\quad n=10 \quad d=5-3=2\\[2pt]
U_n&=a+(n-1)d\\[12pt]
U_{10}&=3+(10-1)2=21\\[2pt]
\end{align*}
\noindent\textbf{Question 19}\\[2pt]
The first four terms of an arithmetic sequence are 5,9,13,17, find $A_7$\\[4pt]
\noindent\textbf{Solution 19}\\[2pt]
\begin{align*}
a&=5 \quad n=7 \quad d=9-5=4\\[2pt]
A_n&=a+(n-1)d\\[2pt]
A_7&=5+(7-1)4=29\\[2pt]
\end{align*}
\noindent\textbf{Question 20}\\[2pt]
The first three terms of an arithmetic sequence are 22,19,16, find $X_6$\\[4pt]
\noindent\textbf{Solution 20}\\[2pt]
\begin{align*}
a&=22 \quad n=6 \quad d=22-19=3\\[2pt]
X_n&=a+(n-1)d\\[2pt]
X_7&=22+(6-1)3=37
\end{align*}
\noindent\textbf{Question 21}\\[2pt]
$a_n$ is an arithmetic sequence, given that $a_3=13$ and $a_6=19$, find $a_{11}$\\[4pt]
\noindent\textbf{Solution 21}\\[2pt]
\begin{align*}
a_n&=a+(n-1)d\\[2pt]
a_3&=a+(3-1)d=a+2d=13\quad (1)\\[2pt]
a_6&=a+(6-1)d=a+5d=19\quad (2)\\[2pt]
(2)-(1)\quad a+5d-(a+2d)&=19-13\\[2pt]
3d&=6\\[2pt]
d&=2\\[12pt]
\text{Sub into} (1) \quad a+2(2)&=13\\[2pt]
a&=9\\[12pt]
a_{11}&=9+(11-1)2=29
\end{align*}
\noindent\textbf{Question 22}\\[2pt]
$U_n$ is an arithmetic sequence, given that $U_4=25$ and $U_9=40$, find $U_{13}$\\[4pt]
\noindent\textbf{Solution 22}\\[2pt]
\begin{align*}
U_n&=a+(n-1)d\\[2pt]
U_4&=a+(4-1)d=a+3d=25\quad (1)\\[2pt]
U_9&=a+(9-1)d=a+8d=40\quad (2)\\[2pt]
(2)-(1)\quad a+8d-(a+3d)&=40-25\\[2pt]
5d&=15\\[2pt]
d&=3\\[12pt]
\text{Sub into} (1) \quad a+3(3)&=25\\[2pt]
a&=16\\[12pt]
U_{13}&=16+(13-1)3=52\\[-40pt]
\end{align*}
\noindent\textbf{Question 23}\\[2pt]
$X_n$ is an arithmetic sequence, given that $X_{13}=51$ and $X_{19}=33$, find $X_{10}$\\[4pt]
\noindent\textbf{Solution 23}\\[2pt]
\begin{align*}
X_n&=a+(n-1)d\\[2pt]
X_{13}&=a+(13-1)d=a+12d=51\quad (1)\\[2pt]
X_{19}&=a+(19-1)d=a+18d=33\quad (2)\\[2pt]
(2)-(1)\quad a+18d-(a+12d)&=33-51\\[2pt]
6d&=-18\\[2pt]
d&=-3\\[12pt]
\text{Sub into} (1) \quad a+12(-3)&=51\\[2pt]
a&=87\\[12pt]
X_{10}&=87+(10-1)(-3)=60\\[-40pt]
\end{align*}
\noindent\textbf{Question 24}\\[2pt]
$u_n$ is an arithmetic sequence, given that $u_{3}=5$ and $u_{7}=13$, for what value of $n$ is $a_n=71$\\[4pt]
\noindent\textbf{Solution 24}\\[2pt]
\begin{align*}
u_n&=a+(n-1)d\\[2pt]
u_3&=a+(3-1)d=a+2d=5 \hspace{15pt} (1)\\[2pt]
u_7&=a+(7-1)d=a+6d=13\quad (2) \\[2pt]
(2)-(1)\quad a+6d-(a+2d)&=13-5\\[2pt]
4d&=8\\[2pt]
d&=2\\[12pt]
\text{Sub into} (1) \quad a+2(2)&=5\\[2pt]
a&=1\\[12pt]
u_n&=1+(n-1)2=71\\[2pt]
n-1&=35\\[2pt]
n&=36\\[-60pt]
\end{align*}
\noindent\textbf{Question 25}\\[2pt]
The first three terms of an arithmetic sequence are 11,14,17, find a $n$ for which $U_n=83$\\[4pt]
\noindent\textbf{Solution 25}\\[2pt]
\begin{align*}
u_n&=83 \quad a=11 \quad d=3\\[2pt]
u_n&=a+(n-1)d\\[2pt]
83&=11+(n-1)3\\[2pt]
n-1&=24\\[2pt]
n&=25
\end{align*}
\noindent\textbf{Question 26}\\[2pt]
$Y_n$ is an arithmetic sequence, given that $Y_{15}=51$ and $X_{19}=71$, find $Y_{26}$\\[4pt]
\noindent\textbf{Solution 26}\\[2pt]
\begin{align*}
Y_n&=a+(n-1)d\\[2pt]
Y_{15}&=a+(15-1)d=a+14d=51 \quad (1)\\[2pt]
Y_{19}&=a+(19-1)d=a+18d=71\quad (2) \\[2pt]
(2)-(1)\quad a+18d-(a+14d)&=71-51\\[2pt]
4d&=20\\[2pt]
d&=5\\[12pt]
\text{Sub into} (1) \quad a+14(5)&=51\\[2pt]
a&=-19\\[12pt]
Y_{26}&=-19+(26-1)5=106\\[-60pt]
\end{align*}
\noindent\textbf{Question 27}\\[2pt]
Kendrick decides to open up a savings account. He puts in £100 for the first month, £120 for the second month and an extra £20 for subsequent months till he's putting in £300 a month. Find the total amount he's saved in 2 years.\\[4pt]
\noindent\textbf{Solution 27}\\[2pt]
Sequence goes: 100,120,140,160,180,200...300,300,300,300...$\\[2pt]$
\begin{align*}
U_n&=a+(n-1)d\\[2pt]
U_n&=300\quad a=100 \quad d=20\\[2pt]
300&=100+(n-1)20\\[2pt]
n&=11\\[12pt]
S_n&=\displaystyle\frac{n}{2}(a+l)\\[2pt]
n&=11\quad a=100 \quad l=300\\[2pt]
S_{11}&=\displaystyle\frac{11}{2}(100+300)\\[2pt]
S_{11}&=2200\\[12pt]
&\text{Every term after is 300}\\[2pt]
\sum_{r=12}^{24}300&=13 \,\, \text{x} \,\, 300\\[2pt]
&=3900\\[12pt]
\Rightarrow \quad \text{Total days}&=2200+3900=6100\\[2pt]
\end{align*}
\noindent\textbf{Question 28}\\[2pt]
Avery is playing with 340 sticks, she puts them in rows. The first row has 7 sticks, next row has 13 sticks, subsequent rows have 6 more sticks then the previous row. She has enough for $k$ rows but not enough for $k+1$ rows. Find k.\\[4pt]
\noindent\textbf{Solution 28}\\[2pt]
Sequence goes: 7,13,19,25,31,37....$\\[2pt]$
Not having enough for k+1 rows means that $S_k\leq340$
\begin{align*}
	S_n&=\displaystyle\frac{n}{2}(2a+(k-1)d)\\[2pt]
	S_k&=\displaystyle\frac{k}{2}(2(7)+(k-1)6)\\[2pt]
	S_k&=k(7+3(k-1))\\[2pt]
	S_k&=k(3k+4)\\[2pt]
	S_k&=3k^2+4k \qquad (1)\\[12pt]
	S_k&\leq 340 \\[2pt]
	(1)\qquad 3k^2+4k& \leq 340\\[2pt]
	3k^2+4k-340&\leq 0\qquad P=-1020 \quad S=4\\[2pt]
	\left(k+\displaystyle\frac{34}{3}\right)(k-10)&\leq 0 \qquad (34,-30) \qquad \left(\displaystyle\frac{34}{3},-10\right)\\[2pt]
	k&=10\\[-80pt]
\end{align*}
\noindent\textbf{Question 29}\\[2pt]
Griffin is training daily for a cycling marathon in 100 days. He cycles 10km on the first day, 11km on the second day and 1 more km then the previous day till he's cycling 40km a day. Calculate the total number of km he's cycled as training for the marathon.\\[4pt]
\noindent\textbf{Solution 29}\\[2pt]
Sequence goes: 10,11,12,13,14,15...40,40,40,40...$\\[2pt]$
\begin{align*}
U_n&=a+(n-1)d\\[2pt]
U_n&=40\quad a=10 \quad d=1\\[2pt]
40&=10+(n-1)1\\[2pt]
n&=31\\[12pt]
S_n&=\displaystyle\frac{n}{2}(a+l)\\[2pt]
n&=31\quad a=10 \quad l=40\\[2pt]
S_{31}&=\displaystyle\frac{31}{2}(10+40)\\[2pt]
S_{31}&=775\\[12pt]
&\text{Every term after is 40}\\[2pt]
\sum_{r=32}^{100}40&=69 \,\, \text{x} \,\, 40\\[2pt]
&=2760\\[12pt]
\Rightarrow \quad \text{Total days}&=775+2760=3535\\[2pt]
\end{align*}
\noindent\textbf{Question 30}\\[2pt]
Heidi is training daily for a swimming competition in 60 days. She swims 10 laps on the first day, 12 laps on the second day and 2 more laps then the previous day till she's swimming 30 laps a day. Calculate the total number of laps she's swum as training for the competition.\\[4pt]
\noindent\textbf{Solution 30}\\[2pt]
Sequence goes: 10,12,14,16,18,20...30,30,30,30...$\\[2pt]$
\begin{align*}
U_n&=a+(n-1)d\\[2pt]
U_n&=30\quad a=10 \quad d=2\\[2pt]
30&=10+(n-1)2\\[2pt]
n&=11\\[12pt]
S_n&=\displaystyle\frac{n}{2}(a+l)\\[2pt]
n&=11\quad a=10 \quad l=30\\[2pt]
S_{11}&=\displaystyle\frac{11}{2}(10+30)\\[2pt]
S_{11}&=220\\[12pt]
&\text{Every term after is 30}\\[2pt]
\sum_{r=12}^{60}30&=49 \,\, \text{x} \,\, 30\\[2pt]
&=1470\\[12pt]
\Rightarrow \quad \text{Total days}&=220+1470=1690
\end{align*}
\noindent\textbf{Question 31}\\[2pt]
Find the sum of all numbers divisible by 3 between $2$ and $200$\\[4pt]
\noindent\textbf{Solution 31}\\[2pt]
\begin{align*}
&200 \div 3 = 66 \,\, \text{remainder}\,\, 2\\[2pt]
&\Rightarrow \quad \text{last term }= 3 \,\,\text{x}\,\, 66=198\\[2pt]
S&=3+6+9+...+198\\[12pt]
a&=3\quad d=3 \quad U_n=198\\[2pt]
U_n&=a+(n-1)d\\[2pt]
198&=3+(n-1)3\\[2pt]
n&=66\\[12pt]
S&=\displaystyle\frac{n}{2}(a+l)\\[2pt]
S&=\displaystyle\frac{66}{2}(3+198)\\[2pt]
S&=6633\\
\end{align*}
\noindent\textbf{Question 32}\\[2pt]
James is playing with 324 sticks, she puts them in rows. The first row has 5 sticks, next row has 9 sticks, subsequent rows have 4 more sticks then the previous row. She has enough for $k$ rows but not enough for $k+1$ rows. Find k.\\[4pt]
\noindent\textbf{Solution 32}\\[2pt]
Sequence goes: 5,9,13,17,21,25....$\\[2pt]$
Not having enough for k+1 rows means that $S_k\leq324$
\begin{align*}
	S_n&=\displaystyle\frac{n}{2}(2a+(k-1)d)\\[2pt]
	S_k&=\displaystyle\frac{k}{2}(2(5)+(k-1)4)\\[2pt]
	S_k&=k(5+2k-2)\\[2pt]
	S_k&=k(2k+3)\\[2pt]
	S_k&=2k^2+3k \qquad (1)\\[12pt]
	S_k&\leq 324 \\[2pt]
	(1)\qquad 2k^2+3k& \leq 324\\[2pt]
	2k^2+3k-324&\leq 0\qquad P=-648 \quad S=3\\[2pt]
	\left(k+\displaystyle\frac{27}{2}\right)(k-12)&\leq 0 \qquad (27,-24)\qquad \left(\displaystyle\frac{27}{2},-12\right)\\[2pt]
	k&=12\\
\end{align*}
\noindent\large{\textbf{Lesson 3 Recurrence Relations}}\\[12pt]
\noindent\textbf{Question 1}\\[2pt]
A sequence is defined by the recurrence relation $X_{n+1}=\sqrt{k}X_n-2, X_1=2,k>0$, given that $X_3=2$ find the value of $k$.\\[4pt]
\noindent\textbf{Solution 1}\\[2pt]
\begin{align*}
X_2&=\sqrt{k}X_1-2\\[2pt]
X_2&=2\sqrt{k}-2\\[12pt]
X_3&=\sqrt{k}X_2-2\\[2pt]
X_3&=\sqrt{k}(2\sqrt{k}-2)-2\\[2pt]
X_3&=2k-2\sqrt{k}-2\quad \text{set}\quad x=\sqrt{k}\\[2pt]
X_3&=2x^2-2x-2 \quad X_3=2\\[2pt]
2&=2x^2-2x-2\\[2pt]
1&=x^2-x-1\\[2pt]
0&=x^2-x-2 \hspace{37pt} S=-1 \quad P=-2\\[2pt]
0&=(x-2)(x+1)\hspace{20pt} (-2,1)\\[2pt]
\sqrt{k}&=2\\[2pt]
k&=4
\end{align*}
\noindent\textbf{Question 2}\\[2pt]
A sequence is defined by the recurrence relation $U_{n+1}=aU_n+\displaystyle\frac{1}{b}, U_1=3$, given that $U_2=7$ and $U_3=15$ find the value of $a$ and $b$.\\[4pt]
\noindent\textbf{Solution 2}\\[2pt]
\begin{align*}
U_2&=aU_1+\displaystyle\frac{1}{b} \quad U_2=7\\[2pt]
7&=3a+\displaystyle\frac{1}{b}\quad (1)\\[12pt]
U_3&=aU_2+\displaystyle\frac{1}{b} \quad U_2=7,U_3=15\\[2pt]
15&=7a+\displaystyle\frac{1}{b}\quad (2)\\[12pt]
(2)-(1)\quad 15-7&=7a+\displaystyle\frac{1}{b}-\left(3a+\displaystyle\frac{1}{b}\right)\\[2pt]
8&=4a\\[2pt]
a&=2\\[2pt]
\text{Sub into}\,\, (1)\quad 7&=3(2)+\displaystyle\frac{1}{b}\\[2pt]
\displaystyle\frac{1}{b}&=1\\[2pt]
b&=1\\[2pt]
\end{align*}
\noindent\textbf{Question 3}\\[2pt]
A sequence is defined by the recurrence relation $a_{n+1}=ka_n-4,k>0, a_1=5$, given that $\displaystyle\sum_{r=1}^{3} a_r=19$, find the value of $k$.\\[4pt]
\noindent\textbf{Solution 3}\\[2pt]
\begin{align*}
a_2&=ka_1-4\\[2pt]
a_2&=5k-4\\[12pt]
a_3&=ka_2-4\\[2pt]
a_3&=k(5k-4)-4\\[2pt]
a_3&=5k^2-4k-4\\[12pt]
\displaystyle\sum_{r=1}^{3} a_r&=a_1+a_2+a_3\\[2pt]
\displaystyle\sum_{r=1}^{3} a_r&=(5)+(5k-4)+(5k^2-4k-4)\\[2pt]
\displaystyle\sum_{r=1}^{3} a_r&=5k^2+k-3\hspace{20pt}\displaystyle\sum_{r=1}^{3} a_r=19\\[2pt]
19&=5k^2+k-3\\[2pt]
0&=5k^2+k-22\hspace{43pt}S=1 \quad P=-110\\[2pt]
0&=\left(k+\displaystyle\frac{11}{5}\right)(k-2)\hspace{20pt}(11,-10)\quad \Rightarrow \quad \left(\displaystyle\frac{11}{5},-2\right)\\[2pt]
k&=2
\end{align*}
\noindent\textbf{Question 4}\\[2pt]
A sequence is defined by the recurrence relation $U_{n+1}=5U_n-\displaystyle\frac{1}{k},  k>0, U_1=2$, given that $\displaystyle\sum_{r=1}^{4} U_r=293$, find the value of $k$.\\[4pt]
\noindent\textbf{Solution 4}\\[2pt]
\begin{align*}
U_2&=5U_1-\displaystyle\frac{1}{k}\\[2pt]
U_2&=5(2)-\displaystyle\frac{1}{k}\\[2pt]
U_2&=10-\displaystyle\frac{1}{k}\\[12pt]
U_3&=5U_2-\displaystyle\frac{1}{k}\\[2pt]
U_3&=5\left(10-\displaystyle\frac{1}{k}\right)-\displaystyle\frac{1}{k}\\[2pt]
U_3&=50-\displaystyle\frac{6}{k}\\[12pt]
U_4&=5U_3-\displaystyle\frac{1}{k}\\[2pt]
U_4&=5\left(50-\displaystyle\frac{6}{k}\right)-\displaystyle\frac{1}{k}\\[2pt]
U_4&=250-\displaystyle\frac{31}{k}\\[12pt]
\displaystyle\sum_{r=1}^{4} U_r&=U_1+U_2+U_3+U_4\\[2pt]
\displaystyle\sum_{r=1}^{4} U_r&=(2)+\left(10-\displaystyle\frac{1}{k}\right)+\left(50-\displaystyle\frac{6}{k}\right)+\left(250-\displaystyle\frac{31}{k}\right)\\[2pt]
\displaystyle\sum_{r=1}^{4} U_r&=312-\displaystyle\frac{38}{k}\quad\displaystyle\sum_{r=1}^{4} U_r=293\\[2pt]
312-\displaystyle\frac{38}{k}&=293\\[2pt]
19&=\displaystyle\frac{38}{k}\\[2pt]
k&=2\\[-30pt]
\end{align*}
\noindent\textbf{Question 5}\\[2pt]
A sequence is defined by the recurrence relation $X_{n+1}=\displaystyle\frac{k}{X_n}+3, X_1=1$, given that $2\displaystyle\sum_{r=1}^{3} X_r=21$, find the value of $k$.\\[4pt]
\noindent\textbf{Solution 5}\\[2pt]
\begin{align*}
X_2&=\displaystyle\frac{k}{X_1}+3\\[2pt]
X_2&=\displaystyle\frac{k}{1}+3\\[2pt]
X_2&=k+3\\[12pt]
X_3&=\displaystyle\frac{k}{X_2}+3\\[2pt]
X_3&=\displaystyle\frac{k}{k+3}+3\\[12pt]
\displaystyle\sum_{r=1}^{3} X_r&=X_1+X_2+X_3\\[2pt]
\displaystyle\sum_{r=1}^{3} X_r&=(1)+(k+3)+\left(\displaystyle\frac{k}{k+3}+3\right)\\[2pt]
\displaystyle\sum_{r=1}^{3} X_r&=k+7+\displaystyle\frac{k}{k+3}\quad 2\displaystyle\sum_{r=1}^{3} X_r=21\\[2pt]
21&=2\left(k+7+\displaystyle\frac{k}{k+3}\right)\\[2pt]
21&=2k+14+\displaystyle\frac{2k}{k+3}\\[2pt]
7&=2k+\displaystyle\frac{2k}{k+3}\\[2pt]
7(k+3)&=2k(k+3)+2k\\[2pt]
7k+21&=2k^2+6k+2k\\[2pt]
0&=2k^2-k-21\hspace{29pt}S=-1\quad P=-42\\[2pt]
0&=\left(k+\displaystyle\frac{7}{2}\right)(k-3)\hspace{10pt}(7,-6)\quad \Rightarrow \quad \left(\displaystyle\frac{7}{2},-3\right)\\[2pt]
k&=3
\end{align*}
\noindent\textbf{Question 6}\\[2pt]
A sequence is defined by the recurrence relation $a_{n+1}=a_n^2-a_n$, given that $a_n$ is a positive sequence and that $a_3=132$ find the value of $a_1$.\\[4pt]
\noindent\textbf{Solution 6}\\[2pt]
\begin{align*}
a_3&=a_2^2-a_2\\[2pt]
132&=a_2^2-a_2\\[2pt]
0&=a_2^2-a_2-132 \hspace{35pt} S=1 \quad P=-132\\[2pt]
0&=(a_2+11)(a_2-12)\hspace{15pt} (11,-12)\\[2pt]
a_2&=12\\[12pt]
a_2&=a_1^2-a_1\\[2pt]
12&=a_1^2-a_1\\[2pt]
0&=a_1^2-a_1-12\\[2pt]
0&=(a_1-4)(a_1+3)\\[2pt]
a_1&=4
\end{align*}
\noindent\textbf{Question 7}\\[2pt]
A sequence is defined by the recurrence relation $U_{n+1}=5U_n-\displaystyle\frac{6}{U_n}$, given that  $U_3 =13, U_2 > 0$, find the value of $U_2$.\\[4pt]
\noindent\textbf{Solution 7}\\[2pt]
\begin{align*}
U_3&=5U_2-\displaystyle\frac{6}{U_2}\\[2pt]
13&=5U_2-\displaystyle\frac{6}{U_2}\\[2pt]
0&=5U_2-13 -\displaystyle\frac{6}{U_2}\\[2pt]
0&=5(U_2)^2-13U_2 -6\hspace{20pt}S=-13\quad P=-30\\[2pt]
0&=\left(U_2+\displaystyle\frac{2}{5}\right)(U_2-3)\hspace{15pt}(2,-15)\quad \left(\displaystyle\frac{2}{5},-3\right)\\[2pt]
U_2&=3
\end{align*}
\noindent\textbf{Question 8}\\[2pt]
A sequence is defined by the recurrence relation $Y_{n+1}=3Y_n-5$, given that  $Y_3 =7$, find the value of $Y_1$.\\[4pt]
\noindent\textbf{Solution 8}\\[2pt]
\begin{align*}
Y_3&=3Y_2-5\\[2pt]
7&=3Y_2-5\\[2pt]
Y_2&=4\\[12pt]
Y_2&=3Y_1-5\\[2pt]
4&=3Y_1-5\\[2pt]
Y_1&=3\\
\end{align*}
\noindent\textbf{Question 9}\\[2pt]
A sequence is defined by the recurrence relation $a_{n+1}=a_n-\displaystyle\frac{2a_n+6}{a_n+3}$, given that  $a_2 =5$, find the value of $a_1$.\\[4pt]
\noindent\textbf{Solution 9}\\[2pt]
\begin{align*}
a_2&=a_1-\displaystyle\frac{2a_1+6}{a_1+3}\\[2pt]
5&=a_1-\displaystyle\frac{2a_1+6}{a_1+3}\\[2pt]
5(a_1+3)&=a_1(a_1+3)-(2a_1+6)\\[2pt]
5a_1+15&=(a_1)^2+3a_1-2a_1-6\\[2pt]
0&=(a_1)^2-4a_1-21\hspace{20pt}S=-4 \quad P=-21\\[2pt]
0&=(a_1+3)(a_1-7)\hspace{20pt}(3,-7)\\[2pt]
a_1&=7
\end{align*}
\noindent\textbf{Question 10}\\[2pt]
A sequence is defined by the recurrence relation $X_{n+1}=3(X_n)^2-11$, given that  $X_1 =2$, find $\displaystyle\sum_{r=1}^{4} X_r$.\\[4pt]
\noindent\textbf{Solution 10}\\[2pt]
\begin{align*}
X_2&=3(X_1)^2-11\\[2pt]
X_2&=3(2)^2-11\\[2pt]
X_2&=1\\[12pt]
X_3&=3(X_2)^2-11\\[2pt]
X_3&=3(1)^2-11\\[2pt]
X_3&=-8\\[12pt]
X_4&=3(X_3)^2-11\\[2pt]
X_4&=3(-8)^2-11\\[2pt]
X_4&=181\\[12pt]
\displaystyle\sum_{r=1}^{4} X_r&=X_1+X_2+X_3+X_4\\[2pt]
\displaystyle\sum_{r=1}^{4} X_r&=(2)+(1)+(-8)+(181)\\[2pt]
\displaystyle\sum_{r=1}^{4} X_r&=176
\end{align*}
\noindent\textbf{Question 11}\\[2pt]
A sequence is defined by the recurrence relation $U_{n+2}=3U_{n+1}-U_n+5$, given that  $U_1 =4,U_2=2$, find $\displaystyle\sum_{r=1}^{4} U_r$.\\[4pt]
\noindent\textbf{Solution 11}\\[2pt]
\begin{align*}
U_3&=3U_2-U_1+5\\[2pt]
U_3&=3(2)-(4)+5\\[2pt]
U_3&=7\\[12pt]
U_4&=3U_3-U_2+5\\[2pt]
U_4&=3(7)-(2)+5\\[2pt]
U_4&=24\\[12pt]
\displaystyle\sum_{r=1}^{4} U_r&=U_1+U_2+U_3+U_4\\[2pt]
\displaystyle\sum_{r=1}^{4} U_r&=4+2+7+24\\[2pt]
\displaystyle\sum_{r=1}^{4} U_r&=37\\[2pt]
\end{align*}
\noindent\textbf{Question 12}\\[2pt]
A sequence is defined by the recurrence relation $Y_{n+1}=21-2Y_n$, given that  $Y_1 =5$, find $\displaystyle\sum_{r=2}^{4} Y_r$.\\[4pt]
\noindent\textbf{Solution 12}\\[2pt]
\begin{align*}
Y_2&=21-2Y_1\\[2pt]
Y_2&=21-2(5)\\[2pt]
Y_2&=11\\[12pt]
Y_3&=21-2Y_2\\[2pt]
Y_3&=21-2(11)\\[2pt]
Y_3&=-1\\[12pt]
Y_4&=21-2Y_3\\[2pt]
Y_4&=21-2(-1)\\[2pt]
Y_4&=23\\[12pt]
\displaystyle\sum_{r=2}^{4} Y_r&=Y_2+Y_3+Y_4\\[2pt]
\displaystyle\sum_{r=2}^{4} Y_r&=11+(-1)+23\\[2pt]
\displaystyle\sum_{r=2}^{4} Y_r&=33
\end{align*}
\noindent\textbf{Question 13}\\[2pt]
A sequence is defined by the recurrence relation $X_{n+1}=5-X_n$, given that  $X_1 =7$, find $\displaystyle\sum_{r=1}^{20} X_r$.\\[4pt]
\noindent\textbf{Solution 13}\\[2pt]
\begin{align*}
X_2&=5-X_1=5-7=-2\\[7pt]
X_3&=5-X_2=5-(-2)=7\\[7pt]
X_4&=5-X_3=5-7=-2\\[7pt]
X_5&=5-X_4=5-(-2)=7\\[7pt]
\displaystyle\sum_{r=1}^{20} X_r &= X_1+X_2+X_3+X_4+...+X_{20}\\[2pt]
\displaystyle\sum_{r=1}^{20} X_r &= -2+7+-2+7+-2+...+7\\[2pt]
\displaystyle\sum_{r=1}^{20} X_r &= 10(-2)+10(7)\\[2pt]
\displaystyle\sum_{r=1}^{20} X_r &= 50\\[2pt]
\end{align*}
\noindent\textbf{Question 14}\\[2pt]
A sequence is defined by the recurrence relation $Y_{n+1}=5+5Y_n-2(Y_n)^3$, given that  $Y_1 =2$, find $Y_{1000}$.\\[4pt]
\noindent\textbf{Solution 14}\\[2pt]
\begin{align*}
Y_2&=5+5Y_1-2(Y_1)^3=5+5(2)-2(2)^3=-1\\[7pt]
Y_3&=5+5Y_2-2(Y_2)^3=5+5(-1)-2(-1)^3=2\\[7pt]
Y_4&=5+5Y_3-2(Y_3)^3=5+5(2)-2(2)^3=-1\\[7pt]
\end{align*}
\begin{align*}
&Y_1&&Y_2&\,\,\,&Y_3&&Y_4&\,\,\,&Y_5&&Y_6&\\[2pt]
&-1&&\,\,2&&-1&&\,\,2&&-1&&\,\,2&\\
\end{align*}
We can see that $Y_2=Y_4=Y_6=Y_{8}=...=2$$\\[2pt]$
Every numbered term divisible by $2$ is $2$$\\[2pt]$
Find a numbered term that is close to $Y_{1000}$ that is divisible by 2$\\[2pt]$
$Y_2=2\quad Y_4=2\quad Y_{100}=2\quad Y_{1000}=2$$\\[2pt]$\\[4pt]
\noindent\textbf{Question 15}\\[2pt]
Given $\displaystyle\sum_{r=1}^{n} x_r = 5n^2-3$, find $\displaystyle\sum_{r=1}^{7} x_r$\\[4pt]
\noindent\textbf{Solution 15}\\[2pt]
\begin{align*}
\displaystyle\sum_{r=1}^{7} x_r&=5(7)^2-3=242\\[2pt]
\end{align*}
\noindent\textbf{Question 16}\\[2pt]
A sequence is defined by the recurrence relation $U_{n+1}=\displaystyle\frac{13-5U_n}{7-3U_n}$, given that  $U_1 =1$, find $U_{50}$.\\[4pt]
\noindent\textbf{Solution 16}\\[2pt]
\begin{align*}
U_2&=\displaystyle\frac{13-5U_1}{7-3U_1}=\displaystyle\frac{13-5(1)}{7-3(1)}=\displaystyle\frac{8}{4}=2\\[7pt]
U_3&=\displaystyle\frac{13-5U_2}{7-3U_2}=\displaystyle\frac{13-5(2)}{7-3(2)}=\displaystyle\frac{3}{1}=3\\[7pt]
U_4&=\displaystyle\frac{13-5U_3}{7-3U_3}=\displaystyle\frac{13-5(3)}{7-3(3)}=\displaystyle\frac{-2}{-2}=1\\[7pt]
U_5&=\displaystyle\frac{13-5U_4}{7-3U_4}=\displaystyle\frac{13-5(1)}{7-3(1)}=\displaystyle\frac{8}{4}=2\\
\end{align*}
\begin{align*}
&U_1&&U_2&&U_3&&U_4&&U_5&&U_6&\\[2pt]
&1&&2&&3&&1&&2&&3&\\
\end{align*}
We can see that $U_3=U_6=U_9=U_{12}=...=3$$\\[2pt]$
Every numbered term divisible by $3$ is 3$\\[2pt]$
Find a numbered term that is close to $U_{50}$ that is divisible by 3$\\[2pt]$
$U_3=3\quad U_9=3\quad U_{30}=3\quad U_{51}=3$$\\[2pt]$
$U_{51}=3\quad\Rightarrow \quad U_{50}=2$ since 2 is the term before 3 in the sequence$\\[2pt]$ i.e. $1,2,3,1,2,3,1,$$\\$\\[4pt]
\noindent\textbf{Question 17}\\[2pt]
Given $\displaystyle\sum_{r=1}^{n} a_r = 2n^3+5$, find $a_2$\\[4pt]
\noindent\textbf{Solution 17}\\[2pt]
\begin{align*}
\displaystyle\sum_{r=1}^{n} a_r &= 2n^3+5\\[2pt]
a_2&=\displaystyle\sum_{r=1}^{2} a_r - \displaystyle\sum_{r=1}^{1} a_r \\[2pt]
a_2&= 2(2)^3+5- (2(1)^3+5) \\[2pt]
a_2&= 21- 7 \\[2pt]
a_2&= 14 \\[2pt]
\end{align*}
\noindent\textbf{Question 18}\\[2pt]
Given $\displaystyle\sum_{r=1}^{n} U_r = 6n^2+11$, find $U_1$\\[4pt]
\noindent\textbf{Solution 18}\\[2pt]
\begin{align*}
\displaystyle\sum_{r=1}^{1} U_r &=U_1=6(1)^2+11=17
\end{align*}
\noindent\textbf{Question 19}\\[2pt]
Given $\displaystyle\sum_{r=1}^{n} u_r = n^3+4$, find $\displaystyle\sum_{r=1}^{5} u_r$\\[4pt]
\noindent\textbf{Solution 19}\\[2pt]
\begin{align*}
\displaystyle\sum_{r=1}^{5} u_r &= (5)^3+4=129\\[2pt]
\end{align*}
\noindent\textbf{Question 20}\\[2pt]
Given $\displaystyle\sum_{r=1}^{n} Y_r = 3n^3-2$, find $Y_3$\\[4pt]
\noindent\textbf{Solution 20}\\[2pt]
\begin{align*}
Y_3&=\displaystyle\sum_{r=1}^{3} - \displaystyle\sum_{r=1}^{2}\\[2pt]
Y_3&=3(3)^3-2 - (3(2)^3-2)\\[2pt]
Y_3&=57
\end{align*}
\noindent\textbf{Question 21}\\[2pt]
Given $\displaystyle\sum_{r=1}^{n} U_r = 3n+7$, find $U_5$\\[4pt]
\noindent\textbf{Solution 21}\\[2pt]
\begin{align*}
U_5&=\displaystyle\sum_{r=1}^{5} U_r - \displaystyle\sum_{r=1}^{4} U_r\\[2pt]
U_5&=3(5)+7 - (3(4)+7)\\[2pt]
U_5&=3\\
\end{align*}
\noindent\textbf{Question 22}\\[2pt]
A sequence is defined by the recurrence relation $U_{n+1}=kU_n-4, U_1=3, k>0$, given that $U_3=0$ find the value of $k$\\[4pt]
\noindent\textbf{Solution 22}\\[2pt]
\begin{align*}
U_2&=kU_1-4\\[2pt]
U_2&=3k-4\\[12pt]
U_3&=kU_2-4\\[2pt]
U_3&=k(3k-4)-4\\[2pt]
U_3&=3k^2-4k-4\quad U_3=0\\[12pt]
0&=3k^2-4k-4\hspace{25pt}S=-4\quad P=-12\\[2pt]
0&=\left(k+\frac{2}{3}\right)(k-2) \hspace{13pt}(2,-6) \quad \Rightarrow \quad \left(\displaystyle\frac{2}{3},-2\right)\\[2pt]
k&=2
\end{align*}
\noindent\textbf{Question 23}\\[2pt]
A sequence is defined by the recurrence relation $a_{n+1}=\displaystyle\frac{a_n}{k}+3, a_1=3,k>0$, given that $a_3=9$ find the value of $k$\\[4pt]
\noindent\textbf{Solution 23}\\[2pt]
\begin{align*}
a_2&=\displaystyle\frac{a_1}{k}+3\\[2pt]
a_2&=\displaystyle\frac{3}{k}+3\\[12pt]
a_3&=\displaystyle\frac{a_2}{k}+3\\[2pt]
a_3&=\displaystyle\frac{\left(\displaystyle\frac{3}{k}+3\right)}{k}+3\\[2pt]
a_3&=\displaystyle\frac{3}{k^2}+\frac{3}{k}+3\quad a_3=9\\[12pt]
9&=\displaystyle\frac{3}{k^2}+\frac{3}{k}+3\\[2pt]
6-\displaystyle\frac{3}{k^2}-\frac{3}{k}&=0\\[2pt]
6k^2-3k-3&=0\\[2pt]
2k^2-k-1&=0\hspace{20pt}S=-1\quad P=-2\\[2pt]
\left(x+\displaystyle\frac{1}{2}\right)(k-1)&=0\hspace{20pt} (1,-2) \quad \Rightarrow \quad \left(\displaystyle\frac{1}{2},-1\right)\\[2pt]
k&=1\\
\end{align*}
\noindent\textbf{Question 24}\\[2pt]
A sequence is defined by the recurrence relation $u_{n+1}=\sqrt{a}\left(u_n-\displaystyle\frac{1}{b}\right),5 u_1=4$, given that $u_2=7$ and $u_3=13$ find the value of $a$ and $b$ .\\[4pt]
\noindent\textbf{Solution 24}\\[2pt]
\begin{align*}
u_2&=\sqrt{a}\left(u_1-\displaystyle\frac{1}{b}\right)\\[2pt]
7&=\sqrt{a}\left(4-\displaystyle\frac{1}{b}\right)\hspace{20pt}(1)\\[2pt]
7&=4\sqrt{a}-\displaystyle\frac{\sqrt{a}}{b}\hspace{20pt}(2)\\[12pt]
u_3&=\sqrt{a}\left(u_2-\displaystyle\frac{1}{b}\right)\\[2pt]
13&=\sqrt{a}\left(7-\displaystyle\frac{1}{b}\right)\\[2pt]
13&=7\sqrt{a}-\displaystyle\frac{\sqrt{a}}{b}\hspace{20pt}(3)\\[12pt]
(3)-(2)\quad13-7&=7\sqrt{a}-\displaystyle\frac{\sqrt{a}}{b}-\left(4\sqrt{a}-\displaystyle\frac{\sqrt{a}}{b}\right)\\[2pt]
6&=3\sqrt{a}\\[2pt]
2&=\sqrt{a}\\[2pt]
a&=4\\[12pt]
\text{Sub into} \,\,(1)\quad 7&=\sqrt{4}\left(4-\displaystyle\frac{1}{b}\right)\\[2pt]
\displaystyle\frac{7}{2}&=4-\displaystyle\frac{1}{b}\\[2pt]
-\displaystyle\frac{1}{2}&=-\displaystyle\frac{1}{b}\\[2pt]
b&=2
\end{align*}
\noindent\textbf{Question 25}\\[2pt]
A sequence is defined by the recurrence relation $a_{n+1}=3-a_n$, given that  $a_1 =1$, find $\displaystyle\sum_{r=1}^{100} a_r$.\\[4pt]
\noindent\textbf{Solution 25}\\[2pt]
\begin{align*}
a_2&=3-a_1=3-1=2\\[7pt]
a_3&=3-a_2=3-2=1\\[7pt]
a_4&=3-a_3=3-1=2\\[7pt]
a_5&=3-a_4=3-2=1\\[7pt]
\displaystyle\sum_{r=1}^{100} a_r &= a_1+a_2+a_3+a_4+...+a_{100}\\[2pt]
\displaystyle\sum_{r=1}^{100} a_r &= 1+2+1+2+1+2+...+2\\[2pt]
\displaystyle\sum_{r=1}^{100} a_r &= 50(2)+50(1)\\[2pt]
\displaystyle\sum_{r=1}^{100} a_r &= 150\\[2pt]
\end{align*}
\noindent\textbf{Question 26}\\[2pt]
A sequence is defined by the recurrence relation $A_{n+1}=\displaystyle\frac{4A_n-16}{3A_n-8}$, given that  $A_1 =0$, find $A_{100}$.\\[4pt]
\noindent\textbf{Solution 26}\\[2pt]
\begin{align*}
A_2&=\displaystyle\frac{4A_1-16}{3A_1-8}=\displaystyle\frac{4(0)-16}{3(0)-8}=\displaystyle\frac{-16}{-8}=2\\[7pt]
A_3&=\displaystyle\frac{4A_2-16}{3A_2-8}=\displaystyle\frac{4(2)-16}{3(2)-8}=\displaystyle\frac{-8}{-2}=4\\[7pt]
A_4&=\displaystyle\frac{4A_1-16}{3A_1-8}=\displaystyle\frac{4(4)-16}{3(4)-8}=0\\[7pt]
A_5&=\displaystyle\frac{4A_1-16}{3A_1-8}=\displaystyle\frac{4(0)-16}{3(0)-8}=\displaystyle\frac{-16}{-8}=2\\
\end{align*}
$a_1\quad a_2\quad a_3\quad a_4\quad a_5\quad a_6$ $\\[2pt]$
$0\hspace{16pt} 2\hspace{16pt} 4\hspace{14pt} 0\hspace{16pt} 2\hspace{14pt} 4$  $\\[2pt]$
We can see that $a_3=a_6=a_9=a_{12}=...=4$$\\[2pt]$
Every numbered term divisible by $3$ is 4$\\[2pt]$
Find a numbered term that is close to $a_{100}$ that is divisible by 3$\\[2pt]$
$a_3=4\hspace{9.5pt} a_9=4\hspace{9.5pt} a_{30}=4\hspace{9.5pt} a_{99}=4$$\\[2pt]$
$a_{99}=4\hspace{9.5pt} \Rightarrow \hspace{9.5pt} a_{100}=0$ since 0 is the next term after 4 in the sequence$\\[2pt]$ i.e. $0,2,4,0,2,4,0$$\\$\\[4pt]
\noindent\large{\textbf{Lesson 4 Arithmetic Sequence 2}}\\[12pt]
\noindent\textbf{Question 1}\\[2pt]
$U_n$ is an arithmetic sequence with $S_n$ being the sum of the first n terms of the sequence. Given that $U_4=11$ and $U_7=23$, find $S_{11}$\\[4pt]
\noindent\textbf{Solution 1}\\[2pt]
\begin{align*}
U_n&=a+(n-1)d\\[2pt]
U_4&=a+(4-1)d=a+3d=11\quad (1)\\[2pt]
U_7&=a+(7-1)d=a+6d=23\quad (2)\\[2pt]
(2)-(1)\quad a+6d-(a+3d)&=23-11\\[2pt]
3d&=12\\[2pt]
d&=4\\[2pt]
\text{Sub into} \quad (1) \quad a+3(4)&=11\\[2pt]
a&=-1\\[12pt]
S_n&=\displaystyle\frac{n}{2}(2(a)+(n-1)d)\\[2pt]
S_{11}&=\displaystyle\frac{11}{2}(2(-1)+(11-1)4)=209\\
\end{align*}
\noindent\textbf{Question 2}\\[2pt]
$U_n$ is an arithmetic sequence with $S_n$ being the sum of the first n terms of the sequence. Given that $U_3=-5$ and $U_5=-11$, find $S_{7}$\\[4pt]
\noindent\textbf{Solution 2}\\[2pt]
\begin{align*}
U_n&=a+(n-1)d\\[2pt]
U_3&=a+(3-1)d=a+2d=-5\quad (1)\\[2pt]
U_5&=a+(5-1)d=a+4d=-11\quad (2)\\[2pt]
(2)-(1)\quad a+4d-(a+2d)&=-11-(-5)\\[2pt]
2d&=-6\\[2pt]
d&=-3\\[2pt]
\text{Sub into} \quad (1) \quad a+2(-3)&=-5\\[2pt]
a&=1\\[12pt]
U_{7}&=1+(7-1)(-3)=-17\\[2pt]
S_{7}&=\displaystyle\frac{7}{2}(1+(-17))=-56\\
\end{align*}
\noindent\textbf{Question 3}\\[2pt]
The first three terms of an arithmetic sequence are $60,58,56...$, there exists a $k^{\text{th}}$ term which $=0$, find the value of $k$, hence of otherwise find the maximum value of $S_n$\\[4pt]
\noindent\textbf{Solution 3}\\[2pt]
\begin{align*}
U_n&=a+(n-1)d\\[2pt]
U_k&=60+(k-1)(-2)=0\\[2pt]
k-1&=30\\[2pt]
k&=31\\[12pt]
\text{maimum value of} \,\,S_n&=S_k\,\, \text{as any term after}\,\, U_k\,\, \text{is negative}\\[2pt]
S_n&=\displaystyle\frac{n}{2}(a+d)\\[2pt]
S_k&=\displaystyle\frac{31}{2}(60+0)\\[2pt]
S_k&=930\\[-80pt]
\end{align*}
\noindent\textbf{Question 4}\\[2pt]
$U_n$ is an arithmetic sequence with $S_n$ being the sum of the first n terms of the sequence. Given that $S_{5}=85$ and $S_8=184$, find $U_{6}$\\[4pt]
\noindent\textbf{Solution 4}\\[2pt]
\begin{align*}
S_n&=\displaystyle\frac{n}{2}(2a+(n-1)d)\\[2pt]
S_5&=\displaystyle\frac{5}{2}(2a+(5-1)d)=85\\[2pt]
2a+4d&=34\quad (1)\\[2pt]
S_8&=\displaystyle\frac{8}{2}(2a+(8-1)d)=184\\[2pt]
2a+7d&=46 \quad (2)\\[2pt]
(2)-(1) \quad 2a+7d-(2a+4d)&=46-34\\[2pt]
3d&=12\\[2pt]
d&=4\\[2pt]
\text{Sub into}\quad (1) \quad 2a+4(4)&=34\\[2pt]
a&=9\\[12pt]
U_6&=a+(6-1)d=9+5(4)=29
\end{align*}
\noindent\textbf{Question 5}\\[2pt]
$U_n$ is an arithmetic sequence with $S_n$ being the sum of the first n terms of the sequence. Given that $U_{5}=19$ and $S_{10}=170$, find $U_{4}$\\[4pt]
\noindent\textbf{Solution 5}\\[2pt]
\begin{align*}
U_n&=a+(n-1)d\\[2pt]
U_5&=a+4d=19 \hspace{62pt} (1)\\[2pt]
S_n&=\displaystyle\frac{n}{2}(2a+(n-1)d)\\[2pt]
S_{10}&=\displaystyle\frac{10}{2}(2a+(10-1)d)=170\\[2pt]
2a+9d&=34\hspace{103pt} (2)\\[2pt]
(2)-2(1) \quad 2a+9d-2(a+4d)&=34-38\\[2pt]
d&=-4\\[2pt]
d&=-4\\[2pt]
\text{Sub into}\quad (1) \quad a+4(-4)&=19\\[2pt]
a&=35\\[12pt]
U_4&=a+(4-1)d=35+(3)(-4)=23
\end{align*}
\noindent\textbf{Question 6}\\[2pt]
$U_n$ is an arithmetic sequence with $S_n$ being the sum of the first n terms of the sequence. Given that $U_{4}=8$ and $S_{12}=0$, find $S_{9}$\\[4pt]
\noindent\textbf{Solution 6}\\[2pt]
\begin{align*}
U_n&=a+(n-1)d\\[2pt]
U_4&=a+3d=8 \hspace{62pt} (1)\\[2pt]
S_n&=\displaystyle\frac{n}{2}(2a+(n-1)d)\\[2pt]
S_{12}&=\displaystyle\frac{12}{2}(2a+(12-1)d)=0\\[2pt]
2a+11d&=0\hspace{103pt} (2)\\[2pt]
(2)-2(1) \quad 2a+11d-2(a+3d)&=0-16\\[2pt]
8d&=-16\\[2pt]
d&=-2\\[2pt]
\text{Sub into}\quad (1) \quad a+3(-2)&=8\\[2pt]
a&=14\\[12pt]
S_9&=\displaystyle\frac{9}{2}(2(14)+(9-1)(-2))=54
\end{align*}
\noindent\textbf{Question 7}\\[2pt]
$U_n$ is an arithmetic sequence with $S_n$ being the sum of the first n terms of the sequence. Given that $U_{3}=4$ and $U_{7}=0$, find $S_{10}$\\[4pt]
\noindent\textbf{Solution 7}\\[2pt]
\begin{align*}
U_n&=a+(n-1)d\\[2pt]
U_3&=a+2d=4 \quad (1)\\[2pt]
U_7&=a+6d=0 \quad (2)\\[2pt]
(2)-(1) \quad a+6d-(a+2d)&=0-4\\[2pt]
4d&=-4\\[2pt]
d&=-1\\[2pt]
\text{Sub into}\quad (1) \quad a+2(-1)&=4\\[2pt]
a&=6\\[12pt]
S_n&=\displaystyle\frac{n}{2}(2a+(n-1)d)\\[2pt]
S_{10}&=\displaystyle\frac{10}{2}(2(6)+(10-1)(-1))=15
\end{align*}
\noindent\textbf{Question 8}\\[2pt]
$U_n$ is an arithmetic sequence with $S_n$ being the sum of the first n terms of the sequence. Given that $U_{4}=10$ and $S_{6}=57$, find $S_{11}$\\[4pt]
\noindent\textbf{Solution 8}\\[2pt]
\begin{align*}
U_n&=a+(n-1)d\\[2pt]
U_4&=a+3d=10 \hspace{85pt} (1)\\[2pt]
S_n&=\displaystyle\frac{n}{2}(2a+(n-1)d)\\[2pt]
S_6&=\displaystyle\frac{6}{2}(2a+(5-1)d)=57\\[2pt]
a+2d&=\displaystyle\frac{19}{2}\hspace{124pt}(2)\\[2pt]
(1)-(2) \quad a+3d-(a+2d)&=10-\displaystyle\frac{19}{2}\\[2pt]
d&=\displaystyle\frac{1}{2}\\[2pt]
\text{Sub into}\quad (1) \quad a+3\left(\displaystyle\frac{1}{2}\right)&=10\\[2pt]
a&=\displaystyle\frac{17}{2}\\[12pt]
S_{11}&=\displaystyle\frac{11}{2}\left(2\left(\displaystyle\frac{17}{2}\right)+(11-1)\left(\displaystyle\frac{1}{2}\right)\right)=121
\end{align*}
\noindent\textbf{Question 9}\\[2pt]
Three consecutive terms in an arithmetic sequence are $3k+2,2k+5,4k+5$, find the value of $k$\\[4pt]
\noindent\textbf{Solution 9}\\[2pt]
\begin{align*}
2k+5-(3k+2)=d&=4k+5-(2k+5)\\[2pt]
-k+3&=2k\\[2pt]
3k&=3\\[2pt]
k&=1\\[-70pt]
\end{align*}
\noindent\textbf{Question 10}\\[2pt]
Three consecutive terms in an arithmetic sequence are $k^2+3,-k,k-1$, find the possible values of $k$\\[4pt]
\noindent\textbf{Solution 10}\\[2pt]
\begin{align*}
-k-(k^2+3)=d&=k-1-(-k)\\[2pt]
-k-k^2-3&=2k-1\\[2pt]
0&=k^2+3k+2\\[2pt]
0&=(k+2)(k+1)\\[-70pt]
\end{align*}
\noindent\textbf{Question 11}\\[2pt]
Three consecutive terms in an arithmetic sequence are $k+16,3k+12,7k-2$, find the value of $k$\\[4pt]
\noindent\textbf{Solution 11}\\[2pt]
\begin{align*}
3k+12-(k+16)=d&=7k-2-(3k+12)\\[2pt]
2k-4&=4k-14\\[2pt]
2k&=10\\[2pt]
k&=5\\[-70pt]
\end{align*}
\noindent\textbf{Question 12}\\[2pt]
The first three terms in an arithmetic sequence are $2k,k+9,3k$, find the smallest $n$ such that $S_n > 117$\\[4pt]
\noindent\textbf{Solution 12}\\[2pt]
\begin{align*}
k+9-2k&=d=3k-(k+9)\\[2pt]
-k+9&=2k-9\\[2pt]
3k&=18\\[2pt]
k&=6\\[12pt]
\Rightarrow\quad U_1&=12 \quad U_2=15 \quad U_3=18\\[2pt]
S_n&=\displaystyle\frac{n}{2}(2a+(n-1)d)\\[2pt]
\displaystyle\frac{n}{2}(2(12)+(n-1)3)&>117\\[2pt]
n(24+3n-3)&>234\\[2pt]
3n^2+21n-234&>0\\[2pt]
n^2+7n-78&>0\quad P=-78 \quad S=7\\[2pt]
(n+13)(n-6)&>0 \quad (13,-6)\\[2pt]
n&=6
\end{align*}
\noindent\textbf{Question 13}\\[2pt]
The first three terms of an arithmetic sequence are $99,96,93...$, there exists a $k^{\text{th}}$ term which $=0$, find the value of $k$, hence of otherwise find the maximum value of $S_n$\\[4pt]
\noindent\textbf{Solution 13}\\[2pt]
\begin{align*}
U_n&=a+(n-1)d\\[2pt]
U_k&=99+(k-1)(-3)=0\\[2pt]
k-1&=33\\[2pt]
k&=34\\[12pt]
\text{maimum value of} \,\,S_n&=S_k\,\, \text{as any term after}\,\, U_k\,\, \text{is negative}\\[2pt]
S_n&=\displaystyle\frac{n}{2}(a+l)\\[2pt]
S_k&=\displaystyle\frac{34}{2}(99+0)\\[2pt]
S_k&=1683
\end{align*}
\noindent\textbf{Question 14}\\[2pt]
The first three terms in an arithmetic sequence are $5,7,9$, find the smallest $n$ such that $S_n > 252$\\[4pt]
\noindent\textbf{Solution 14}\\[2pt]
\begin{align*}
S_n&=\displaystyle\frac{n}{2}(2a+(n-1)d)\\[2pt]
S_n&=\displaystyle\frac{n}{2}(2(5)+(n-1)2)>252\\[2pt]
n(5+n-1)&>252\\[2pt]
n^2+4n-252&>0 \quad P=-252\quad S=4\\[2pt]
(n+18)(n-14)&>0 \quad (18,-14)\\[2pt]
n&=14
\end{align*}
\noindent\textbf{Question 15}\\[2pt]
The first three terms in an arithmetic sequence are $9,12,15$, find the smallest $n$ such that $S_n > 750$\\[4pt]
\noindent\textbf{Solution 15}\\[2pt]
\begin{align*}
S_n&=\displaystyle\frac{n}{2}(2a+(n-1)d)\\[2pt]
S_n&=\displaystyle\frac{n}{2}(2(9)+(n-1)3)>750\\[2pt]
n(18+3n-3)&>1500\\[2pt]
3n(5+n)&>1500\\[2pt]
n(5+n)&>500\\[2pt]
n^2+5n-500&>0 \quad P=-500\quad S=5\\[2pt]
(n+25)(n-20)&>0 \quad (25,-20)\\[2pt]
n&=20
\end{align*}
\noindent\textbf{Question 16}\\[2pt]
The first three terms in an arithmetic sequence are $12,16,20,24$, find the smallest $n$ such that $S_n > 672$\\[4pt]
\noindent\textbf{Solution 16}\\[2pt]
\begin{align*}
S_n&=\displaystyle\frac{n}{2}(2a+(n-1)d)\\[2pt]
S_n&=\displaystyle\frac{n}{2}(2(12)+(n-1)4)>672\\[2pt]
n(12+2n-2)&>672\\[2pt]
2n^2+10n-672&>0\\[2pt]
n^2+5n-336&>0 \quad P=-336\quad S=5\\[2pt]
(n+21)(n-16)&>0 \quad (21,-16)\\[2pt]
n&=16
\end{align*}
\noindent\textbf{Question 17}\\[2pt]
Judith is playing with 294 sticks, she puts them in rows. The first row has 8 sticks, next row has 10 sticks, subsequent rows have 2 more sticks then the previous row. She has enough for $k$ rows but not enough for $k+1$ rows. Find k.\\[4pt]
\noindent\textbf{Solution 17}\\[2pt]
Sequence goes: 8,10,12,14,18,20....$\\[2pt]$
Not having enough for k+1 rows means that $S_k\leq294$
\begin{align*}
S_n&=\displaystyle\frac{n}{2}(2a+(k-1)d)\\[2pt]
S_k&=\displaystyle\frac{k}{2}(2(8)+(k-1)2)\\[2pt]
S_k&=k(8+k-1)\\[2pt]
S_k&=k(k+7)\\[2pt]
S_k&=k^2+7k \qquad (1)\\[12pt]
S_k&\leq 294 \\[2pt]
(1)\qquad k^2+7k& \leq 294\\[2pt]
k^2+7k-294&\leq 0\qquad P=294 \quad S=7\\[2pt]
(k+21)(k-14)&\leq 0 \qquad (21,-14)\\[2pt]
k&=14\\[-70pt]
\end{align*}
\noindent\textbf{Question 18}\\[2pt]
$U_n$ is an arithmetic sequence with $S_n$ being the sum of the first n terms of the sequence. Given that $S_{11}=0$ and $U_2=8$, find $U_{6}$\\[4pt]
\noindent\textbf{Solution 18}\\[2pt]
\begin{align*}
U_n&=a+(n-1)d\\[2pt]
U_2&=a+(2-1)d=a+d=8\hspace{59pt} (1)\\[2pt]
S_n&=\displaystyle\frac{n}{2}(2a+(n-1)d)\\[2pt]
S_{11}&=\displaystyle\frac{11}{2}(2a+(11-1)d)=0\\[2pt]
S_{11}&=a+5d=0\hspace{119pt} (2)\\[2pt]
(2)-(1)\quad a+5d-(a+d)&=0-8\\[2pt]
4d&=-8\\[2pt]
d&=-2\\[2pt]
\text{Sub into} \quad (1) \hspace{20pt} a+(-2)&=8\\[2pt]
a&=10\\[12pt]
U_{6}&=1+(7-1)(-2)=-11\\[2pt]
\end{align*}
\noindent\textbf{Question 19}\\[2pt]
The first three terms of an arithmetic sequence are $44,41,38...$, there exists a $k^{\text{th}}$ term which is the smallest positive term in the sequence, find the value of $k$, hence of otherwise find the maximum value of $S_n$\\[4pt]
\noindent\textbf{Solution 19}\\[2pt]
\begin{align*}
U_n&=a+(n-1)d\\[2pt]
U_k&=44+(k-1)(-3)=0\\[2pt]
k-1&=\displaystyle\frac{44}{3}\\[2pt]
k&=\displaystyle\frac{44}{3}+1=15.6\\[2pt]
k&=15\\[12pt]
\text{maimum value of} \,\,S_n&=S_k\,\, \text{as any term after}\,\, U_k\,\, \text{is negative}\\[2pt]
S_n&=\displaystyle\frac{n}{2}(2a+(n-1)d)\\[2pt]
S_k&=\displaystyle\frac{34}{2}(2(99)+(15-1)(-3))\\[2pt]
S_k&=2652\\[-90pt]
\end{align*}
\noindent\textbf{Question 20}\\[2pt]
At the start of the year 2000, Tony the farmer has $50m^2$ of land, he buys $7m^2$ of land at the end of each year. At the beginning of this year, Tony owns $141m^2$ of land. What year is it?\\[4pt]
\noindent\textbf{Solution 20}\\[2pt]
Sequence goes from the start of every year: 50,57,64,71,78,85....
\begin{align*}
U_n&=a+(n-1)d\\[2pt]
U_n&=141 \quad a=50 \quad d=7\\[2pt]
141&=50+(n-1)7\\[2pt]
n-1&=13\\[2pt]
n&=14\\[2pt]
\text{Year}&=2000+14=2014\\[-70pt]
\end{align*}
\\[2pt]
\noindent\huge{\textbf{Unit 2 Core 2}}\\[18pt]
\noindent\huge{\textbf{Chapter 1 Logarithms}}\\[15pt]
\noindent\large{\textbf{Lesson 1 Basic logarithms}}\\[12pt]
\noindent\textbf{Question 1}\\[2pt]
Express $\log_{x+5}10=4$ in power form\\[4pt]
\noindent\textbf{Solution 1}\\[2pt]
\begin{align*}
\log_{x+5}10&=4\\[2pt]
(x+5)^4&=10
\end{align*}
\noindent\textbf{Question 2}\\[2pt]
Express $\log_{a+b}6=c$ in power form\\[4pt]
\noindent\textbf{Solution 2}\\[2pt]
\begin{align*}
\log_{a+b}6&=c\\[2pt]
(a+b)^c&=6
\end{align*}
\noindent\textbf{Question 3}\\[2pt]
Express $\log_{xy}3=2$ in power form\\[4pt]
\noindent\textbf{Solution 3}\\[2pt]
\begin{align*}
\log_{xy}3&=2\\[2pt]
(xy)^2&=3
\end{align*}
\noindent\textbf{Question 4}\\[2pt]
Express $a^b=c$ in log form\\[4pt]
\noindent\textbf{Solution 4}\\[2pt]
\begin{align*}
a^b&=c\\[2pt]
\log_ac&=b
\end{align*}
\noindent\textbf{Question 5}\\[2pt]
Express $5^2=25$ in log form\\[4pt]
\noindent\textbf{Solution 5}\\[2pt]
\begin{align*}
5^2&=25\\[2pt]
\log_{5}25&=2
\end{align*}
\noindent\textbf{Question 6}\\[2pt]
Express $(xy)^5=20$ in log form\\[4pt]
\noindent\textbf{Solution 6}\\[2pt]
\begin{align*}
(xy)^5&=20\\[2pt]
\log_{xy}20&=5
\end{align*}
\noindent\textbf{Question 7}\\[2pt]
Express $\log_{2}(x^2y)-\log_{2}x$ as a single logarithm\\[4pt]
\noindent\textbf{Solution 7}\\[2pt]
\begin{align*}
&\log_{2}(x^2y)-\log_{2}x\\[2pt]
=&\log_{2}((x^2y) \div x)\\[2pt]
=&\log_{2}xy
\end{align*}
\noindent\textbf{Question 8}\\[2pt]
Express $a^{bc}=6$ in log form\\[4pt]
\noindent\textbf{Solution 8}\\[2pt]
\begin{align*}
a^{bc}&=6\\[2pt]
\log_{a}6&=bc
\end{align*}
\noindent\textbf{Question 9}\\[2pt]
Express $(a+b)^4=15$ in log form\\[4pt]
\noindent\textbf{Solution 9}\\[2pt]
\begin{align*}
(a+b)^4&=15\\[2pt]
\log_{(a+b)}15&=4
\end{align*}
\noindent\textbf{Question 10}\\[2pt]
Express $(x+4)^4=5$ in log form\\[4pt]
\noindent\textbf{Solution 10}\\[2pt]
\begin{align*}
(x+4)^4&=5\\[2pt]
\log_{(x+4)}5&=4
\end{align*}
\noindent\textbf{Question 11}\\[2pt]
Express $\log_{4}(x+y)+\log_{4}6$ as a single logarithm\\[4pt]
\noindent\textbf{Solution 11}\\[2pt]
\begin{align*}
&\log_{4}(x+y)+\log_{4}6\\[2pt]
=&\log_{4}((x+y) \,\, \text{x} \,\, 6)\\[2pt]
=&\log_{4}6(x+y)
\end{align*}
\noindent\textbf{Question 12}\\[2pt]
Express $\log_x9=2$ in power form\\[4pt]
\noindent\textbf{Solution 12}\\[2pt]
\begin{align*}
\log_x9&=2\\[2pt]
x^2&=9
\end{align*}
\noindent\textbf{Question 13}\\[2pt]
Express $3\log_{3}(a+b)+\log_{3}4$ as a single logarithm\\[4pt]
\noindent\textbf{Solution 13}\\[2pt]
\begin{align*}
&3\log_{3}(a+b)+\log_{3}4\\[2pt]
=&\log_{3}(a+b)^3+\log_{3}4\\[2pt]
=&\log_{3}((a+b)^3 \,\, \text{x} \,\, 4)\\[2pt]
=&\log_{3}4(a+b)^3\\[-80pt]
\end{align*}
\noindent\textbf{Question 14}\\[2pt]
Express $\log_{4}(a^2-b^2)-2\log_{4}(a+b)$ as a single logarithm\\[4pt]
\noindent\textbf{Solution 14}\\[2pt]
\begin{align*}
&\log_{4}(a^2-b^2)-2\log_{4}a+b\\[2pt]
=&\log_{4}(a^2-b^2)-\log_{4}(a+b)^2\\[2pt]
=&\log_{3}((a^2-b^2) \,\, \div \,\, (a+b)^2)\\[2pt]
=&\log_{3}\left(\displaystyle\frac{(a+b)(a-b)}{(a+b)^2}\right)\\[2pt]
=&\log_{3}\displaystyle\frac{a-b}{a+b}\\[-130pt]
\end{align*}
\noindent\textbf{Question 15}\\[2pt]
Express $\log_{x}(4a-6b)+\log_{x}\displaystyle\frac{1}{2}$ as a single logarithm\\[4pt]
\noindent\textbf{Solution 15}\\[2pt]
\begin{align*}
&\log_{x}(4a-6b)+\log_{x}\displaystyle\frac{1}{2}\\[2pt]
=&\log_{x}\displaystyle\frac{1}{2}(4a-6b)\\[2pt]
=&\log_{x}(2a-3b)
\end{align*}
\noindent\textbf{Question 16}\\[2pt]
Express $\log_{4}(6a)-\log_{4}(2a)$ as a single logarithm\\[4pt]
\noindent\textbf{Solution 16}\\[2pt]
\begin{align*}
&\log_{4}(6a)-\log_{4}(2a)\\[2pt]
=&\log_{4}(6a \,\, \div \,\, 2a)\\[2pt]
=&\log_{4}3
\end{align*}
\noindent\textbf{Question 17}\\[2pt]
Express $\log_{10}(15)-\log_{10}(3)$ as a single logarithm\\[4pt]
\noindent\textbf{Solution 17}\\[2pt]
\begin{align*}
&\log_{10}(15)-\log_{10}(3)\\[2pt]
=&\log_{10}(15 \,\, \div \,\, 3)\\[2pt]
=&\log_{10}5
\end{align*}
\noindent\textbf{Question 18}\\[2pt]
Express $3\log_{y}(5)+\log_{y}(4)$ as a single logarithm\\[4pt]
\noindent\textbf{Solution 18}\\[2pt]
\begin{align*}
&3\log_{y}(5)+\log_{y}(4)\\[2pt]
=&\log_{y}5^3+\log_{y}4\\[2pt]
=&\log_{y}(5^3 \,\, \text{x} \,\, 4)\\[2pt]
=&\log_{y}500
\end{align*}
\noindent\textbf{Question 19}\\[2pt]
Express $3\log_{a}(4)-4\log_{a}(2)$ as a single logarithm\\[4pt]
\noindent\textbf{Solution 19}\\[2pt]
\begin{align*}
&\log_{a}(4^3)-\log_{a}(2^4)\\[2pt]
=&\log_{a}(64)-\log_{a}(16)\\[2pt]
=&\log_{y}(64 \div 16)\\[2pt]
=&\log_{y}4
\end{align*}
\noindent\textbf{Question 20}\\[2pt]
Express $\log_{3}7=a+b^2$ in power form\\[4pt]
\noindent\textbf{Solution 20}\\[2pt]
\begin{align*}
\log_{3}7&=a+b^2\\[2pt]
3^{a+b^2}&=7
\end{align*}
\noindent\textbf{Question 21}\\[2pt]
Express $4\log_{9}5-2\log_{3}(15)$ as a single logarithm\\[4pt]
\noindent\textbf{Solution 21}\\[2pt]
\begin{align*}
&4\log_{9}5-2\log_{3}(9)\\[2pt]
=&4\left(\displaystyle\frac{\log_{3}5}{\log_{3}9}\right)-2\log_{3}(15)\\[2pt]
=&\left(\displaystyle\frac{4\log_{3}5}{2}\right)-\log_{3}(15^2)\\[2pt]
=&2\log_{3}5-\log_{3}(15^2)\\[2pt]
=&\log_{3}(5^2 \div 15^2)\\[2pt]
=&\log_{3}\displaystyle\frac{25}{225}\\[2pt]
=&\log_{3}\displaystyle\frac{1}{9}
\end{align*}
\noindent\textbf{Question 22}\\[2pt]
Express $\log_{a}4+\log_{a}5$ as a single logarithm\\[4pt]
\noindent\textbf{Solution 22}\\[2pt]
\begin{align*}
&\log_{a}4+\log_{a}5\\[2pt]
=&\log_{a}(4 \,\, \text{x} \,\, 5)\\[2pt]
=&\log_{a}20
\end{align*}
\noindent\textbf{Question 23}\\[2pt]
Express $2\log_{16}8-4\log_{4}(2)$ as a single logarithm\\[4pt]
\noindent\textbf{Solution 23}\\[2pt]
\begin{align*}
&2\log_{16}8-4\log_{4}(2)\\[2pt]
=&2\left(\displaystyle\frac{\log_{4}8}{\log_416}\right)-4\log_{4}(2)\\[2pt]
=&\left(\displaystyle\frac{2\log_{4}8}{2}\right)-\log_{4}(16)\\[2pt]
=&\log_{4}(8 \div 16)\\[2pt]
=&\log_{4}\displaystyle\frac{1}{2}
\end{align*}
\\[2pt]
\noindent\Huge{\textbf{OCR A Level Maths}}\\[5pt]
\noindent\large{To be added ......}\\[20pt]
\noindent\Huge{\textbf{AQA A Level Maths}}\\[5pt]
\noindent\large{To be added ......}\\[20pt]
\noindent\Huge{\textbf{MEI A Level Maths}}\\[5pt]
\noindent\large{To be added .....}\\[20pt]
\noindent\huge{\textbf{Unit 1 C 1}}\\[18pt]
\noindent\huge{\textbf{Chapter 1 asdfasfd}}\\[15pt]
\\[2pt]
\noindent\Huge{\textbf{Edexcel A Level Further Maths}}\\[5pt]
\noindent\large{To be added ......}\\[20pt]
\noindent\Huge{\textbf{MEI A Level Further Maths}}\\[5pt]
\noindent\large{To be added ......}\\[20pt]
\end{document}
